\begin{definition}[sublinear function]\rm\nextline
	Let $X$ be a vector space. \func{p}{X}{\real} is a sublinear function if the following holdsL
	\begin{itemize}
		\item $p(kx)=kp(x),$ for all $x\in X$ and $k\geq0$
		\item $p(x+y)=p(x)+p(y),\,\,\forall x,y\in X$
	\end{itemize}
\end{definition}

\begin{example}[sublinear function]\rm\nextline
    Any linear map is sublinear. $p(x)=\norm{x}_X$ on $X$ is also sublinear.
\end{example}

\begin{theorem}[Hahn-Banach]\rm\nextline
    Let $X$ be a normed space and $M\subset X$ a linear subspace. \func{p}{X}{\real} sublinear, \func{f}{M}{\real} linear with 
    $f(x)\leq p(x)\,\,\forall x\in M$.(*)
    Then there exists a linear map \func{F}{X}{\real} with $F|M=f$ and $F(x)\leq p(x)\,\,\forall x\in X$
\end{theorem}

\begin{remark}[induction]\nextline\rm
    \placeholder
    X
\end{remark}

% proof begins
    If $M=X$, simply choose $F=f$.\\
    Else, choose $x\not \in M$ and set $M_1=\{x+tx_1:x\in M,t\in\real\}$, which is a linear subspace of $X$. Idea is to extend $F$ to $M_1$ such that  (*) holds on $M_1$ and then repeat this process.\\
    PART I\\
    For all $x,y\in M$:
    $$
    f(x)+f(y)
    \stackrel{f\,linear}{=}f(x+y)
    \stackrel{ii)}{\leq}p(x-x_1)+p(x_1+y)
    $$
    Hence
    $$
    (**)\forall x,y,\in M:\quad f(x)-p(x-x_1)\leq p(x_1+y)-f(y)
    $$
    Take supremum, y fixed, get 
    $$
\alpha\stackrel{def}{=}\sup_{x\in M}\left(f(x)-p(x-x_1)\right)<\infty
    $$
    Define \func{f_1}{M_1}{\real} by 
    $$f_1(x+tx_1)=f(x)+t\alpha,\,\,x\in M,\,t\in \real$$
    \begin{lemma}
        \begin{itemize}
            \item $f_1$ is linear 
            \item $f|M=f$
            \item $f_1(x)\leq p(x)\,\forall x\in M_1$ 
        \end{itemize}
    \end{lemma}
    1) and 2) follows from checking definition.\\
    proof of 3):
    By definition of $\alpha$,  $\forall x\in M$ : $f(x)=\alpha\leq f(x)-(f(x)-p(x-x_1))=p(x-x_1)$. On the other hand, taking supremum over $x$ in (**) yields $\forall y \in M$ : $f(y)+\alpha\leq p(y+x_1)$. Overall:
    $$
    \forall x\in M:\quad f_1(x\pm x_1)=f(x)\pm \alpha\leq p(x\pm x_1)
    $$
    Apply with $t^{-1}(\in M)$ for $t>0$, in place of $x$ and multiply both side by $t$ to find
    $$
    \forall x\in M:\quad f_1(x\pm x_1)\leq  tp(t^{-1}x\pm x_1)\stackrel{i)}{p(x\pm x_1)}
    $$

    NB: If $X$ has a countable basis ($e_i,\,i\geq 1
    $), then one can take $x_1=e_1$ and proceed by induction. For general cases, we use Zorn's lemma.

    II)\\
    % Zorn's lemma
    \begin{lemma}[Zorn's lemma]%\label{Zorn's Lemma}
        \rm\nextline
        If $P$ is a nonempty partially ordered set and that every totally ordered subset of $P$ has an upper bound , then $P$ has
        a maximal element.
    \end{lemma}

    Take 
    $$
P=\{(N,g):N\in X,\,linear,\, g:N\rightarrow\real,\,linear,\,g|_M=f,\,g\leq p\,\,on\,\, N\}
    $$
    and define
    $$
    (N,g)\leq(\sigma,h)\stackrel{def.}{\equiv}N\subset\sigma,\,h|_N=g
    $$
    Then $(P,\leq)$ is partially ordered, $(M,f)\in P$ so $P\not=\emptyset$. Assume $(N_i,g_i)_{i\in I}$ totally ordered. set $N=\cup_{i\in I}N_i$ and for $x\in N$, $g(x)=g_i(x)$ if $x\in N_i$.

    Then $(N,g)\in P$. Indeed $N\subset X$ is linear and $g$ is well-defined. To see this, consider $x\in N_i\cap N_k$, $N_i\subset N_k$, then  $g|_{N_i}=g_i$ have $g_i(x)=g_k(x)$. Also $g$ is linear with $g\leq $p on N. To see linearity, take $x,y\in N$ then $x\in N_i$ and $y \in N_k$ for some $i,k\in I$ and $N_i\subset N_k$ (reverse same). so, $x,y\in N_k$ and 
    $$
g(x+y)\stackrel{def}{=}g_k(x+y)\stackrel{g,lin}{=}g_k(x)+g_k(y)\stackrel{def}{=}g(x)+g(y)
    $$
    Similarly, one can check $g\leq p$ on $N$. (exercise)

    $(N,g)$ is an upper bound for $(N_i,g_i)_{i\in I}$ since $N_i\subset N$ and $g|_{N_i}=g_i$ by definition.

    So Zorn's lemma applies here and yields that $(P,\leq)$ has a maximal element $(N,g)\in P$. set $F=g$. By definition of $P$. all properties required of $F$ hold and $N=X$. For other wise we can apply Part I) to $(N,g)$ and find $(N_1,g_1)\in P$ with $(N,g)<(N_1,g_1)$,which violates maximality of $(N,g)$. $\square$
    
%--------proof ends---------

\begin{remark}
    There is a version of Hahn-Banach theorem for $X$ over $\comp$, where \func{p}{X}{\real} is called $\comp$-sublinear if \begin{itemize}
        \item $p(ax)=|a|p(x)\,\,\forall x\in X$ and $a\in \comp$
        \item ii) holds.
    \end{itemize}
    \placeholder
\end{remark}

From now on $X$ is linear space over $\real$ and endowed with a norm.

\subsection{Application of Hahn-Banach} 
\begin{corollary}[extending a linear functional]\rm\nextline
    $M\subset X$ linear, $f\in \dual M$. Then $\exists F\in \dual X$ with $F|_M=f$ and $\norm{F}_{\dual{X}}=\norm{f}_{\dual{X}}$
    \prf\\
    Define \func{p}{X}{\real} via $p(x)=\norm{x}_X\norm{f}_{\dual M}$
    $p$ is sublinear and $\forall x\in M$:

    $$
f(x)\leq|f(x)=\norm{x}_X\frac{|f(x)|}{\norm{x}_X}\leq\norm{x}_X\norm{f}_{\dual M}=p(x)
    $$
    Then apply Hahn-Banach theorem.
\end{corollary}

For $\dual x\in \dual X$ recall notation$ \dual x(x)=\inne{\dual x}{x}$, $x\in X$
\begin{theorem}
    $\forall x \in  X$, $\exists \dual x \in \dual X$, s.t. $\inne{\dual x}{x}=\norm{x}_X^2=\norm{\dual x}_{\dual X}^2$
    \prf:
    $M\stackrel{def}{span\{x\}}$ Define
  $f(t)=t\norm{x}^2_X\,\forall t\in \real$

    Then \func{f}{M}{\real} is linear and 
    $$
    \norm{f}_{\dual M}=\sup_{\norm{tx}_X\leq 1}|f(x)|=\norm{x}_X
    $$
    so $f\in \dual M$. Apply corollary to extend $f$ to $\dual x\stackrel{def}{=}F\in \dual X$ with $\norm{\dual x}_{\dual X}=\norm{f}_{\dual X}=\norm{x}_X$ and $\inne {\dual x}{x}=f(x)=\norm{x}^2_X. $$\square$

\end{theorem}
\begin{remark}
    The theorem gives dual characterization of norm.
\end{remark}

Using H-B, one can "separate" all sort of things.

\begin{proposition}[separating points by dual element]\rm\nextline
	Let  $x,y\in X$, $\exists l\in\dual{X}$ such that $l(x)\neq l(y)$.
	\begin{pf}{}{}
		Choose $l\in \dual{X}$ to be the dual functional of $(y-x)\in X$. \\Then $l(y-x)=\norm{y-x}^2_X>0\implies l(y)\neq l(x)$.
	\end{pf}
\end{proposition}



\begin{proposition}[separation from subspaces]\rm\nextline
    $M\subset X$ is a linear space, closed. Assume $x_0\not\in M$ such that $d\equiv dist(x_0,M)$, then $\exists \ell\in\dual{X}$ with $\ell|_M=0$ and $\norm{\ell}_\dual{X}=1$ and $\ell(x_0)=d$.
    \begin{pf}{}{}
        Let $M_0=\{xt+x_0:x\in M\}$. Define \func{f}{M_0}{\real} which is linear, $f(x+tx_0)=td$. Then apply corollary to obtain extension $l\stackrel{def}{=}F\in \dual X$ with $\norm{l}_{\dual X}=1$.$\square$
    \end{pf}
    \end{proposition}

    The proven theorem has a lot of milage. For instance:
    \begin{corollary}\rm\nextline
        Apply Thm wth $M=\{0\}$, $x_0=x/{\norm{x}_X}$ to recaer theorem p.34 with $\dual x\stackrel{def}{=}\norm{x}_Xl$
    \end{corollary}

        \begin{theorem}
            $\dual X$ separable $\implies$ $X$ separable.
            \\
            Proof uses theorem.
        \end{theorem}
In particular this gives the follwoing corollary:
\begin{corollary}
    $\dual{\ell^\infty}\not\cong \ell^1$
\end{corollary}

From theorem p.34 one gets a dual characterization of the norm
\begin{corollary}\rm\nextline
    i) 
    $$
    \forall x\in X:\quad\norm{x}_X=\sup_{\norm{\dual x}_{\dual X}\leq1}|\inne{\dual x}{x}|
    $$

     ii) 
     $$
     \forall {\dual x}\in {\dual X}:\quad\norm{{\dual x}}_{\dual X}=\sup_{\norm{x}_{X}\leq1}|\inne{\dual x}{x}|$$
Moreover, the supremum in i) is always attained.
\begin{pf}{}{}
    
\end{pf}

\end{corollary}

\begin{theorem}
    Let $X$,$Y$ be normed spaces and $A\in\mathcal{L}(X,Y)$. The dual operator \func{\dual A}{\dual Y}{\dual X} is bounded and $\norm{\dual A}=norm{A}$
\end{theorem}
\subsection{Zorn's lemma}
\begin{definition}[Partial Order]\rm \nextline
    A partial order on set $X$, is a binary relation, written generically $\leq$, satisfying following property.
    \begin{itemize}
        \item transitivity: if $a\leq b$ and $b\leq c$ then $a\leq c$
        \item reflexivity: $a\leq a$
        \item anti-symmetry: if $a\leq b$ and $b\leq a$ then $a=b$

    \end{itemize}
    If we also have that for any $a$ and $b$, either $a\leq b$ or $b\leq a$, then we say $\leq$ is a total order.

\end{definition}

\begin{definition}[Upper bound]\rm\nextline
    Let $X$ be a set partially ordered by $\leq$ and $Y\subset X$, we say an element $x\in X$ is an {\bf upper bound} of $Y$ if $y\leq x\,\,\forall y\in Y$.

\end{definition}

\begin{definition}[Maximal element]\rm\nextline
    Let $X$ be a set partially ordered by $\leq$ and $Y\subset X$. say $x\in X$ is a maximal element of $X$ if $x\leq m$ implies $m=x$.

\end{definition}
\begin{lemma}[Zorn's lemma]\label{Zorn's Lemma}\rm\nextline
    If $X$ is a nonempty partially ordered set with the
    property that every totally ordered subset of $X$ has an upper bound in $X$, then $X$ has
    a maximal element.
\end{lemma}