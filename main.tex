\documentclass{article}
\usepackage{geometry}
\geometry{left=3cm,right=3cm,top=2cm,bottom=2cm}
\usepackage[utf8]{inputenc}
\usepackage{amsmath}
\usepackage[framemethod=TikZ]{mdframed}
\usepackage{amsfonts}
\usepackage{mathrsfs}
\usepackage{comment}
\usepackage{enumerate}
\usepackage{xcolor}
\usepackage{titlesec}


\usepackage[hidelinks,backref]{hyperref}
\usepackage{cleveref}
\usepackage[most]{tcolorbox}
\titleformat*{\section}{\LARGE\bfseries}
\titleformat*{\subsection}{\Large\bfseries}
\titleformat*{\subsubsection}{\Large\bfseries}
\titleformat*{\paragraph}{\large\bfseries}
\titleformat*{\subparagraph}{\large\bfseries}
\title{Functional Analysis}
\author{RIF Galaxy }
\date{August 2022}

\begin{document}

\maketitle
%Remark\\
On this note:
This note is primarily made for personal study purpose. %One should realise that any part that is unnecessarily detailed are those trouble the author with subtle logical problem. He kept asking himself stupid questions on those details, resulting in detailed answers.\\
%The note is made prior to the course taking place,relevant materials to the lectures taking place at Imperial College London 3rd year FA course will be labelled with {\color{red}\bf\S\S\S},important for exam will be labelled by {\color{red}\bf***}
The note is made prior to the course taking place, and will be updated according to the lectures for Imperial College London 3rd year FA course(2022).

% General 
\newcommand{\nextline}{\hfill\break}
\newcommand{\nl}{\nextline}
\newcommand{\plz}{{\texorpdfstring{\color{red}\bf\S\S\S}{}}}
\newcommand{\plzaq}{{\texorpdfstring{\color{red}\bf***}{}}}
\newcommand{\placeholder}{{\color{red} NOOOOOOT COMPLEEEEEEET! COOOOOOOM BAAAAAACK!!!}}

% FA and LA
\newcommand{\inne}[2]{\left<{#1},{#2}\right>}
\newcommand{\norm}[1]{\left\|{#1}\right\|}
\newcommand{\hbs}{$\mathscr{H}$ }
\newcommand{\hbp}{\mathscr{H}}
\newcommand{\dual}[1]{{#1}^*}
\newcommand{\sequ}[1]{\left({#1}\right)_1^\infty}
\newcommand{\func}[3]{${#1}:{#2}\xrightarrow{}{#3}$}
\newcommand{\prf}{\textit{proof}:   }

\newcommand{\cgr}[3]{{#1}\equiv {#2} \,(\bmod \,\,{#3})}
\newcommand{\ncgr}[3]{{#1}\not\equiv {#2} \,(\bmod \,\,{#3})}

% Fields 
\newcommand{\real}{\mathbb{R}}
\newcommand{\comp}{\mathbb{C}}
\newcommand{\inte}{\mathbb{Z}}
\newcommand{\natu}{\mathbb{N}}





% Theorems
\newtheorem{example}{Example}[subsection]
\newtheorem{definition}[example]{Definition}
\newtheorem{proposition}[example]{Proposition}
\newtheorem{remark}[example]{Remark}
\newtheorem{theorem}[example]{Theorem}
\newtheorem{lemma}[example]{Lemma}
\newtheorem{corollary}[example]{Corollary}


\newtcbtheorem[no counter]{pf}{Proof}{
  enhanced,
  rounded corners,
  attach boxed title to top,
  colback=white,
  colframe=black!25,
  fonttitle=\bfseries,
  coltitle=black,
  boxed title style={
    rounded corners,
    size=small,
    colback=black!25,
    colframe=black!25,
  } 
}{prf}

\begin{comment}
\begin{pf}{Name}{Hyper}
    asdassad
\end{pf}
\end{comment}




\tableofcontents



\newpage
\section{List of content by week}
\subsection{Week 1}
\begin{itemize}
	\item Structure and arrangement of the course
	\item \hyperref[vector space defs]{Linear space, definition and examples}
	\item \hyperref[lp space1]{$L^p$ spaces, proof that it's a normed linear space}
	\item \hyperref[Minkowski-holder]{Minkowski's inequality}
	\item \hyperref[Hölder's inequality]{Hölder's inequality}
	\item \hyperref[banach space def]{Banach space, definition and method of checking its properties}
\end{itemize}
\subsection{Week2}
\begin{itemize}
	\item \hyperref[Metric linear space]{Metric linear space}
	\item \hyperref[Jensen's inquality]{Jensen's inquality and convex function}
	\item \hyperref[topology]{Topology: dense, separable space}
	\item \hyperref[separable space example]{Examples of separable spaces}
	\item \hyperref[Schauder basis]{Schauder basis, existence of it implies separability}
	\item \hyperref[Inner product]{Inner product, definition}
	\item \hyperref[Hilbert space def]{Hilbert space, definition}
	\item \hyperref[convexity]{Convex set}
	\item \hyperref[Nearest Point Property]{Nearest point property}
	      \item\hyperref[parallelogram]{Parallelogram law}
	\item \hyperref[ortho comp]{Orthogonal complement}
\end{itemize}

\subsection{Week3}
\begin{itemize}
	\item \hyperref[finite dimensional Banach]{Finite dimensional Banach Space}

	\item \hyperref[compactness]{Compactness, closedness and boundedness}
	\item \hyperref[compact unit balls]{Compactness and closed unit ball}
	\item \hyperref[continuity of LO]{Boundedness and continuity of Linear operator}
	\item \hyperref[operator norm]{Operator Norm}
	\item \hyperref[Linear Functionals]{\color{red}Content On linear functionals, helpful to understanding operator}
\end{itemize}

\subsection{Week4}
\begin{itemize}
	\item \hyperref[Riesz representation theory]{Riesz representation theory}
	\item \hyperref[dual space Hilbert]{Dual space of Hilbert space}
	\item \hyperref[dual space Banach]{Dual space of Banach space}
	\item \hyperref[lp dual]{Dual space of $\ell^p$ space}
	\item \hyperref[dual operator]{Dual Operator(Not finished!)}
\end{itemize}


\begin{comment}
\subsection{Week3}
\begin{itemize}
	\item \hyperref[]{}
	\item \hyperref[]{}
	\item \hyperref[]{}
	\item \hyperref[]{}
	\item \hyperref[]{}
	\item \hyperref[]{}
	\item \hyperref[]{}
	\item \hyperref[]{}
\end{itemize}
\end{comment}

\begin{definition}[sublinear function]\rm\nextline
	Let $X$ be a vector space. \func{p}{X}{\real} is a sublinear function if the following holdsL
	\begin{itemize}
		\item $p(kx)=kp(x),$ for all $x\in X$ and $k\geq0$
		\item $p(x+y)=p(x)+p(y),\,\,\forall x,y\in X$
	\end{itemize}
\end{definition}

\begin{example}[sublinear function]\rm\nextline
    Any linear map is sublinear. $p(x)=\norm{x}_X$ on $X$ is also sublinear.
\end{example}

\begin{theorem}[Hahn-Banach]\rm\nextline
    Let $X$ be a normed space and $M\subset X$ a linear subspace. \func{p}{X}{\real} sublinear, \func{f}{M}{\real} linear with 
    $f(x)\leq p(x)\,\,\forall x\in M$.(*)
    Then there exists a linear map \func{F}{X}{\real} with $F|M=f$ and $F(x)\leq p(x)\,\,\forall x\in X$
\end{theorem}

\begin{remark}[induction]\nextline\rm
    \placeholder
    X
\end{remark}

% proof begins
    If $M=X$, simply choose $F=f$.\\
    Else, choose $x\not \in M$ and set $M_1=\{x+tx_1:x\in M,t\in\real\}$, which is a linear subspace of $X$. Idea is to extend $F$ to $M_1$ such that  (*) holds on $M_1$ and then repeat this process.\\
    PART I\\
    For all $x,y\in M$:
    $$
    f(x)+f(y)
    \stackrel{f\,linear}{=}f(x+y)
    \stackrel{ii)}{\leq}p(x-x_1)+p(x_1+y)
    $$
    Hence
    $$
    (**)\forall x,y,\in M:\quad f(x)-p(x-x_1)\leq p(x_1+y)-f(y)
    $$
    Take supremum, y fixed, get 
    $$
\alpha\stackrel{def}{=}\sup_{x\in M}\left(f(x)-p(x-x_1)\right)<\infty
    $$
    Define \func{f_1}{M_1}{\real} by 
    $$f_1(x+tx_1)=f(x)+t\alpha,\,\,x\in M,\,t\in \real$$
    \begin{lemma}
        \begin{itemize}
            \item $f_1$ is linear 
            \item $f|M=f$
            \item $f_1(x)\leq p(x)\,\forall x\in M_1$ 
        \end{itemize}
    \end{lemma}
    1) and 2) follows from checking definition.\\
    proof of 3):
    By definition of $\alpha$,  $\forall x\in M$ : $f(x)=\alpha\leq f(x)-(f(x)-p(x-x_1))=p(x-x_1)$. On the other hand, taking supremum over $x$ in (**) yields $\forall y \in M$ : $f(y)+\alpha\leq p(y+x_1)$. Overall:
    $$
    \forall x\in M:\quad f_1(x\pm x_1)=f(x)\pm \alpha\leq p(x\pm x_1)
    $$
    Apply with $t^{-1}(\in M)$ for $t>0$, in place of $x$ and multiply both side by $t$ to find
    $$
    \forall x\in M:\quad f_1(x\pm x_1)\leq  tp(t^{-1}x\pm x_1)\stackrel{i)}{p(x\pm x_1)}
    $$

    NB: If $X$ has a countable basis ($e_i,\,i\geq 1
    $), then one can take $x_1=e_1$ and proceed by induction. For general cases, we use Zorn's lemma.

    II)\\
    % Zorn's lemma
    \begin{lemma}[Zorn's lemma]%\label{Zorn's Lemma}
        \rm\nextline
        If $P$ is a nonempty partially ordered set and that every totally ordered subset of $P$ has an upper bound , then $P$ has
        a maximal element.
    \end{lemma}

    Take 
    $$
P=\{(N,g):N\in X,\,linear,\, g:N\rightarrow\real,\,linear,\,g|_M=f,\,g\leq p\,\,on\,\, N\}
    $$
    and define
    $$
    (N,g)\leq(\sigma,h)\stackrel{def.}{\equiv}N\subset\sigma,\,h|_N=g
    $$
    Then $(P,\leq)$ is partially ordered, $(M,f)\in P$ so $P\not=\emptyset$. Assume $(N_i,g_i)_{i\in I}$ totally ordered. set $N=\cup_{i\in I}N_i$ and for $x\in N$, $g(x)=g_i(x)$ if $x\in N_i$.

    Then $(N,g)\in P$. Indeed $N\subset X$ is linear and $g$ is well-defined. To see this, consider $x\in N_i\cap N_k$, $N_i\subset N_k$, then  $g|_{N_i}=g_i$ have $g_i(x)=g_k(x)$. Also $g$ is linear with $g\leq $p on N. To see linearity, take $x,y\in N$ then $x\in N_i$ and $y \in N_k$ for some $i,k\in I$ and $N_i\subset N_k$ (reverse same). so, $x,y\in N_k$ and 
    $$
g(x+y)\stackrel{def}{=}g_k(x+y)\stackrel{g,lin}{=}g_k(x)+g_k(y)\stackrel{def}{=}g(x)+g(y)
    $$
    Similarly, one can check $g\leq p$ on $N$. (exercise)

    $(N,g)$ is an upper bound for $(N_i,g_i)_{i\in I}$ since $N_i\subset N$ and $g|_{N_i}=g_i$ by definition.

    So Zorn's lemma applies here and yields that $(P,\leq)$ has a maximal element $(N,g)\in P$. set $F=g$. By definition of $P$. all properties required of $F$ hold and $N=X$. For other wise we can apply Part I) to $(N,g)$ and find $(N_1,g_1)\in P$ with $(N,g)<(N_1,g_1)$,which violates maximality of $(N,g)$. $\square$
    
%--------proof ends---------

\begin{remark}
    There is a version of Hahn-Banach theorem for $X$ over $\comp$, where \func{p}{X}{\real} is called $\comp$-sublinear if \begin{itemize}
        \item $p(ax)=|a|p(x)\,\,\forall x\in X$ and $a\in \comp$
        \item ii) holds.
    \end{itemize}
    \placeholder
\end{remark}

From now on $X$ is linear space over $\real$ and endowed with a norm.

\subsection{Application of Hahn-Banach} 
\begin{corollary}[extending a linear functional]\rm\nextline
    $M\subset X$ linear, $f\in \dual M$. Then $\exists F\in \dual X$ with $F|_M=f$ and $\norm{F}_{\dual{X}}=\norm{f}_{\dual{X}}$
    \prf\\
    Define \func{p}{X}{\real} via $p(x)=\norm{x}_X\norm{f}_{\dual M}$
    $p$ is sublinear and $\forall x\in M$:

    $$
f(x)\leq|f(x)=\norm{x}_X\frac{|f(x)|}{\norm{x}_X}\leq\norm{x}_X\norm{f}_{\dual M}=p(x)
    $$
    Then apply Hahn-Banach theorem.
\end{corollary}

For $\dual x\in \dual X$ recall notation$ \dual x(x)=\inne{\dual x}{x}$, $x\in X$
\begin{theorem}
    $\forall x \in  X$, $\exists \dual x \in \dual X$, s.t. $\inne{\dual x}{x}=\norm{x}_X^2=\norm{\dual x}_{\dual X}^2$
    \prf:
    $M\stackrel{def}{span\{x\}}$ Define
  $f(t)=t\norm{x}^2_X\,\forall t\in \real$

    Then \func{f}{M}{\real} is linear and 
    $$
    \norm{f}_{\dual M}=\sup_{\norm{tx}_X\leq 1}|f(x)|=\norm{x}_X
    $$
    so $f\in \dual M$. Apply corollary to extend $f$ to $\dual x\stackrel{def}{=}F\in \dual X$ with $\norm{\dual x}_{\dual X}=\norm{f}_{\dual X}=\norm{x}_X$ and $\inne {\dual x}{x}=f(x)=\norm{x}^2_X. $$\square$

\end{theorem}
\begin{remark}
    The theorem gives dual characterization of norm.
\end{remark}

Using H-B, one can "separate" all sort of things.

\begin{proposition}[separating points by dual element]\rm\nextline
	Let  $x,y\in X$, $\exists l\in\dual{X}$ such that $l(x)\neq l(y)$.
	\begin{pf}{}{}
		Choose $l\in \dual{X}$ to be the dual functional of $(y-x)\in X$. \\Then $l(y-x)=\norm{y-x}^2_X>0\implies l(y)\neq l(x)$.
	\end{pf}
\end{proposition}



\begin{proposition}[separation from subspaces]\rm\nextline
    $M\subset X$ is a linear space, closed. Assume $x_0\not\in M$ such that $d\equiv dist(x_0,M)$, then $\exists \ell\in\dual{X}$ with $\ell|_M=0$ and $\norm{\ell}_\dual{X}=1$ and $\ell(x_0)=d$.
    \begin{pf}{}{}
        Let $M_0=\{xt+x_0:x\in M\}$. Define \func{f}{M_0}{\real} which is linear, $f(x+tx_0)=td$. Then apply corollary to obtain extension $l\stackrel{def}{=}F\in \dual X$ with $\norm{l}_{\dual X}=1$.$\square$
    \end{pf}
    \end{proposition}

    The proven theorem has a lot of milage. For instance:
    \begin{corollary}\rm\nextline
        Apply Thm wth $M=\{0\}$, $x_0=x/{\norm{x}_X}$ to recaer theorem p.34 with $\dual x\stackrel{def}{=}\norm{x}_Xl$
    \end{corollary}

        \begin{theorem}
            $\dual X$ separable $\implies$ $X$ separable.
            \\
            Proof uses theorem.
        \end{theorem}
In particular this gives the follwoing corollary:
\begin{corollary}
    $\dual{\ell^\infty}\not\cong \ell^1$
\end{corollary}

From theorem p.34 one gets a dual characterization of the norm
\begin{corollary}\rm\nextline
    i) 
    $$
    \forall x\in X:\quad\norm{x}_X=\sup_{\norm{\dual x}_{\dual X}\leq1}|\inne{\dual x}{x}|
    $$

     ii) 
     $$
     \forall {\dual x}\in {\dual X}:\quad\norm{{\dual x}}_{\dual X}=\sup_{\norm{x}_{X}\leq1}|\inne{\dual x}{x}|$$
Moreover, the supremum in i) is always attained.
\begin{pf}{}{}
    
\end{pf}

\end{corollary}

\begin{theorem}
    Let $X$,$Y$ be normed spaces and $A\in\mathcal{L}(X,Y)$. The dual operator \func{\dual A}{\dual Y}{\dual X} is bounded and $\norm{\dual A}=norm{A}$
\end{theorem}
\subsection{Zorn's lemma}
\begin{definition}[Partial Order]\rm \nextline
    A partial order on set $X$, is a binary relation, written generically $\leq$, satisfying following property.
    \begin{itemize}
        \item transitivity: if $a\leq b$ and $b\leq c$ then $a\leq c$
        \item reflexivity: $a\leq a$
        \item anti-symmetry: if $a\leq b$ and $b\leq a$ then $a=b$

    \end{itemize}
    If we also have that for any $a$ and $b$, either $a\leq b$ or $b\leq a$, then we say $\leq$ is a total order.

\end{definition}

\begin{definition}[Upper bound]\rm\nextline
    Let $X$ be a set partially ordered by $\leq$ and $Y\subset X$, we say an element $x\in X$ is an {\bf upper bound} of $Y$ if $y\leq x\,\,\forall y\in Y$.

\end{definition}

\begin{definition}[Maximal element]\rm\nextline
    Let $X$ be a set partially ordered by $\leq$ and $Y\subset X$. say $x\in X$ is a maximal element of $X$ if $x\leq m$ implies $m=x$.

\end{definition}
\begin{lemma}[Zorn's lemma]\label{Zorn's Lemma}\rm\nextline
    If $X$ is a nonempty partially ordered set with the
    property that every totally ordered subset of $X$ has an upper bound in $X$, then $X$ has
    a maximal element.
\end{lemma}
\input{Chapter/Linear space Preliminaries.tex}
\input{Chapter/Normed,Banach,Hilbert.tex}
\input{Chapter/Operator and functional.tex}
\newpage
\section{Hahn-Banach Theorem}\label{Hahn-Banach Theorem}
One of the most import results in functional analysis, {\bf Hahn-Banach theorem} is a theorem dealing with extending linear maps from a subspace to the whole space. The theorem says that any bounded linear functional defined on a subspace can be  extended to the whole space, while preserving the norm. The result does not rely on completeness of the space, so it's a result for all normed linear spaces. The proof of this theorem involves using a version of \textit{\bf axiom of choice (AC) }, Zorn's lemma. We shall review this lemma first.

\begin{definition}[Partial Order]\rm \nextline
	A partial order on set $X$, is a binary relation, written generically $\leq$, satisfying following property.
	\begin{itemize}
		\item transitivity: if $a\leq b$ and $b\leq c$ then $a\leq c$
		\item reflexivity: $a\leq a$
		\item anti-symmetry: if $a\leq b$ and $b\leq a$ then $a=b$

	\end{itemize}
	If we also have that for any $a$ and $b$, either $a\leq b$ or $b\leq a$, then we say $\leq$ is a total order.

\end{definition}

\begin{definition}[Upper bound]\rm\nextline
	Let $X$ be a set partially ordered by $\leq$ and $Y\subset X$, we say an element $x\in X$ is an {\bf upper bound} of $Y$ if $y\leq x\,\,\forall y\in Y$.

\end{definition}

\begin{definition}[Maximal element]\rm\nextline
	Let $X$ be a set partially ordered by $\leq$ and $Y\subset X$. say $x\in X$ is a maximal element of $X$ if $x\leq m$ implies $m=x$.

\end{definition}
\begin{lemma}[Zorn's lemma]\rm\nextline
	If $X$ is a nonempty partially ordered set with the
	property that every totally ordered subset of $X$ has an upper bound in $X$, then $X$ has
	a maximal element.
\end{lemma}

\begin{theorem}[Hahn-Banach Theorem]\rm\nextline
	Let $X$ be a normed vector space over $\mathbb F$ ($\mathbb{C}$ or $\mathbb{R}$), $Y$ is a proper subspace of $X$. If $T_0:Y\xrightarrow{}\mathbb{F}$ is a bounded linear functional, then there exists a bounded linear functional $T:X\xrightarrow{}\mathbb{F}$ satisfying:
	\begin{itemize}
		\item $T(y)=T_0(y)$ for all $y\in Y$
		\item $\norm{T}=\norm{T_0}$
	\end{itemize}
\end{theorem}
To prove the theorem, the idea is first to show that we can extend linear functional by one dimension, with induction to show that extension can be done to "arbitrarily high dimension". Then by using Zorn's lemma we show that such extension "reaches" every dimension of the space. We first provide real version of the theorem.

\begin{lemma}[one-dimensional extension]\label{ODEX}\rm\nextline
	Let $X$ be a normed vector space over $\mathbb F$ ($\mathbb{C}$ or $\mathbb{R}$), $Y_n$ is a proper subspace of $X$. Let $v\in X\backslash Y$, $X_{n+1}=\{x+hv:x\in X_n,\,h\in\mathbb{C}\}$
	. If $T_n:Y\xrightarrow{}\mathbb{F}$ is a bounded linear functional, then there exists a bounded linear functional $T_{n+1}:X_{n+1}\xrightarrow{}\mathbb{F}$ satisfying:
	\begin{itemize}
		\item $T_{n+1}(x)=T_n(x)$ for all $x\in X_n$
		\item $\norm{T_{n+1}}=\norm{T_n}$
	\end{itemize}
	\begin{pf}{One-dimensional extension}{}
		Define linear functional $P:X_{n+1}\xrightarrow{}\mathbb{R}$ by
		$$
			P(x+kv)=T_n(x)-Ck,\,\forall\,x\in X_n,\,k\in\mathbb{R}
		$$
		where C is a constant to be determined.
		First we shall check linearity, which is left as an exercise.
		Then we shall show that we can find a proper constant $C$ so that $\norm{P}=\norm{T_n}$. Note that $X_n\subset X_{n+1}$, so we have
		\begin{equation}
			\begin{split}
				\norm{P}&=\sup_{x\in X_{n+1}}(\{|Px|:\norm{x}=1\})\\
				&\geq\sup_{x\in X_{n}}(\{|Px|:\norm{x}=1\})\\
				&=\sup_{x\in X_{n}}(\{|T_nx|:\norm{x}=1\})\\
				&=1
			\end{split}
		\end{equation}
		So by choosing $C$ such that $P(x+kv)\leq \norm{x+kv}$ for any $x\in X_n$ and $k\in \mathbb{R}$, we will have that $\norm{P}\leq 1$, giving $\norm{P}=1$. Thus it remains to show that we can find such a constant $C$.\\
		We aim to find $C$ such that
		$$
			|P(x+kv)|=|T_n(x)-Ck|\leq \norm{x+kv},\,\forall x\in X_n,\,\forall k\in\mathbb{R}
		$$
		Hence,
		$$
			T_n(x)-\norm{x+kv}\leq Ck\leq T_n(x)+\norm{x+kv},\,\forall x\in X_n,\,\forall k\in\mathbb{R}
		$$
		Note that for all $x,y\in X_n$ we have:
		\begin{equation}
			\begin{split}
				T_nx-T_ny&=T_n(x-y)\\
				&\leq \norm{x-y}\\
				&=\norm{(x+kv)-(kv+y)}\\
				&\leq\norm{x+kv}+\norm{y+kv}
			\end{split}
		\end{equation}
		Thus
		$$
			l^-=\sup_{x\in X_n,k\in\mathbb{R}}(T_n(x)-\norm{x+kv})\leq  \inf_{x\in X_n,k\in\mathbb{R}}(T_n(x)+\norm{x+kv})=l^+
		$$
		Hence we can always find a $C$ such that
		$$
			T_n(x)-\norm{x+kv}\leq l^-\leq Ck\leq l^+ \leq T_n(x)+\norm{x+kv},\,\forall x\in X_n,\,\forall k\in\mathbb{R}
		$$
		Which finishes the proof.
	\end{pf}
\end{lemma}
\subsection{Proof of Hahn-Banach theorem,real case}
Starting from $T_0:Y\xrightarrow{}\mathbb{R}$, by \ref{ODEX} we can define  $T_{n+1}$ to be the one-dimensional extension of $T_n$ for any $n\in\mathbb{N}$, with domain $Y_{n+1}$ extended from $Y_n$, for convenience we let$Y_0=Y$.
Then consider the set
$$
	M=
	\left\{
	(T_n,Y_n),n\in\mathbb{N}
	\right\}
$$
which can be partially ordered by $\leq$ defined as
$$
	(T_a,Y_a)\leq(T_b,Y_b)\,\text{ if } Y_a\subset Y_b,\,\text{ and } T_b=T_a\, \text{ on }\, Y_a
$$
Now let $S=\{(T_i,Y_i),i\in I\}$ (where $I$ is the index set) be a totally ordered subset of $M$. Consider $Y'=\cup_{i\in I}Y_i$ with $T'(x)=T_i(x)$ if $x\in Y_i$, we have that $(T',Y')\in M$ is an upper bound of $S$. By Zorn's lemma, we know that $M$ has a maximal element, denoted as $(T_\infty,Y_\infty)$. We claim that $Y_\infty=X$, because if not, we can do one-dimensional extension to $Y_\infty$, resulting in $X\subset Y_\infty+1$, contradicting with maximality. Thus we have $Y_\infty=X$, and  $T_\infty$ is the desired extension to $X$.

\subsection{Proof of Hahn-Banach theorem,complex case}
To prove the statement for complex case, we shall exploit a connection between real valued functional and complex one.\\
\begin{proposition}\rm\nextline
	Let $T:X\xrightarrow{}\mathbb{C}$ be a complex linear functional. Define $u(x)=Re(T(x))$ for all $x\in X$. Then
	\begin{itemize}
		\item $u(x)$ is a real-valued linear functional
		\item $T(x)=u(x)-iu(ix)$
		\item $\norm{u}=\norm{T}$
	\end{itemize}
	Moreover, given any linear functional $u(x)$, $T(x)=u(x)-iu(ix)$ defines a complex linear functional
	\begin{pf}{}{}

	The first two are very easy to show. The hardest part is on the third statement. We first show that $\norm{T}\geq\norm{u}$:
	\begin{equation}
		\begin{split}
			\norm{T}^2&=\sup\{|T x|^2,\norm{x}=1\}\\
			&=\sup\{|u(x)-iu(ix)|^2,\norm{x}=1\}\\
			&=\sup \{[u(x)]^2+[u(ix)]^2,\norm{x}=1\}\\
			&\geq \sup \{[u(x)]^2,\norm{x}=1\}\\
			&=\norm{u}^2
		\end{split}
	\end{equation}
	On the other hand, pick any $x\in X$ with $\norm{x}=1$, denote $T(x)=re^{i\theta}$, we have
	$$|T(x)|=|e^{-i\theta}||T(x)|=|T(e^{-i\theta}x)|=|u(e^{-i\theta}x)-iu(e^{-i\theta}x)|$$
	\begin{equation}
		\begin{split}
			|T(x)|&=|e^{-i\theta}||T(x)|\\
			&=|T(e^{-i\theta}x)|\\
			&=|u(e^{-i\theta}x)-iu(e^{-i\theta}x)|
		\end{split}
	\end{equation}
	But we have that $T(e^{-i\theta}x)=r\in\mathbb{R}$, thus
	$$
		|T(e^{-i\theta}x)|=|Re(T(e^{-i\theta}x))|=|u(e^{-i\theta}x)|\leq\norm{u}
	$$
	Hence $\norm{T}=\norm{u}$.
\end{pf}
\end{proposition}
The remaining part of the proof is simply combining last result and the proof of real case. However, last result also gives a insight on complex bounded linear functional

\subsection{Hahn-Banach with sublinear function}
Previous discussion of Hahn Banach theorem considers extension preserving norm. However, we can consider extension in a more general setting, namely sublinear.
\begin{definition}[sublinear function]\label{sublinear map}\rm\nextline
	Let $X$ be a vector space. \func{p}{X}{\real} is a sublinear function if the following holdsL
	\begin{itemize}
		\item $p(kx)=kp(x),$ for all $x\in X$ and $k\geq0$
		\item $p(x+y)=p(x)+p(y),\,\,\forall x,y\in X$
	\end{itemize}
\end{definition}

\begin{definition}[Restriction]\rm\nextline
	Let \func{f}{X}{Y}. Let $S\subset X$, if \func{g}{M}{Y} is such that $g(x)=f(x),\,\,\forall x\in M$ then we say $f$ restricted to $M$ is $g$, denoted by $g=f|_M$
\end{definition}
\begin{theorem}[Hahn-Banach Theorem, sublinear]\rm\nextline
	Let $M\in X$ be a linear subspace of $X$, where $X$ is a linear space. \func{p}{X}{\real} is sublinear, and \func{f}{M}{\real} is a linear map such that $f(x)\leq p(x),\,\,\forall x\in M$.\\
	Then, there exists a linear map \func{F}{X}{\real} with $F|_M=f$ and $F(x)\leq p(x),\,\,\forall x\in M$.
	\begin{pf}{Hahn-Banach Theorem, sublinear}{}
		\placeholder
	\end{pf}
\end{theorem}

\subsection{Results of Hahn-Banach}
\begin{theorem}[Dual functional]\label{Dual functional}\rm\nextline
	Let $X$ be a normed linear space. $\forall x\in X,\,\exists x^*\in \dual{X}$ s.t. $\inne{\dual{x}}{x}\equiv\dual{x}(x)=\norm{x}^2_X=\norm{\dual{x}}^2_{\dual{X}}$\\
	Note that here $\inne{\cdot}{\cdot}$ does not necessarily stand for inner product.
	\begin{pf}{Dual functional}{}
		Let $M=span(x)$. Define $f(tx)=t\norm{x}^2_X,\,\,\forall t\in \real$. Clearly $f$ is linear, and that $\norm{f}_{\dual{M}}=\norm{x}_X$. \\
		Then we apply Hahn-Banach theorem to extend $f$ to $\dual{x}=F\in \dual{X}$, with $\norm{\dual{x}}_{\dual{X}}=\norm{f}_M=\norm{x}_X$
	\end{pf}
\end{theorem}
When the space is a Hilbert space, this theorem becomes Riesz representation theorem. (without changing notation! That's why bracket is a good notation here) In short, this theorem says that you can always find a linear functional such that for its value for a chosen element is precisely the norm of this element.

\begin{proposition}[Point-point separation]\label{separations}\rm\nextline
	Let $x$ and $y$ be points in $X$ \placeholder \\
	(Question here is do  we need X to be normed? or even Banach?)
	There exists $l\in\dual{X}$ such that $l(x)\neq l(y)$.
	\begin{pf}{}{}
		Choose $l\in \dual{X}$ to be the dual functional of $(y-x)\in X$. \\Then $l(y-x)=\norm{y-x}^2_X>0\implies l(y)\neq l(x)$.
	\end{pf}
\end{proposition}

\begin{proposition}[separation from proper closed subspaces]\rm\nextline
$M\subset X$ is a linear space, closed and (naturally) convex, Assume $x_0\not\in M$ such that $d\equiv dist(x_0,M)$, then $\exists \ell\in\dual{X}$ with $\ell|_M=0$ and $\norm{\ell}_\dual{X}=1$ and $\ell(x_0)=d$.
\begin{pf}{}{}
	Let $M_0=\{xt+x_0:x\in M\}$. Define \func{f}{M_0}{\real} which is linear, $f(x+tx_0)=td$. Then\\
	\placeholder
\end{pf}
\end{proposition}
\input{Chapter/UBP.tex}
\newpage
\section{Open mapping theorem}
\begin{comment}
\bsegin{comment}
\begin{definition}[Open map]\label{open map}\nl
	An open map is one for which the image of every open set is open.
\end{definition}
\begin{theorem}[Open Mapping Theorem]\label{OMT}\nl
	Let $X,Y$ be Banach spaces, $T:X\xrightarrow{}Y$ a continuous surjective linear map. Then $T$ is an open map.
\end{theorem}
\eend{comment}

\begin{definition}[Open Ball]\nl
	An open ball in normed linear space $X$ with radius $r>0$ centered at $x\in X$ is
	$$
		B_X(x,r)=\{y\in X:\norm{y-x}_X<r\}
	$$
	Also, when $x=0$ we write
	$$
		B_X(0,r)\equiv B_x(r)
	$$
\end{definition}

\begin{definition}[Open map]\label{open map}\nl
	Let $X$, $Y$ be linear spaces. \func{A}{X}{Y} is {\underline{open}} if $A(U)\subset Y $ is open.
\end{definition}

\begin{remark}\hfill

	\begin{itemize}
		\item $A$ being continuous means $A^{-1}(V)\subset{X}$ open $\forall V\subset Y$ open.
		\item $A$ being continuous need not be open. e.g. $Ax\stackrel{def}{=}0\in Y$
	\end{itemize}
\end{remark}

\begin{theorem}[Open Mapping Theorem]\label{OMT}\nl
	Let $X,Y$ be Banach, $A\subset\mathcal{L}(X,Y)$. Then:
	\begin{itemize}
		\item[i)] if $A$ is surjective, $A$ is open.
		\item[ii)] if $A$ is bijective, then $A^{-1}\in \mathcal{L}(X,Y)$. (Inverse operator theorem)
	\end{itemize}
\end{theorem}

\begin{remark}\nl
	ii) important in application. If $A\in\mathcal{L}(X,Y)$ is bijective then \func{A^{-1}}{X}{Y} liner is easy (why?). The point is $A^{-1}$ is also bounded, or equivalently continuous.
\end{remark}

The main step of the proof is the following:
\begin{lemma}[$A$ as in i)]\nl
	$\exists r>0$ s.t. $B_Y(r)\subset \overline{A(B_x(1))}$
	\begin{pf}{}{}\rm
		Since $A$ is surjective
		$$
			Y=\bigcup_{k=1}^\infty A(B_X(k))
		$$
		Since $Y$ is complete, by  Baire Category theorem, $\exists k_0$ s.t.
		$$ int(\overline{A(B_X(1))})\neq\emptyset$$
		So by surjectivity of $A$, one can find $y_0=Ax_0\in Y$, $r_0>0$ s.t.
		$$ \underbrace{B_Y(y_0,r_0)}_{=Ax_0+B_Y(r_0)}\subset \overline{A(B_X(k_0))}$$
		By linearity of $A$,
		\begin{equation}\nonumber
			\begin{split}
				B_Y(r_0)&\subset\overline{A(B_X(k_0))}-Ax_0 \\ &=\overline{A(B_X(k_0)-x_0)}\\
				&\subset \overline{A(B_X(k_0+M))}   \\
				&=(k_0+M)\overline{A(B_X(1))}\\
			\end{split}
		\end{equation}
		Where $M\stackrel{def}{=}\norm{x_0}_X$. So pick $r=\frac{r_0}{k_0+M}$.
	\end{pf}

\end{lemma}
Proof of theorem:
\begin{pf}{}{}
	i) Pick $r$ as in Lemma.\\
	Claim: $B_Y(r/2)\subset A(B_X(1)))$.\\
	If claim holds, then for $U\subset X$ open, pick $x_0\in U$, $s>0$ small so that $B_X(x_0,s)\subset U$. Letting $y_0\stackrel{def}{=}Ax_0$, get
	$$
		B_Y(y_0,rs/2)=y_0+sB_Y(r/2)\stackrel{claim}{\subset}Ax_0+sA(B_X(1)\stackrel{lin.}{=}A(B_X(x_0,s))\subset A(U)
	$$
	which proves i). To see i) $\implies$ ii), it's enough to show that $B=A^{-1}:Y\to X$ is continuous; but for any $U\subset X$ open, $B^{-1}(U)=(A^{-1})^{-1}(U)=A(U)$
	which is open by i). $\square$
\end{pf}
Proof of claim:
\begin{pf}{}{}
	Fix $y\in B_Y(r/2)$. Need to show: $y=Ax$ for some $x\in X$ with $\norm{x}_X<1.$\\
	We construct a sequence $(x_k)\subset X$ with
	$$
		\sum_{k=1}^\infty \norm{x_k}_X<1\,\,\text{and}\,\,\sum_{k=1}^\infty Ax_k\stackrel{wrt\norm{\cdot}_Y}{\longrightarrow{}}y,\,n\to\infty
	$$
	By completeness of $X$, $\sum_{k=1}^\infty x_k\stackrel{def.}{=}x$ exists, $x\in B_X(1)$ and by continuity of $A$,
	$$Ax=\sum_{k=1}^\infty Ax_k=y$$
	By lemma above,
	$$\forall s>0, B_Y(sr)\subset \overline{A(B_X(s))}\,\,(*)$$
	$s=1/2$. Pick $x_1\in B_X(1/2)$ s.t. $\norm{Ax_1-y}<r/2$. Now set $y_1=y-Ax(\in B_X(r/2)$. Iterate. Assume that for some $\geq 1$ have $x_1,......x_k,y_1,......y_k$ s.t.
	$$
		\forall 1\leq\tilde{k}\leq k:\,\,\norm{\tilde{x_k}}_X<2^{-k},\,y_{\tilde{k}}=y_{\tilde{k}-1}-Ax_{\tilde{k}}\in B_Y(2^{-\tilde{k}}r
	$$
	Then using $(*)$ with $s=2^{-(k+1)}$ find $x_{k+1}\in B_X(2^{-(k+1)})$ such that
	$$
		y_{k+1}\stackrel{def}{=}y_k-Ax_{k+1}\in B_Y(2^{-(k+1)}r
	$$
	This yields $\sum{k=1}^{\infty}\norm{x_k}_X<1$ and
	$$
		y-\sum_{k=1}^n Ax_k=y_1-\sum_{k=2}^n Ax_k=...=y_n\to0\,(n\to \infty)\quad\square
	$$
\end{pf}

\begin{example}[Equivalence of Norm]\label{Equinorm}\nl
	Let $X=Y$, with norms $\norm{\cdot}_1$ and $\norm{\cdot}_2$ and assume $\exists C>0$ s.t. $$\norm{x}_2\leq C\norm{x}_1,\,\forall x\in X\quad(1)$$
	If $X$ is complete, with respect to both $\norm{\cdot}_1$ and $\norm{\cdot}_2$ then consider $A=id:(X,\norm{\cdot}_1)\to(X,\norm{\cdot}_2)$ is open by Theorem (indeed thm applies $b/c$ $A$ is bounded by $(1)$. Since $A$ is bijective, ii) gives that $A^{-1}=id:(X,\norm{\cdot}_2)\to(X,\norm{\cdot}_1)$ is bounded, i.e.
	$$
		\exists C': \norm{A^{-1}}_1=\norm{x}_1\leq C'\norm{x}_2
	$$
	so $\norm{\cdot}_1$ and $\norm{\cdot}_2$ are actually equivalent.
\end{example}

\begin{example}[Completeness of $Y$]\nl
	Consider $X=C(=C^0[0,1])$ with $\norm{\cdot}_1=\norm{\cdot}_\infty$, $\norm{\cdot}_2=\norm{\cdot}_{L^1}$. Then $A=id:(X,\norm{\cdot}_1)\to(X,\norm{\cdot}_2)$ is continuous:
	$$
		\norm{Af}_2
		=\norm{f}_2
		=\int_0^1|f(t)|dt
		\leq\norm{f}_\infty
		=\norm{f}_1
	$$
	but not open. Else by 1),  $\norm{\cdot}_1$ and $\norm{\cdot}_2$ would be equivalent. However consider counter example:
	\begin{equation}\nonumber
		f_n(x)=\left\{
		\begin{aligned}
			 & {2n^2x}     & x\in[0,\frac{1}{2n}]           \\
			 & {-2n^2x+2n} & x\in(\frac{1}{2n},\frac{1}{n}] \\
			 & 0           & x\in(\frac{1}{n},1]            \\
		\end{aligned}
		\right.\quad\text{satisfy}\quad\norm{f_n}_2=1,\norm{f_n}_1=n\to\infty
	\end{equation}
	This shows $Y$ needs to be complete in theorem.
\end{example}
\begin{example}[Completeness of $X$]\nl
	This example shows completness of $X$ is also required.
	Take
	$$
		X=Y=\{(x_n)\in\ell^\infty:\exists N:x_n=0\,\forall m\geq N\}\subset\ell^\infty
	$$
	with norm $\norm{\cdot}_X=\norm{\cdot}_Y=\norm{\cdot}_\infty$. This is a linear normed space. It's not complete (Exercise: show directly $\overline{X}=c_0$). Another way:
	Define \func{A}{X}{X},
	$$
		Ax=(x_1,\frac{x_2}{2},\frac{x_3}{3}\underbrace{......}_{0\,eventually})\quad if \,\,x=(x_1,x_2......)
	$$
	Then $A$ is linear, bijective with
	$$
		A^{-1}:X\to ,\,\,\,A^{-1}x=(x_1,2x_2,3x_3\underbrace{......}_{0\,eventually})
	$$
	and $A$ is bounded.
	$$
		\norm{Ax}_\infty=\sup_{n\geq1}\frac{|x_n|}{n}\leq\sup_{n\geq1}|x_n|=\norm{x}_\infty
	$$
	so $\norm{A}\leq1$. But $A^{-1}$ is unbounded.
	Pick $x^{(n)}=(\overbrace{1,1,1,1}^{n},0,......)$ then $\norm{x^{(n)}}_\infty=1$ but $\norm{A^{-1}x^{(n)}}=n$. Hence $A^{-1}\not\in\mathcal{L}(X)$ and $X$ cannot be complete, else by theorem i), $A^{-1}$ would be bounded.
\end{example}



\subsection{Closed Graph Theorem}

\begin{definition}[Graph,closed graph and closed operator]\label{Closed Operator}\nl
	Let $X$, $Y$ be normed spaces. Linear operator $T$ defined on $D(T)\subset X$, linear subspace of $X$. Graph of $T$ is $\Gamma (T)=\{(x,Tx):x\in D(T)\}$. If $\Gamma (T)$ is closed in $(X\times Y,\norm{\cdot}_{X\times Y}$, where $\norm{\cdot}_{X\times Y}$ can be set to $max(\norm{\cdot}_{X},\norm{\cdot}_{Y)}$ or $\norm{\cdot}_{X}+\norm{\cdot}_{Y}$, we say $\Gamma T$ is closed, and $T$ is called closed operator.
\end{definition}





\begin{theorem}[Closed Graph Theorem]\label{CGT}\nl
	Let $X$, $Y$ be Banach. $T$ an linear operator \func {T}{X}{Y}. The following two statements are equivalent:
	\begin{itemize}
		\item $T$ is closed.
		\item $T$ is continuous.
	\end{itemize}
	\begin{pf}{$bounded\implies closed$}{}
		We should check that limit of a convergent sequence sequence in $\Gamma(T)$ is inside $\Gamma(T)$.
		By
		\sref{Completeness of product of Banach spaces} 
		$X\times Y$ is complete.
		Since $T$ is continuous, consider $\sequ{a_n}\subset X\times Y$ where $a_n=(x_n,Tx_n)$ with $x_n\to x\in X$. By continuity of $T$ and completeness of $Y$ we have that $Ax_n\to y=Ax\in T(X)\subset Y$. Then we have that
		$$\norm{(x,Tx)-(x_n,Tx_n)}_{X \times Y}=\underbrace{\norm{x-x_n}_{X}}_{\to 0}+\underbrace{\norm{y-Tx_n}}_{\to 0}\to0\quad \text{as  }n\to0$$
		This means that $a_n\to(x,Tx)\in \Gamma(T)$. Thus the graph is closed, hence $T$ is closed.
	\end{pf}
	\begin{pf}{$closed\implies bounded$}{}
		Idea of this part of the proof is recover $T$ by the partly inverting the map from $(X,Y)$ to $\Gamma(T)$.
		Consider two linear maps:
		$$
			\Pi_A():\Gamma(T)\to X,\,\Pi_A((x,Tx))=x\qquad \Pi_B():\Gamma(T)\to Y,\,\Pi_A((x,Tx))=Ax
		$$
		We can use sequential continuity to check that both maps are continuous. Moreover, $\Pi_A$ is  surjective and injective from its definition, thus bijective. By open mapping theorem, we have that $\exists {\Pi_A}^{-1}\in\mathcal{L}(X,\Gamma(T))$, which means that ${\Pi_A}^{-1}$ is continuous. Thus we have
		$$
			T=\underbrace{{\Pi_A}^{-1}}_{conti.}\circ\underbrace{{\Pi_B}}_{conti.}
		$$
		Is also countinuous.
	\end{pf}
\end{theorem}
\begin{corollary}[Continuous Inverse]\nl
	$X$, $Y$ Banach, $A:(DA)\subset X\to Y$ linear, closed and bijective. Then $\exists B=A^{-1}\in\mathcal{L}(Y,X)$ with $AB=id_Y$ and $BA=id_{D(A)}$.
	Proof is left as an exercise. Hint: similar to CGT, consider $\Pi_Y:\Gamma_A\to Y$, $B\stackrel{def.}{=}\Pi_X\circ \Pi_Y^{-1}$
	
\end{corollary}

\begin{example}[???]\nl
	A is surjective: for $g\in C$ define $f(t)=\int_0^t g(s) ds$. Then by FTC, $Af=g$.\\
	A is not injective: $Af=A\tilde{f}\implies f=\tilde{f}+c,c\in\real$. 
	Let $D(A)\defeq C_0^1[0,1]=\{f\in C^1[0,1]:f(0)=0\}$
	Then $A:D(A)\to C$ is bijective and has continuous inverse $B=A^{-1}$ by corollary. In fact, $Bf(t)=\int_0^tf(s)ds$ with $Bf\in D(A)$.
\end{example}

\end{comment}


\section{Open mapping theorem}
\begin{definition}[Open Ball]\nl
An open ball in normed linear space $X$ with radius $r>0$ centered at $x\in X$ is
$$
B_X(x,r)=\{y\in X:\norm{y-x}_X<r\}
$$
Also, when $x=0$ we write 
$$
B_X(0,r)\equiv B_x(r)
$$
\end{definition}

\begin{definition}[Open map]\label{open map}\nl
	Let $X$, $Y$ be linear spaces. \func{A}{X}{Y} is {\underline{open}} if $A(U)\subset Y $ is open.
\end{definition}
\begin{remark}\hfill

\begin{itemize}
    \item $A$ being continuous means $A^{-1}(V)\subset{X}$ open $\forall V\subset Y$ open.
    \item $A$ being continuous need not be open. e.g. $Ax\stackrel{def}{=}0\in Y$
\end{itemize}
\end{remark}

\begin{theorem}[Open Mapping Theorem]\label{OMT}\nl
	Let $X,Y$ be Banach, $A\subset\mathcal{L}(X,Y)$. Then:
	\begin{itemize}
	    \item[i)] if $A$ is surjective, $A$ is open.
	    \item[ii)] if $A$ is bijective, then $A^{-1}\in \mathcal{L}(X,Y)$. (Inverse operator theorem)
	\end{itemize}
\end{theorem}

\begin{remark}\nl
ii) important in application. If $A\in\blf{X}{Y}$ is bijective then \func{A^{-1}}{X}{Y} liner is easy (why?). The point is $A^{-1}$ is also bounded, or equivalently continuous.
\end{remark}

The main step of the proof is the following:
\begin{lemma}[$A$ as in i)]\nl
$\exists r>0$ s.t. $B_Y(r)\subset \overline{A(B_x(1))}$
\begin{pf}{}{}\rm
Since $A$ is suerjective 
$$
Y=\bigcup_{k=1}^\infty A(B_X(k))
$$
Since $Y$ is complete, by  Baire Category theorem, $\exists k_0$ s.t. 
$$ int(\overline{A(B_X(1))})\neq\emptyset$$
So by surjectivity of $A$, one can find $y_0=Ax_0\in Y$, $r_0>0$ s.t. 
$$ \underbrace{B_Y(y_0,r_0)}_{=Ax_0+B_Y(r_0)}\subset \overline{A(B_X(k_0))}$$
By linearity of $A$,
\begin{equation}\nonumber
    \begin{split}
        B_Y(r_0)&\subset\overline{A(B_X(k_0))}-Ax_0 \\ &=\overline{A(B_X(k_0)-x_0)}\\
        &\subset \overline{A(B_X(k_0+M))}   \\
        &=(k_0+M)\overline{A(B_X(1))}\\
    \end{split}
\end{equation}
Where $M\stackrel{def}{=}\norm{x_0}_X$. So pick $r=\frac{r_0}{k_0+M}$.
\end{pf}

\end{lemma}
Proof of theorem:
\begin{pf}{}{}
i) Pick $r$ as in Lemma.\\
Claim: $B_Y(r/2)\subset A(B_X(1)))$.\\
If claim holds, then for $U\subset X$ open, pick $x_0\in U$, $s>0$ small so that $B_X(x_0,s)\subset U$. Letting $y_0\stackrel{def}{=}Ax_0$, get 
$$
B_Y(y_0,rs/2)=y_0+sB_Y(r/2)\stackrel{claim}{\subset}Ax_0+sA(B_X(1)\stackrel{lin.}{=}A(B_X(x_0,s))\subset A(U)
$$
which proves i). To see i) $\implies$ ii), it's enough to show that $B=A^{-1}:Y\to X$ is continuous; but for any $U\subset X$ open, $B^{-1}(U)=(A^{-1})^{-1}(U)=A(U)$
which is open by i). $\square$
\end{pf}
Proof of claim:
\begin{pf}{}{}
Fix $y\in B_Y(r/2)$. Need to show: $y=Ax$ for some $x\in X$ with $\norm{x}_X<1.$\\
We construct a sequence $(x_k)\subset X$ with 
$$
\sum_{k=1}^\infty \norm{x_k}_X<1\,\,\text{and}\,\,\sum_{k=1}^\infty Ax_k\stackrel{wrt\norm{\cdot}_Y}{\longrightarrow{}}y,\,n\to\infty
$$
By completeness of $X$, $\sum_{k=1}^\infty x_k\stackrel{def.}{=}x$ exists, $x\in B_X(1)$ and by continuity of $A$,
$$Ax=\sum_{k=1}^\infty Ax_k=y$$
By lemma above, 
$$\forall s>0, B_Y(sr)\subset \overline{A(B_X(s))}\,\,(*)$$
$s=1/2$. Pick $x_1\in B_X(1/2)$ s.t. $\norm{Ax_1-y}<r/2$. Now set $y_1=y-Ax(\in B_X(r/2)$. Iterate. Assume that for some $\geq 1$ have $x_1,......x_k,y_1,......y_k$ s.t.
$$
\forall 1\leq\tilde{k}\leq k:\,\,\norm{\tilde{x_k}}_X<2^{-k},\,y_{\tilde{k}}=y_{\tilde{k}-1}-Ax_{\tilde{k}}\in B_Y(2^{-\tilde{k}}r
$$
Then using $(*)$ with $s=2^{-(k+1)}$ find $x_{k+1}\in B_X(2^{-(k+1)})$ such that
$$
y_{k+1}\stackrel{def}{=}y_k-Ax_{k+1}\in B_Y(2^{-(k+1)}r
$$
This yields $\sum{k=1}^{\infty}\norm{x_k}_X<1$ and 
$$
y-\sum_{k=1}^n Ax_k=y_1-\sum_{k=2}^n Ax_k=...=y_n\to0\,(n\to \infty)\quad\square
$$
\end{pf}

\begin{example}[Equivalence of Norm]\label{Equinorm}\nl
Let $X=Y$, with norms $\norm{\cdot}_1$ and $\norm{\cdot}_2$ and assume $\exists C>0$ s.t. $$\norm{x}_2\leq C\norm{x}_1,\,\forall x\in X\quad(1)$$
If $X$ is complete, with respect to both $\norm{\cdot}_1$ and $\norm{\cdot}_2$ then consider $A=id:(X,\norm{\cdot}_1)\to(X,\norm{\cdot}_2)$ is open by Theorem (indeed thm applies $b/c$ $A$ is bounded by $(1)$. Since $A$ is bijective, ii) gives that $A^{-1}=id:(X,\norm{\cdot}_2)\to(X,\norm{\cdot}_1)$ is bounded, i.e.
$$
\exists C': \norm{A^{-1}}_1=\norm{x}_1\leq C'\norm{x}_2
$$
so $\norm{\cdot}_1$ and $\norm{\cdot}_2$ are actually equivalent.
\end{example}

\begin{example}[Completeness of $Y$]\nl
Consider $X=C(=C^0[0,1])$ with $\norm{\cdot}_1=\norm{\cdot}_\infty$, $\norm{\cdot}_2=\norm{\cdot}_{L^1}$. Then $A=id:(X,\norm{\cdot}_1)\to(X,\norm{\cdot}_2)$ is continuous:
$$
\norm{Af}_2
=\norm{f}_2
=\int_0^1|f(t)|dt
\leq\norm{f}_\infty
=\norm{f}_1
$$
but not open. Else by 1),  $\norm{\cdot}_1$ and $\norm{\cdot}_2$ would be equivalent. However, consider counter-example:
\begin{equation}\nonumber
f_n(x)=\left\{
\begin{split}
    &{2n^2x} &x\in[0,\frac{1}{2n}]\\
    &{-2n^2x+2n} &x\in(\frac{1}{2n},\frac{1}{n}]\\
    &0 &x\in(\frac{1}{n},1]\\
\end{split}
\right.\quad\text{satisfy}\quad\norm{f_n}_2=1,\norm{f_n}_1=n\to\infty
\end{equation}
This shows $Y$ needs to be complete in theorem.
\end{example}
\begin{example}[Completeness of $X$]\nl
This example shows completeness of $X$ is also required.
Take 
$$
X=Y=\{(x_n)\in\ell^\infty:\exists N:x_n=0\,\forall m\geq N\}\subset\ell^\infty
$$
with norm $\norm{\cdot}_X=\norm{\cdot}_Y=\norm{\cdot}_\infty$. This is a linear normed space. It's not complete (Exercise: show directly $\overline{X}=c_0$). Another way:
Define \func{A}{X}{X}, 
$$
Ax=(x_1,\frac{x_2}{2},\frac{x_3}{3}\underbrace{......}_{0\,eventually})\quad if \,\,x=(x_1,x_2......)
$$
Then $A$ is linear, bijective with 
$$
A^{-1}:X\to ,\,\,\,A^{-1}x=(x_1,2x_2,3x_3\underbrace{......}_{0\,eventually})
$$
and $A$ is bounded. 
$$
\norm{Ax}_\infty=\sup_{n\geq1}\frac{|x_n|}{n}\leq\sup_{n\geq1}|x_n|=\norm{x}_\infty
$$
so $\norm{A}\leq1$. But $A^{-1}$ is unbounded. 
Pick $x^{(n)}=(\overbrace{1,1,1,1}^{n},0,......)$ then $\norm{x^{(n)}}_\infty=1$ but $\norm{A^{-1}x^{(n)}}=n$. Hence $A^{-1}\not\in\mathcal{L}(X)$ and $X$ cannot be complete, else by theorem i), $A^{-1}$ would be bounded.
\end{example}

\section{Closed Graph Theorem}
Consider $X$, $Y$ normed spaces. Often an operator $A$ not defined on all of $A$ but on a "domain" $D(A)$. So we assume that 
$D(A)\subset X$ is a linear subspace on which $A:D(A)(\subset X)\to Y$, linear is defined.
\begin{example}{Running Example}\nl
$Y=X=C=C^0[0,1]$ with $\norm{\cdot}_X=\norm{\cdot}_\infty$ and $A=\frac{d}{dt}$, with $D(A)\stackrel{eg}{=}C^1[0,1]\subset X$ or subspaces thereof. Prime example of  (\underline{unbounded}) operator with dense domain $D(A)$: indeed $C^1[0,1]=C$ using e.g. Weierstrass Approximation Theorem (Polynomials are already $\norm{\cdot}_\infty$-dense in C)
\end{example}
\begin{definition}[Graph]\nl
Let $X$, $Y$ be normed space, $A:D(A)(\subset X)\to Y$ . Graph of $A$ (really  of $(A,D(A))$) is the linear (!) space 
$$
\Gamma_A=\{(x,Ax):x\in D(A)\}\subset X\times Y
$$
We endowed $X\times Y$ with the norm $\norm{(x,y)}_{X\times Y}=\norm{x}_X+\norm{y}_Y$, for all $x\in X$, $y\in Y$.
    
\end{definition}
\begin{definition}[Closed Operator]\label{Closed Operator}
$A$ is called \underline{closed} if $\Gamma_A$ is closed in  $(X\times Y,\norm{\cdot}_{X\times Y})$
\end{definition}

\begin{example}
Let $A\in\mathcal{L}(X,Y)$ with $D(A)=X$. Then $A$ is closed.
\begin{pf}{}{}
	Let $(x_k,y_k)_k\subset \Gamma_A$  with $\norm{(x_k,y_k)-(x,y)}_{X\times Y}\xrightarrow{k\to\infty}0$ for some $(x,y)\in X\times Y$\\
	NTS: $(x,y)\in \Gamma_A$ i.e. $y=Ax$. 
	Know $ y_k=Ax_k$ and $\norm{x_k-x}_X\xrightarrow{k\to\infty}0$, $\norm{Ax-y}_Y\xrightarrow{k\to\infty}0$
	But $\forall k\geq 1$
	$$\norm{y-Ax}_Y\leq\norm{y-Ax}_Y+\norm{Ax_k-ax}_Y
	\leq \norm{y-Ax}_Y\norm{A}\norm{x_k-x}_X$$
	Thus 
	$$\lim_{k\to\infty}\norm{y-Ax}_Y\leq\lim_{k\to\infty}\norm{y-Ax}_Y\norm{A}\norm{x_k-x}_X=0$$
\end{pf}
\end{example}

\begin{theorem}[Closed Graph]\label{CGT}\nl
Let $X$, $Y$ be Banach $A:X\to Y$ linear. The following are equivalent:
\begin{itemize}
    \item [i)] $A\in\mathcal{L}(X,Y)$
    \item [ii)] $A$ is closed
\end{itemize} 
\begin{pf}{}{}
    i) $\implies$ ii): see example\\
    ii) $\implies$ i): If $X$, $Y$ complete, then so is $(X\times Y,\norm{\cdot}_{X\times Y})$ (exercise). A closed means $\Gamma_A$ is closed in $(X\times Y,\norm{\cdot}_{X\times Y})$, so $(\Gamma_A,\norm{\cdot}_{X\times Y})$ is complete. Consider:
    \begin{equation}
        \begin{aligned}
            \Pi_X:\,\,\Gamma_A&\to X \qquad\qquad& \Pi_Y:\Gamma_A&\to Y\\
        (x,Ax)&\mapsto x  & (x,Ax)&\mapsto Ax\\
        \end{aligned}    
    \end{equation}
$\Pi_X$, $\Pi_Y$ are continuous with $\norm{\Pi_X},\norm{\Pi_Y}\leq 1$, $\Pi_X$ is injective, and surjective. By OMT, ii), ${\Pi_X}^{-1}\in\mathcal{L}{(X,\Gamma_A)}$ and so
$$
A=\Pi_Y\circ \Pi_X^{-1}\in\mathcal{L}(X,Y)
$$
\end{pf}
\end{theorem}

\begin{remark}
	ii) is simpler than i), but equivalent.\\
	i) says A is continuous, i.e. if $(x_n)\subset X$, $x\in X$
	$$\norm{x_n-x}_X\rightarrow{n\to\infty}0\implies\norm{Ax_n-Ax}_Y\rightarrow{n\to\infty}0$$
	This contains two things to check: $(Ax_n)$ converges and limit is $Ax$.\\
	ii) says $A$ is closed, i.e.
	\begin{equation}
		\left\{
		\begin{aligned}
			&\norm{x_n-x}_X\rightarrow{n\to\infty}0\\
			&\norm{Ax_n-y}_Y\rightarrow{n\to\infty}0\\
		\end{aligned}
		\right.\implies
		Ax=y
	\end{equation}
	Which is only one condition to check.
\end{remark}

\begin{example}[running example continues]
	$(D(A),\norm{\cdot}_\infty)$ with $D(A)=C^1[0,1]$ is NOT Banach, and $A:D(A)\to C$ is an example of an operator which is:\\
	claim: \\
	i) closed, but\\
	ii) not continuous\\
	For ii), take $f_n(t)=t^n\in D(A)$, $Af_n=nf_{n-1}$ so $\norm{f_n}_\infty=1$, $\norm{Af_n}\infty=n\norm{f_{n-1}}\infty=n$. So 
	$$ \sup_{f\in D(A),\norm{f}\infty\leq1}\norm{Af}_\infty=\infty$$
	For i), if $(f_n,f_n')\to(f,g)$ in $(D(A)\times C)$ then $\norm{f-f_n}_\infty\to 0$, $\norm{f_n'-g}_\infty\to0$ but
	$$
	\forall t\in(0,1],\,\underbrace{f_n(t)}_{\rightarrow{n\to\infty}f(t)}
	=\underbrace{\int_0^t f'_n(x) dx}_{\rightarrow{DCT}\int_0^t g(x) dx}
	+f_n(0)
	$$
	so $f'=g$ by fundamental theorem of calculus(FTC), i.e. $(f,g)=(f,f')\in\Gamma_A$.
\end{example}


\begin{corollary}[Continuous Inverse]\nl
	$X$, $Y$ Banach, $A:(DA)\subset X\to Y$ linear, closed and bijective. Then $\exists B=A^{-1}\in\mathcal{L}(Y,X)$ with $AB=id_Y$ and $BA=id_{D(A)}$.
	Proof is left as an exercise. Hint: similar to CGT, consider $\Pi_Y:\Gamma_A\to Y$, $B\stackrel{def.}{=}\Pi_X\circ \Pi_Y^{-1}$
	
\end{corollary}

\begin{example}[???]\nl
	A is surjective: for $g\in C$ define $f(t)=\int_0^t g(s) ds$. Then by FTC, $Af=g$.\\
	A is not injective: $Af=A\tilde{f}\implies f=\tilde{f}+c,c\in\real$. 
	Let $D(A)\stackrel{def.}{=} C_0^1[0,1]=\{f\in C^1[0,1]:f(0)=0\}$
	Then $A:D(A)\to C$ is bijective and has continuous inverse $B=A^{-1}$ by corollary. In fact, $Bf(t)=\int_0^tf(s)ds$ with $Bf\in D(A)$.
\end{example}





\input{Chapter/Appendix.tex}
\end{document}

