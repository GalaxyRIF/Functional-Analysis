\documentclass{article}
\usepackage{geometry}
\geometry{left=3cm,right=3cm,top=2cm,bottom=2cm}
\usepackage[utf8]{inputenc}
\usepackage{amsmath}
\usepackage[framemethod=TikZ]{mdframed}
\usepackage{amsfonts}
\usepackage{mathrsfs}
\usepackage{comment}
\usepackage{enumerate}
\usepackage{xcolor}
\usepackage{titlesec}


\usepackage[hidelinks,backref]{hyperref}
\usepackage{cleveref}
\usepackage[most]{tcolorbox}
\titleformat*{\section}{\LARGE\bfseries}
\titleformat*{\subsection}{\Large\bfseries}
\titleformat*{\subsubsection}{\Large\bfseries}
\titleformat*{\paragraph}{\large\bfseries}
\titleformat*{\subparagraph}{\large\bfseries}
\title{Functional Analysis}
\author{RIF Galaxy }
\date{August 2022}

\begin{document}

\maketitle
%Remark\\
On this note:
This note is primarily made for personal study purpose. %One should realise that any part that is unnecessarily detailed are those trouble the author with subtle logical problem. He kept asking himself stupid questions on those details, resulting in detailed answers.\\
%The note is made prior to the course taking place,relevant materials to the lectures taking place at Imperial College London 3rd year FA course will be labelled with {\color{red}\bf\S\S\S},important for exam will be labelled by {\color{red}\bf***}
The note is made prior to the course taking place, and will be updated according to the lectures for Imperial College London 3rd year FA course(2022).

% General 
\newcommand{\nextline}{\hfill\break}
\newcommand{\nl}{\nextline}
\newcommand{\plz}{{\texorpdfstring{\color{red}\bf\S\S\S}{}}}
\newcommand{\plzaq}{{\texorpdfstring{\color{red}\bf***}{}}}
\newcommand{\placeholder}{{\color{red} NOOOOOOT COMPLEEEEEEET! COOOOOOOM BAAAAAACK!!!}}

% FA and LA
\newcommand{\inne}[2]{\left<{#1},{#2}\right>}
\newcommand{\norm}[1]{\left\|{#1}\right\|}
\newcommand{\hbs}{$\mathscr{H}$ }
\newcommand{\hbp}{\mathscr{H}}
\newcommand{\dual}[1]{{#1}^*}
\newcommand{\sequ}[1]{\left({#1}\right)_1^\infty}
\newcommand{\func}[3]{${#1}:{#2}\xrightarrow{}{#3}$}
\newcommand{\pf}{\textit{proof}:   }

\newcommand{\cgr}[3]{{#1}\equiv {#2} \,(\bmod \,\,{#3})}
\newcommand{\ncgr}[3]{{#1}\not\equiv {#2} \,(\bmod \,\,{#3})}

% Fields 
\newcommand{\real}{\mathbb{R}}
\newcommand{\comp}{\mathbb{C}}
\newcommand{\inte}{\mathbb{Z}}
\newcommand{\natu}{\mathbb{N}}





% Theorems
\newtheorem{example}{Example}[subsection]
\newtheorem{definition}[example]{Definition}
\newtheorem{proposition}[example]{Proposition}
\newtheorem{remark}[example]{Remark}
\newtheorem{theorem}[example]{Theorem}
\newtheorem{lemma}[example]{Lemma}
\newtheorem{corollary}[example]{Corollary}



\tableofcontents



\newpage
\section{List of content by week}
\subsection{Week 1}
\begin{itemize}
	\item Structure and arrangement of the course
	\item \hyperref[vector space defs]{Linear space, definition and examples}
	\item \hyperref[lp space1]{$L^p$ spaces, proof that it's a normed linear space}
	\item \hyperref[Minkowski-holder]{Minkowski's inequality}
	\item \hyperref[Hölder's inequality]{Hölder's inequality}
	\item \hyperref[banach space def]{Banach space, definition and method of checking its properties}
\end{itemize}
\subsection{Week2}
\begin{itemize}
	\item \hyperref[Metric linear space]{Metric linear space}
	\item \hyperref[Jensen's inquality]{Jensen's inquality and convex function}
	\item \hyperref[topology]{Topology: dense, separable space}
	\item \hyperref[separable space example]{Examples of separable spaces}
	\item \hyperref[Schauder basis]{Schauder basis, existence of it implies separability}
	\item \hyperref[Inner product]{Inner product, definition}
	\item \hyperref[Hilbert space def]{Hilbert space, definition}
	\item \hyperref[convexity]{Convex set}
	\item \hyperref[Nearest Point Property]{Nearest point property}
	      \item\hyperref[parallelogram]{Parallelogram law}
	\item \hyperref[ortho comp]{Orthogonal complement}
\end{itemize}

\subsection{Week3}
\begin{itemize}
	\item \hyperref[finite dimensional Banach]{Finite dimensional Banach Space}

	\item \hyperref[compactness]{Compactness, closedness and boundedness}
	\item \hyperref[compact unit balls]{Compactness and closed unit ball}
	\item \hyperref[continuity of LO]{Boundedness and continuity of Linear operator}
	\item \hyperref[operator norm]{Operator Norm}
	\item \hyperref[Linear Functionals]{\color{red}Content On linear functionals, helpful to understanding operator}
\end{itemize}

\subsection{Week4}
\begin{itemize}
	\item \hyperref[Riesz representation theory]{Riesz representation theory}
	\item \hyperref[dual space Hilbert]{Dual space of Hilbert space}
	\item \hyperref[dual space Banach]{Dual space of Banach space}
	\item \hyperref[lp dual]{Dual space of $\ell^p$ space}
	\item \hyperref[dual operator]{Dual Operator(Not finished!)}
\end{itemize}


\begin{comment}
\subsection{Week3}
\begin{itemize}
	\item \hyperref[]{}
	\item \hyperref[]{}
	\item \hyperref[]{}
	\item \hyperref[]{}
	\item \hyperref[]{}
	\item \hyperref[]{}
	\item \hyperref[]{}
	\item \hyperref[]{}
\end{itemize}
\end{comment}

\newpage

\section{Preliminaries}\label{vector space defs}
This section aims to provide preliminary knowledge to functional analysis. This field of maths is decorated by ideas of both algebra and analysis, particularly linear algebra and real analysis. It is thus important to get familiar with the relevant ideas, as lack of either viewpoint stops you from getting the whole story. At some point, since we are talking about different spaces, topological concepts also comes in. Luckily, they're generally not complicated and presented here as preliminaries.
\subsection{Linear space}
Mathematicians usually talks about spaces. However they are simply sets with additional structure. Linear spaces, also called vector spaces, are those with linear structure. This means you have vector addition, scalar multiplication, commutativity and distributivity.
\begin{definition}[Linear space]\rm\nextline
	A linear space $(V,\oplus,(\mathbb{F},+,\cdot),\odot)$ over a field $\mathbb{F}$, where
	\begin{itemize}
		\item $(V,\oplus)$ is an abelian group
		\item $(\mathbb{F},+,\cdot)$ is a field
	\end{itemize}
	and multiplication by a scalar $\odot:\mathbb{F}\times V\xrightarrow{}V$ satisfies for every $\alpha,\beta\in\mathbb{F}$ with $v,\omega\in V$
	\begin{itemize}
		\item $\alpha\odot(v\oplus\omega)=\alpha\odot v+\alpha\odot\omega$
		\item $(\alpha+\beta)\odot v=\alpha\odot v+\beta\odot v$
		\item $\alpha\odot(\beta\odot v)=(\alpha\cdot\beta)\odot v$
		\item $\mathbf{1}\cdot v=v$, where $\mathbf{1}$ is unit element in $\mathbb{F}$
	\end{itemize}
\end{definition}
\subsubsection{Examples of linear spaces}
In this section we have both example and counter example.
\begin{example}[Vector space over field]\rm\nextline
	$\mathbb{F}^n$ where $\mathbb{F}$ is a field, $n\in\\natu$ is a linear space. This includes 3 or 2-dimensional vector space over $\real$, or 3-dimensional vector field over $\mathbb{F}_p$. For example, let's check
	$$
		V=\mathbb{F}_2^2=\left\{(v_1,v_2)|v_1,v_2\in\mathbb{F}_2\right\}
	$$
	with the natural definition of scalar multiplication and term-wise addition over $\mathbb{F}_2$. Note that this is indeed a space of four elements:
	$$
		V=\left\{(0,0),(0,1),(1,0),(1,1)\right\}
	$$
	With scalars only 1 or 0. Thus it's easy to closure under scalar multiplication. Now consider vector addition, one just have to check every pair of addition and see if still falls into $V$, like
	$$
		(1,0)+(1,1)=(0,1)\in V
	$$
\end{example}
\begin{example}\rm\nextline
	Consider set of convergent sequence:
	$$
		V=\left\{\{x_n\}_1^\infty:x_i\in\mathbb{F}\,\forall i\in\mathbb{N},\,and\, \lim_{n\to \infty}x_n\to C\right\}
	$$
	First we consider the addition to be term-wise and multiplication applied to whole sequence:
	$$
		x+y=\{x_i+y_i\}_1^\infty,\,\alpha\odot x=\{\alpha\cdot x_i\},\,\,\forall x,y\in V,\,\alpha\in\mathbb{F}
	$$
	When $C$ is fixed, this is a vector space if and only if $C=0$
	When $C$ is not fixed, this becomes a vector space.
\end{example}

\begin{example}[Polynomials]\rm\nextline
	$V$ is set of all polynomials:
	$$
		f(z)=\sum_{j=0}^{n}a_jz^j,\,n\in\mathbb{N}
	$$
	with $z\in\mathbb{C}$ and $a_j\in{\mathbb{Q}}$ with addition and multiplication of polynomials. Unfortunately this is not a vector space, since the field is set to be $\mathbb{C}$. When we multiply an element in V by a complex number, say $1+2i$, we could end up in some polynomials with complex coefficient. But this would be a vector space if we change to $a_j\in \mathbb{C}$ or $z\in\mathbb{Q}$
\end{example}

\begin{example}[Analytic functions]\rm\nextline
	Consider set of all analytic functions $f:\mathbb{C}\xrightarrow{}\mathbb{C}$ satisfying:
	$$
		\frac{d^2}{dz^2}f-\frac{d}{dz}f-2z=0
	$$
	This is not a vector space. Take a non-trivial $f$, consider $g=2f$:
	\begin{equation}
		\begin{split}\nonumber
			&\frac{d^2}{dz^2}g-\frac{d}{dz}g-2z\\
			=&\frac{d^2}{dz^2}(2f)-\frac{d}{dz}(2f)-2z\\
			=&2\frac{d^2}{dz^2}f-2\frac{d}{dz}f-2z\\
			=&4z-2z=2z\neq0
		\end{split}
	\end{equation}
	However, removing $-2z$ will make this a vector space.
\end{example}

\begin{example}[Weak $L^p$ space]\rm\nextline
	Define $D_f(t)=\{\lambda{x\in\mathbb{R}:|f(x)|>t}\}$, where $\lambda$ is Lebesgue measure on $\mathbb{R}$. Consider:
	$$
		L^{p,w}=
		\left\{
		f:\mathbb{R}\xrightarrow{}\mathbb{R}:f,\,\,measurable,\,\, and\,\, \exists C>0,\,\,s.t.\,\,D_f(t)<\frac{C^p}{t_p}\,\forall t>0
		\right\}
	$$
	This is a vector space for $p>0$ and $L^p(\mathbb{R})\subset L^{p,w}(\mathbb{R})$. Distributivity and commutativity of scalar operation follows immediately from their definition. The point is to check closure under scalar multiplication and addition.\\
	$\mathbf{f+g\in L^{p,w}}:$\\
	Let $f,g\in L^{p,w}$, then we can find $C_1>0 $ and $C_2>0$ with
	\begin{equation}
		\begin{split}\nonumber
			C_1^p>t^p\cdot\lambda\{x:|f(x)|>t\},\,\,\forall t>0\\
			C_2^p>t^p\cdot\lambda\{x:|g(x)|>t\},\,\,\forall t>0
		\end{split}
	\end{equation}
	Now by triangular inequality we have $|f(x)|+|g(x)|\geq|f(x)+g(x))|$.\\
	Thus $|(f+g)(x)|>t\implies|f(x)|+|g(x)|>t\implies 2\cdot max(|f(x)|,|g(x)|)>t$.
	So consider set $A,B,C$:
	\begin{equation}
		\begin{split}\nonumber
			X:&=\{x\in\mathbb{R}:|f(x)+g(x)|>t\}\\
			Y:&=\{x\in\mathbb{R}:|f(x)|+|g(x)|>t\}\\
			Z:&=\{x\in\mathbb{R}:2max(|f(x)|,|g(x)|)>t\}=\left\{x\in\mathbb{R}:max(|f(x)|,|g(x)|)>t/2\right\}
		\end{split}
	\end{equation}
	We have $X\subseteq Y\subseteq Z$, thus $\lambda(X)\leq\lambda(Y)\leq\lambda(Z)$.\\
	Thus $t^p\cdot\lambda(A)\leq 2^p (t/2)^p\lambda\{x:max(|f(x)|,|g(x)|)>t/2\}\leq2^p\cdot max(C_1^p,C_2^p)<\infty$ for all $t>0$.\\
	Showing closure under scalar multiplication is a bit easier. Take any $t>0,$ we have
	\begin{equation}
		\begin{split}\nonumber
			&t^p\cdot\lambda\{x\in\mathbb{R}:|2f(x)|>t\}\\
			=&2^p\left(\frac{t}{2}\right)^p\cdot\lambda\{x\in\mathbb{R}:|f(x)|>t/2\}\\
			<&2^pC_1^p
		\end{split}
	\end{equation}
	Which completes the proof.
	
\end{example}


\subsection{Metric Linear space}\label{Metric linear space}
\begin{definition}[Metric linear space]\rm\nextline
	A metric space $(V,d)$ is called metric linear space if its vector addition and scalar multiplication $\oplus$, $\odot$ are continuous
\end{definition}


\begin{remark}[Equivalence of metrics]\rm\nextline
	$\oplus$ can be considered as a function: \func{\oplus}{V\times V}{V}, endowed with metric \func{\rho_1}{V\times V}{V} defined as $\rho_1(x_1+y_1,x_2+y_2)=max(d(x_1,x_2),d(y_1,y_2))$
	or sum \func{\rho_2}{V\times V}{V} defined as $\rho_2(x_1+y_1,x_2+y_2)=d(x_1,x_2)+d(y_1,y_2)$. The two metrics are topologically equivalent, i.e. they induces same topology.\\
	Similarly, $\odot$ can be considered as a function: \func{\odot}{\mathbb{F}\times V}{V}, endowed with metric \func{\rho_1}{\mathbb{F}\times V}{V} defined as $\rho_1(k_1x_1,k_2x_2)=max(|k_1-k_2|,d(x_1,x_2))$
	or sum \func{\rho_2}{V\times V}{V} defined as $\rho_2(x_1+y_1,x_2+y_2)=|k_1-k_2|+d(x_1,x_2)$. Again, the two metrics are topologically equivalent.
\end{remark}

\begin{definition}[Translation invariant]\rm\nextline
	A metric $\rho$ is translation invariant if for all $x,y,z\in V$, we have  $\rho(x,y)=\rho(x-z,y-z)$
\end{definition}

\begin{proposition}\rm\nextline
	Addition is continuous with respect to translation invariant metic.
\end{proposition}

\begin{proposition}[Metric induced by norm]\rm\nextline
	Let $\norm{\cdot}$ be a norm on $V$, then its induced metric $\rho(x,y)=\norm{x-y}$ is translation invariant. See definition of norm {\color{red} \hyperref[definition of norm]{here}}.
\end{proposition}

\subsection{Topology}\label{topology}

\subsubsection{Separability}

\begin{definition}[dense]\nl
	A set in $S$ metric space $(V,d)$ is dense if there it intersects with any open subset of $V$. Equivalently this is $\forall x\in V,\forall \varepsilon>0$, $D\cap B_{x,\varepsilon}\neq \emptyset$.
\end{definition}


\begin{definition}[separable]\nl
	A metric space $(V,d)$ is separable if it contains a countable dense subset.
\end{definition}
\begin{comment}
\begin{example}[separable space example]\label{separable space example}
\end{example}
\end{comment}
\subsubsection{Schauder Basis and Hamel basis}

\begin{definition}[Schauder basis]\label{Schauder basis}\nl
	A Schauder basis of Normed vector space $(v,\norm{\cdot})$ is a set $B\subset V$ that is linearly independent, and that $\forall x\in V$, we have a sequence $\{a_n\}$ with $\lim_{n\to\infty}\sum_{k=1}^n a_k b_k\to x$. In plain language, this means that every element of the set can be expressed as a infinite linear combination of the basis.
\end{definition}


\begin{proposition}[Schauder implies separability]\rm\nextline
	If a normed vector space has a Schauder basis, then it's separable.
\end{proposition}

\begin{definition}[Hamel basis]\rm\nextline
	A set $H\subset X$ is a Hamel basis for linear space $X$, if and only if $H$ is linearly independent and $\forall x\in X$, $\exists !a_i\in \real$ for all $1\leq i\leq n$ for some n and 
	$$
	x=\sum_{i=1}^n a_ih_i,\,\,h_i\in H
	$$
\end{definition}

\begin{remark}[Hamel basis and Schauder basis]\rm\nextline
	Having Hamel basis means that one can express any element by a necessarily finite linear combination of the basis. For Schauder basis, it could be infinite linear combintation.
\end{remark}

\subsubsection{Compactness}
\begin{definition}[Compactness]\label{compactness}\rm\nextline
	Let $(X,d)$ be a metric space. A subset $S\subset X$ is {\bf compact} if any sequence $(x_n)$ in $S$ has a convergent subsequence converges in to some $x\in S$
\end{definition}


\begin{remark}\rm\nextline
	The definition is "sequential compactness". There're many versions of definition of compactness. One shall really pay attention to the definition in linear algebra or real analysis that compactness is equivalence to closedness and boundedness. We'll see later that this is guaranteed to true only for finite-dimensional spaces. However, it is true that compactness always implies closedness and boundedness. Following examples we prove the implications in detail and present the fact that compactness of space can studied by looking at closed unit ball.
\end{remark}


\begin{proposition}[Compact $\implies$ closed and bounded]\rm\nextline
	If a set $K$ is compact, then it's closed and bounded. Here, bounded means $\exists L>0$ such that for all $x,y\in K$, we have $d(x,y)\leq L$.
	\begin{pf}{}{}
		First we show that compact implies closed. Let $(x_n)$ be a convergent sequence in $K$, then by compactness, there is a subsequence of $(x_n)$, name it $(y_n)$ which converges to $y\in K$, but by uniqueness of limit $(x_n)$ also converges to $y\in K$.\\
		Now we show boundedness by contradiction. Assume $K$ is not bounded. Then we fix $a\in K$, and by unboundedness we have that $\forall n\in \natu,\,\exists x\in K$ with $d(x,a)>n$. Now consider a sequence $(x_n)$ where  $d(x_n,a)>n$ $\forall n>0$. This sequence has no convergent subsequence.
	\end{pf}
\end{proposition}


\begin{theorem}[F.Riesz]\label{compact unit balls}\rm\nextline
	Let $(X,\norm{\cdot})$ be a normed vector space. Following are equivalent:
	\begin{itemize}
		\item $dim(X)\leq\infty$
		\item Closed unit ball $B_U=\{x\in X:\norm{x}\leq 1\}$ is compact.
	\end{itemize}
	\prf Proof uses a lemma which is shown later\placeholder
\end{theorem}

\begin{lemma}[F.Riesz]\rm\nextline
	Let $(X,\norm{\cdot}$ be a normed linear space with a subspace $Y\subset X$,$Y\neq X$.
	Then for all $\varepsilon\in(0,1)$, there exists $x\in X$ with $\norm{x}=1$ and $d(x,Y)\equiv\inf_{y\in Y} \norm{x-y}>1-\varepsilon$
	\prf \placeholder
\end{lemma}



\newpage







\section{Banach Space}\label{banach space def}%\plz


Banach space is the one of the core concepts of functional analysis.  It is a special type of {\bf vector space}, with a {\bf norm} working on the space as well as the property that every {\bf Cauchy sequence converges} in the space.\\
One should pay attention that a considerable portion of content in this chapter is not based on completeness but only requires a norm on the space. However, it is obvious that they can be applied to Banach spaces and most importantly, these results do relate themselves to Banach spaces.


\subsection{Definitions}

\begin{definition}[Normed vector space]\rm\label{definition of norm}\nextline
	A normed vector space $\mathbf{X}$ is a vector space (over $\mathbb{F}$, usually $\mathbb{C}$ or $\mathbb{R}$), equipped with a norm function $|| \cdot||:\mathbf{X}\times \mathbf{X}\xrightarrow{}{\mathbb{R}}$ satisfying following:

	\begin{itemize}
		\item $||x||\geq0,\,\forall x\in\mathbf{X}$
		\item $||x||=0$ if and only if $x=0$
		\item $||ax||=a||x||,\, \forall x\in\mathbf{X},\, \forall a\in\mathbb{F}$
		\item $||x+y||\leq||x||+||y||,\forall x,y\in\mathbf{X}$
	\end{itemize}
	A normed vector  space is also called a normed linear space.
\end{definition}


\begin{definition}[Cauchy Sequence]\rm\nextline
	\label{Cauchy Sequence}
	A Cauchy sequence in a normed vector space $V$ is a sequence $\{a_n\}_1^\infty$, where each $a_n\in V$, satisfying the following:
	$\epsilon>0,\exists\, N\in\mathbb{N}\,\,s.t.\,\,
		\forall\, m,n>N, ||a_m-a_n||<\epsilon$

\end{definition}

\begin{definition}[Convergence]\rm\nextline
	A sequence $\{a_n\}_1^\infty$in a normed vector space $V$ is convergent if
	$\exists a\in V$ s.t.
	$\forall \epsilon>0, \exists N>0 s.t. \forall n>N, ||a_n-a||<\epsilon$.

\end{definition}

\begin{definition}[Completeness]\rm\nextline
	A normed vector space is complete if every \href{Cauchy Sequence}{Cauchy Sequence} converges to a point in the space.

\end{definition}

\begin{theorem}[Every metric space can be completed]\rm\nextline
	Result is put \hyperref[completion of metric space]{\color {red} here} in appendix.
\end{theorem}
\begin{definition}[Banach space]\rm\nextline
	A Banach space is a complete normed vector space.
\end{definition}


%One might ask, is it possible for a Cauchy sequence in a normed vector space not converge??



\subsection{Examples}
\begin{example}[$\mathbb{R}^n$]\rm\nextline
	Our acquainted three-dimensional vector space over $\mathbb{R}$  is a Banach space under the standard vector norm, this is the case when $n=3$. This is a trivial result, since such norm gives the "length" of a vector, and a Cauchy sequence of vectors indicates that 'endpoints' of vectors come arbitrarily close. It's easy to prove that such a sequence converges to a three-dimensional vector. In fact, any n-dimensional  vector space over $\mathbb R$ is a Banach space.
\end{example}

\begin{example}[Real valued functions]\rm\nextline
	The set of all real-valued function on [0,1] with norm $\|f\|=\max_{t \in [0,1]} |f(t)|$
	is a Banach space. To verify this (thoroughly) we shall first show that this is a vector space. This is trivial since sum and and scalar multiplication of a real-valued function is also real-valued. Then we should show that this function is indeed a norm. Finally, we should check that every Cauchy sequence under this norm is convergent. Intuitively, this norm gives the 'maximal pointwise difference' between two functions, hence if a sequence is Cauchy, the 'maximal pointwise difference' converges to zero. Rigorous proof is left to readers.
\end{example}

\begin{example}[Continuous function under supremum norm]\rm\nextline
	Consider $C[0,1]$, set of continuous real-valued defined on $[0,1]$ with supremum norm:
	$$\|f\|_\infty=\sup_{t \in [0,1]} |f(t)|$$ This is a Banach space (why?). Now a relevant example is $C^1[0,1]$,set of complex-valued function defined on $[0,1]$ with continuous first derivative. Unfortunately this is not a Banach space under supremum norm, however, if we equip the space with a new norm: $\|f\|=\|f\|_\infty+\|f'\|_\infty$, we end up having a Banach space.\hfill
\end{example}

\begin{example}[L-p spaces]\rm\label{lp space1}\nextline
	L-p spaces are function spaces with finite $p-norm$. Let $(S,\Sigma,\mu)$ be measure space, $p\in[1,+\infty]$. L-p space consists of functions $S\xrightarrow{}\mathbb{C}$ with $p-norm$:
	$$\norm{f}_p\equiv\left(\int_S|f|^p \,d\mu\right)^\frac{1}{p}<\infty$$
	It is not obvious how $p-norm$ really gives a norm, the difficulties lie in the part of proving the triangular inequality. In the specific background of L-p spaces, the inequality is precisely \textit{Minkowski's inequality}. Detailed material on Hölder's inequality and Minkowski's inequality are put in appendix.
\end{example}


\subsection{Finite dimensional Normed space}\label{finite dimensional Banach}
Bare in mind that in this section we do not assume that spaces are complete, so please pay attention which results are based on completeness. However, we shall see that all finite dimensional vector space over complete fields are complete.\\
Some main theorems to focus on in this section, presented in plain language:
\begin{itemize}
	\item All finite dimensional vector space over a complete field are complete
	\item All norms on finite dimensional vector spaces are equivalent
	\item $Bounded+Closed=Compact$ if and only if dimension is finite.
\end{itemize}
Reference: \href{https://math.mit.edu/~stevenj/18.335/norm-equivalence.pdf}{norm-equivalence note}

\subsubsection{Equivalence of Norms and topology}
There are many norms. Some acts similarly, some acts differently. One may have seen different types of matrix norm, for example, Frobinius norm and operator norm. Computational mathematicians don't seem to care about this, why is it? Maybe they don't make much difference!
\begin{definition}[Equivalence of Norm]\label{equivalent norms}\rm\nextline
	Let $X$ be a vector space. Two norms $\norm{\cdot}_a$, $\norm{\cdot}_b$ on $X$ are {\bf equivalent} if their exists $m,n>0$ satisfying following equation:
	$$
		m\norm{x}_a\leq \norm{x}_b\leq M\norm{x}_a,\,\,\forall x\in X
	$$
\end{definition}

\begin{theorem}[Equivalence of finite dimensional norms]\rm\nextline
	Let $X$ be a finite dimensional vector space, then any two norms $\norm{\cdot}_a$ and $\norm{\cdot}_b$ are equivalent.\\
	\prf The proof are divided into four steps:
	\begin{itemize}
		\item Showing that equivalence of norm is transitive
		\item Showing that equivalence of equivalence on unit sphere implies equivalence on $X$
		\item Showing that any norm is continuous with respect to $\norm{\cdot}_1$
		\item Showing that any norm is equivalent to $\norm{x}_1\equiv\sum_{s=1}^n |x_s|$
	\end{itemize}
	\begin{pf}{STEP I}{}
		Let $\norm{\cdot}_a$ be equivalent to $\norm{\cdot}_b$ and $\norm{\cdot}_b$ equivalent to $\norm{\cdot}_c$.
		Then $\exists m_1,m_2,M_1,M_2>0$ with
		\begin{equation}
			\begin{split}\nonumber
				&m_1\norm{x}_a\leq \norm{x}_b\leq M_1\norm{x}_a,\,\,\forall x\in X\\
				&m_2\norm{x}_b\leq \norm{x}_c\leq M_2\norm{x}_b,\,\,\forall x\in X
			\end{split}
		\end{equation}
		Then
		\begin{equation}
			\begin{split}\nonumber
				&m_1m_2\norm{x}_a\leq m_2\norm{x}_b\leq \norm{x}_c,\,\,\forall x\in X\\
				&\norm{x}_c\leq M_2\norm{x}_b\leq M_1M_2\norm{x}_c\,\,\forall x\in X
			\end{split}
		\end{equation}
		Which gives $m_1m_2\norm{x}_a\leq\norm{c}\leq M_1M_2\norm{x}_a$ for arbitrary $x$. Hence $\norm{\cdot}_a$ and$\norm{\cdot}_c$ are equivalent.
	\end{pf}
	\begin{pf}{STEP II}{}
		Now let us assume that $\norm{\cdot}_a$ is equivalent to $\norm{\cdot}_b$ on $U_a=\{s\in X:\norm{s}_a=1\}$.
		Then let $x\in X$ be non-zero. Then we have
		$$
			m\norm{\frac{x}{\norm{x}_a}}_a\leq \norm{\frac{x}{\norm{x}_a}}_b\leq M\norm{\frac{x}{\norm{x}_a}}_a
		$$
		So
		$$
			m\norm{x}_a\frac{1}{\norm{x}_a}\leq \norm{x}_b\frac{1}{\norm{x}_a}\leq M\norm{x}_a\frac{1}{\norm{x}_a}
		$$
		And since ${\norm{x}_a}$ is non-zero,we have that
		$$
			m{\norm{x}_a}\leq \norm{x}_b\leq M\norm{x}_a
		$$
		Now since $x$ is arbitrary, the proof is completed.	
	\end{pf}
	\begin{pf}{STEP III}{}
		Now we shall proof continuity of any norm under $\norm\cdot _1$. This can be done by showing for a sequence $(x_n)$ converging to $x$ under the metric induced by $\norm{\cdot}_1$, the norm if its terms under $\norm{\cdot}_a$ converges to $\norm{x}_a$. So let $(x_n)$ be a sequence in $X$ with $x_n\xrightarrow[]{n\to \infty}x$. We have
		\begin{equation}
			\begin{split}\nonumber
				|\norm{x_n}_a-\norm{x}_a|&\leq\norm{x_n-x}_a\leq M\norm{x_n-x}_1\to0\quad \text{when}\,\,n\to\infty
			\end{split}
		\end{equation}
		Thus
		$$
			\lim_{n\to\infty}|\norm{x_n}_a-\norm{x}_a|=0
		$$
	\end{pf}
\begin{pf}{STEP IV}{}
	Now we shall apply extreme value theorem to obtain our final result here. Using the theorem requires unit sphere to be a compact set. Proof of this fact is given later, one should realise that the proof does not depend on equivalence of norms, as we only require compactness in $X,\norm{\cdot}_1$. However, it is true that unit sphere is compact under any norm in finite dimensional cases. So we have that $U_1=\{s\in X:\norm{s}_1=1\}$ is compact with  a  function $\norm{\cdot}_a$ continuous on it, so by extreme value theorem it attains maximum $M_U=max\{\norm{x}_a:x\in U_1\}$ and minimum $m_u=min\{\norm{x}_a:x\in U_1\}$ on $U_1$, thus for any $x\in U_1$ we have
	$$
		m_u \norm{x}_1=m_u\leq\norm{x}_a\leq M_u=M_u \norm{x}_1
	$$
	Hence we show that any norm is equivalent to $\norm{\cdot}_1$ on unit sphere.
\end{pf}
By combining the results of the four steps, we finish the proof of the theorem.
\end{theorem}

\begin{remark}\rm\nextline
	The proof is not unique. We can also choose other norms to be the "bridging" norm, say supremum norm which in finite dimensional case becomes the max norm: $\norm{x}\equiv max\{|x_i|\}$. About the meaning of equivalence here, in fact, equivalent norms are equivalent in the sense that they induces same topology, so it is also called "topologically"equivalent. Generally speaking, this means that topological properties such as open, close, compact, convergence, continuity which holds for one norm will hold in its equivalent norms.
\end{remark}
Following results are simple exercises to check statements above.

\begin{proposition}[Equivalence of openness]\rm\nextline
	Open sets in $(X,\norm{\cdot}_a)$ are open in $(X,\norm{\cdot}_b)$ (Following notation in definition of equivalent norms).
	\begin{pf}{}{}
	It suffices to check open balls. Let $B^a_x(r)\equiv\{s\in X:\norm{x-s}_a<r\}$ be open balls with radius $r>0$ centered at $x$, which is an open ball in $(X,\norm{\cdot}_a)$. Choose $p\in B^a_x(r)$, we should show that $\exists \varepsilon>0$ with $B^b_p(\varepsilon)\equiv\{s\in X:\norm{p-s}_b<\varepsilon\}\subset B^a_x(r)$.\\
	By openness of $B^a_x(r)$, we have that $\exists \varepsilon_a>0$ with
	\begin{equation}
		\begin{split}\nonumber
			B^a_p(\varepsilon_a)&\equiv\{s\in X:\norm{p-s}_a<\varepsilon_a\}\\
			&=\{s\in X:m\norm{p-s}_a<m\varepsilon_a\}\\
			&\supseteq \{s\in X:\norm{p-s}_b<m\varepsilon_a\}\\
			&=B^b_p(m\varepsilon_a)
		\end{split}
	\end{equation}
	Note that $B^b_p(m\varepsilon_a)\subseteq B^a_p(\varepsilon_a)\subset B^a_x(r)$, hence $\varepsilon=m\varepsilon_a$.
	\end{pf}
	
\end{proposition}


\subsection{Subspace of Normed linear space}
As shown previously, finite dimensional vector spaces are all complete (we will assume that the fields of vector spaces are complete, $\real$ or $\comp$). This is also true for finite dimensional subspace of normed linear spaces. Note that we don't need the space to be complete when asserting completeness of their finite dimensional subspace.
\begin{proposition}
	[Completeness of finite-dimensional subspace]\rm\nextline
	Let $(x,\norm{\cdot})$ be a normed linear space. Then a linear subspace $Y\subset X$ with $dim(Y)<\infty$ endowed with $\norm{\cdot}$ induced in $Y$ is a Banach space.
\end{proposition}
Also, we have closedness of finite-dimensional subspace.
\begin{proposition}
	[Closedness of finite-dimensional subspace]\rm\nextline
	Let $(x,\norm{\cdot})$ be a normed linear space. Then a linear subspace $Y\subset X$ with $dim(Y)<\infty$ endowed with $\norm{\cdot}$ induced in $Y$ is closed.
\end{proposition}
Proof of this two proposition is simple use of the results of finite dimensional normed linear spaces begin Banach, which is left to readers. We shall now take a look at few examples showing different subspaces of normed vector spaces.

\begin{example}[Finite-dim subspace being complete]\rm\nextline
	\placeholder
\end{example}

\begin{example}[Infinite-dim subspace not complete]\rm\nextline
	Consider $X=C[0,2]\subset L^1 [0,2]$, set of all real-valued continuous function on $[0,1$, endowed with 1-norm: $\norm{f}_1=\int_0^1|f(t)|dt$. Consider sequence of function $(f_n)$:
	\begin{equation}
		f_n(t)=\left\{
		\begin{aligned}\nonumber
			 & t^n & 0\leq t<1     \\
			 & 1   & 1\leq t\leq 2
		\end{aligned}
		\right.
	\end{equation}
	Now $f_n$ is Cauchy in 1-norm, but its limit is not in $C[0,1]$:
	\begin{equation}
		f(t)=\left\{
		\begin{aligned}\nonumber
			 & 0 & 0\leq t<1     \\
			 & 1 & 1\leq t\leq 2
		\end{aligned}
		\right.
	\end{equation}
	Thus X is not a complete subspace of $L^1[0,2$, as we have a Cauchy sequence that does not converge to a point in $X$.
\end{example}

\begin{example}[Completeness depends on choice of norm]\rm\nextline
	We continue considering the settings in our last example, but endow $X$ with supremum norm: $\norm{\cdot}_\infty$. Now our $(f_n)$ is no longer Cauchy. One can prove that $(X,\norm{\cdot}_\infty$ is actually complete. Intuitively, supremum norm is sensitive to "discontinuity at points", so convergence in supremum norm ensures no "jump" of value at any point.
\end{example}
\newpage

\newpage
\subsection{Fixed point}
Contraction mapping theorem, sometimes called Banach fixed point theorem, guarantees the existence and uniqueness of fixed points of certain self-maps of metric spaces, and provides a constructive method to find those fixed points.

\begin{theorem}[Fixed point]\rm \nextline
	Let $X$ be Banach space, $f:X\xrightarrow{}X$ is a contraction mapping, i.e. a mapping satisfying
	$
		d(f(x),f(y))\leq qd(x,y),\,\forall x,y\in X,
	$
	where $q<1$ is a fixed constant.
	Then $f$ admits a unique fixed point $x^*\in X$, which can be found by
	defining sequence ${x_n}$ in $X$, with $x_{n+1}=f(x_n)$, then $\lim_{n\to \infty}x_n=x^*$\\
	\textit{proof}:\\
	By assumption, we have $\norm{x_n+1-x_n}=\norm{f(x_n)-f(x_{n+1})}\leq q\norm{x_n-x_{n+1}}$.\\
	So if we let $k=\norm{x_2-x_1}/q$, we have that $\norm{x_{n+1}-x_n}\leq k q^n$ with $q<1$.
	This sequence is clearly Cauchy. To see this, choose $m,n\in\mathbb{N}$ with $m<n$
	\begin{equation}
		\begin{split}
			\norm{x_m-x_n}&\leq\sum_{i=m}^{n-1} \norm{x_{i+1}-x_i}\\
			&\leq\sum_{i=m}^{n-1} k q^i\\
			&=\frac{kq^{m} (1-q^{(n-m)})}{1-q}\\
			&< \frac{kq^{m}}{1-q}\\
			%        &=kq^m\sum_{i=0}^{n-m-1} q^i\\
		\end{split}
	\end{equation}
	Fix $\varepsilon>0$, we can find large  $N$ so that

	$$
		q^N\leq \frac{\varepsilon(1-q)}{k}
	$$
	Then for all $m,n>N$ we have
	$$
		\norm{x_m-x_n}<\frac{kq^{m}}{1-q}<\frac{\varepsilon(1-q)}{k}\frac{k}{1-q}=\epsilon
	$$
	Hence ${x_n}$ is Cauchy, thus convergent to a unique point $x^*\in X$
\end{theorem}

\begin{remark}\rm\nextline
	This theorem is easy to prove, however it's very useful. We may see examples of fixed point by playing with calculator: choose any real number and calculate its $\cos$ value, and continue input the result to $cos$, we may find that the result become stable around $0.7390......$. This is a fixed point of $cos$. $sin$ also has fixed point, which is zero. It can be used to give sufficient condition where Newton method for finding root converges. In study of ODE contraction mapping can be used to guarantee that Picard iteration converges to a certain function. See \href{https://en.wikipedia.org/wiki/Picard–Lindelöf_theorem}{Picard–Lindelöf theorem} on Wikipedia.
\end{remark}
\subsection{Exam}
\subsubsection{Proof of completeness}
Generally there are three steps in proving completeness.
\begin{itemize}
	\item Find a candidate for the "limit" of a Cauchy sequence.
	\item Show that it's indeed the "limit'.
	\item Show that it's still in the space.
\end{itemize}

\newpage
\section{Hilbert Space}
Hilbert space is a special class of Banach space. Apart from completeness and norm, it is also equipped with an additional structure, {\bf inner product}. This allow us to explore nice geometric properties of the space,like orthogonality and angle. We'll see later that this structure resemble Euclidean space in many ways. A Hilbert space is naturally Banach, while the reverse may not be true.
\subsection{Definitions and notations}
\begin{definition}[inner product]\rm\label{Inner product}\nextline
	Let $X$ be a vector space over $\mathbb C$. An {\bf inner product} is a function $\left<\cdot,\cdot\right>:X\times X\xrightarrow{}{\mathbb C}$ satisfying following: $\forall x,y,z\in X,\alpha$ a scalar,
	\begin{itemize}
		\item[1] $\left<x,y\right>={\overline{\left<y,x\right>}},\forall x,y\in X$
		\item[2] $\left<x,x\right>\geq0$
		\item[3] $\left<x,x\right>=0$ iff $x=0$
		\item[4] $\left<x+y,z\right>=\left<x,z\right>+\left<y,z\right>$
		\item[5] $\left<ax,z\right>=a\left<x,z\right>$
	\end{itemize}
\end{definition}
1 is complex conjugation. 2 and 3 is positive-definiteness. 4 and 5 is left-linearity.

\begin{proposition}\rm\nextline
	Let X be a vector space, $\left<\cdot,\cdot\right>$ an inner product on the space, $x,y,z,w\in X$ and $\alpha$ a scalar. Then:
	\begin{itemize}
		\item $\left<x,ay\right>={\overline{a}}\left<x,y\right>$
		\item $\left<x,y+z\right>=\left<x,y\right>+\left<x,z\right>$
		\item $\left<x,ay+z\right>={\overline{a} }\left<x,y\right>+\left<x,z\right>$
		\item $\inne{x+y}{z+w}=\inne{x}{z}+\inne yz+\inne yw+\inne xw$
	\end{itemize}
	Proof is trivial. Left to readers.
\end{proposition}

\begin{proposition}[Cauchy-Schwartz]\rm\nextline
	If $\left<\cdot,\cdot\right>$ is an inner product on $X$,
	then $\forall x,y\in X$,
	$$|\left<x,y\right>|^2\leq\left<x,y\right>\cdot\left<y,y\right>$$

\end{proposition}

\begin{proposition}[induced norm]\rm\nextline
	If $\left<\cdot,\cdot\right>$ is an inner product on $X$,
	then
	$$
		\norm{x}\equiv\inne{x}{x}^\frac{1}{2}
	$$
	defines a norm on X.
\end{proposition}

\begin{definition}[Hilbert space]\rm\label{Hilbert space def}\nextline
	A (complex) Hilbert space $\mathscr{H}$ is a vector space over $\mathbb C$ that is complete in the metric
	$$
		d(x,y)=\norm{x-y}=\inne{x-y}{x-y}^\frac{1}{2}
	$$
\end{definition}
One may also say, Hilbert space is a complete inner product space. The followings are a few examples of Hilbert space.

\begin{example}[Euclidean Space over $\mathbb R$]\rm\nextline
	It happens that 2-D or 3-D vector space over $\mathbb R$ is an example of Hilbert space, under the standard definition of vector inner product.
\end{example}

\begin{example}[Euclidean Space over $\mathbb C$]\rm\nextline
	${\mathbb C}^n$ with inner product
	$$
		\inne{x}{y}=\sum_{i=1}^n {\overline{x_i}}\,y_i
	$$
	is a Hilbert space. This is a generalization of last example. Still, this is a finite-dimensional Hilbert space.
\end{example}

\begin{example}[Sequence space]\rm\nextline
	Complex sequence space:
	$$\ell^2=
		\left\{\{x_n\}_1^\infty:\sum_{k=1}^\infty |x_k|^2<\infty \right\}$$
	with  inner product,denoting $x=\{x_n\}_1^\infty$ and $y=\{y_n\}_1^\infty$
	$$
		\inne{x}{y}=\sum_{k=1}^\infty \,{\overline{x_k}\, y_k}
	$$
\end{example}



\subsection{Hilbert space Geometry}
Structure of inner product allows discussion for nice geometric property of Hilbert spaces. This includes orthogonality, angles and nearest distance etc.

\subsubsection{Orthogonality}
Orthogonality is the generalization of two lines being perpendicular. In euclidean geometry, we have Pythagorean theorem closely related to such property. Results on orthogonality in Hilbert spaces in many ways resemble their Euclidean version.
\begin{definition}[Orthogonality]\rm\nextline
	Let $\mathscr{H}$ be a Hilbert space and $f\in\mathscr{H}$. Say $g$ is orthogonal to $f$ if $\inne{f}{g}=0$, writes $f\perp g$. For two sets $A,B\in \mathscr{H}$, write $A\perp B$ if $a\perp b$ for all $a\in A$ and $b\in B$


\end{definition}

\begin{definition}[Orthogonal Complement]\rm\label{ortho comp}\nextline
	Let $A$ be a set in Hilbert space $\mathscr{H}$. Its orthogonal complement, $A^{\perp}$ is the set of all vectors $f \in \mathscr{H}$ such
	that $f\perp g\,,\forall g \in A$. $A^\perp$ is always a subset of $\mathscr{H}$. Moreover, $A^\perp=\cup_{a\in A}(a)^\perp$, and we can prove that $A^\perp$ is always a closed subset of \hbs.
\end{definition}

\begin{proposition}[Pythagorean Theorem]\rm\nextline
	Let $f_1,\,f_2,\,......,f_n$ be pairwise orthogonal vectors in Hilbert space \hbs. Then
	$$
		\norm{\sum_{k=1}^n f_k}^2={\sum_{k=1}^n \norm{f_k}^2}
	$$
	\textit{proof:}\\
	It suffices to show for n=2, and proceed with induction. Consider $a,b\in \hbp$ with $a\perp b$.
	Then
	\begin{equation}
		\begin{split}
			\norm{a+b}^2&=\inne{a+b}{a+b}\\
			&=\inne{a}{a}+\inne{a}{b}+\inne{b}{a}+\inne{b}{b}\\
			&=\inne{a}{a}+0+0+\inne{b}{b}\\
			&=\norm{a}^2+\norm{b}^2
		\end{split}
	\end{equation}


\end{proposition}

\begin{remark}[Parallelogram equality]\rm\label{parallelogram}\nextline
	Let $a,b\in \hbp$ be arbitrary vectors.
	Then
	\begin{equation}
		\begin{split}
			\norm{a+b}^2&=\inne{a+b}{a+b}\\
			&=\inne{a}{a}+\inne{a}{b}+\inne{b}{a}+\inne{b}{b}\\
			&=\inne{a}{a}+\inne{a}{b}+{\overline{\inne{a}{b}}}+\inne{b}{b}\\
			&=\norm{a}^2+2Re(\inne{a}{b})+\norm{b}^2
		\end{split}
	\end{equation}

	Similarly,
	\begin{equation}
		\begin{split}
			\norm{a-b}^2&=\inne{a-b}{a-b}\\
			&=\inne{a}{a}-\inne{a}{b}-\inne{b}{a}+\inne{b}{b}\\
			&=\inne{a}{a}-\inne{a}{b}-{\overline{\inne{a}{b}}}+\inne{b}{b}\\
			&=\norm{a}^2-2Re(\inne{a}{b})+\norm{b}^2
		\end{split}
	\end{equation}

	Adding up, we obtain {\bf{\emph{ parallelogram equality}}}:
	$$
		\norm{a+b}^2+\norm{a-b}^2=2\norm{a}^2+2\norm{b}^2
	$$


	This equation holds for all Hilbert spaces including 2-d vector space over $\mathbb R$. Readers may find this form identical to the parallelogram equality in that vector space. It is also where the name comes from.
\end{remark}



\subsubsection{Nearest Point Property}
In Euclidean geometry, choosing a point and a line we can find the minimal distance form the point to any point on the line, defined as the distance from the point to the line. This can be generalized, with line extending to a {\bf{\emph{closed convex set}}} and space becoming a Hilbert space.


\begin{definition}[Convexity]\rm\label{convexity}\nextline
	A set $S\subset \mathscr{H}$ is convex if $\forall f,g\in S,\forall t\in\left[0,1\right]$, we have $(tf+(1-t)g)\in S$.

\end{definition}
Intuitively, this means that given any two points in the a convex set, the "segment" connecting the points stays inside the set. Hence a closed convex set is such a set with the property that every convergent sequence in the set converges to a point in the set.

\begin{definition}[Set-point distance]\rm\nextline
	Let \hbs be a Hilbert space, $S\in \mathscr{H}$ be a closed subset, $x\in \hbp$ be arbitrary vector. We define the {\bf{distance}} from $x$ to $S$ to be the infimum of distance between $x$ and element of $S$:

	$$\text{dist}(S,x)\equiv\inf_{s\in S} (\norm{s-x})$$

\end{definition}
\begin{proposition}[Nearest Point Property]\rm\label{Nearest Point Property}\nextline
	Let \hbs be a Hilbert space, $S\in \mathscr{H}$ be a closed subset, $x\in \hbp$ be arbitrary vector. Then there exists a {\bf{unique}} $s_0\in S$ such that
	$$
		\norm{s_0-x}= \text{dist}(S,x)
	$$
	\begin{pf}{}{}
	The result is trivial when $x\in S$, we can choose $s_0=x$ and dist$(S,x)=0$.\\
	When $x\notin S$, the idea of the proof is, by definition of distance especially the "infimum" we have a sequence of points in $S$ whose distance with $x$ converges to dist$(S,x)=0$. Then from the convergence of distance we can show that this sequence is Cauchy, hence by completeness, convergent. Finally by cussedness we conclude that it converges to a point in S and convexity can be used to show uniqueness. \\
	So we choose $x\notin S$ and a sequence $\left\{y_n\right\}_1^\infty$ in $S$ such that $\norm{y_n-x}\xrightarrow{}0$. And let $d=\text{dist}(S,x)$.
	Fix $\varepsilon_0>0$, we can find $N_0\in\mathbb{N}$ such that for all $m,n>N_0$, we have $$\norm{x-y_n}<\frac{\varepsilon_0}{2}\quad and\quad \norm{x-y_m}<\frac{\varepsilon_0}{2}
	$$
	Then for all $m,n>N_0$,
	\begin{equation}
		\begin{split}
			\norm{y_m-y_n}&=\norm{(y_m-x)+(x-y_n)}\\
			&\leq\norm{y_m-x}+\norm{y_n-x}\\
			&<\frac{\varepsilon_0}{2}+\frac{\varepsilon_0}{2}\\
			&=\varepsilon_0
		\end{split}
	\end{equation}
	Thus we conclude that $\left\{y_n\right\}_1^\infty$ is Cauchy. By completeness of \hbs  \, and closedness of $S$ we conclude that it converges to a point $y$ in $S$.\\
	It still remains to check uniqueness. We assume that exists $y_1$ and $y_2$ that satisfies the definition. Then by parallelogram equality:
	\begin{equation}
		\begin{split}
			\norm{(y_1-x)-(y_2-x)}^2&=\norm{y_1-x}^2+\norm{y_2-x}^2-\norm{y_1+y_2-2x}^2\\
			&=2d^2-\norm{y_1+y_2-2x}^2\\
			&=2d^2-2\norm{\frac{y_1+y_2}{2}-x}^2\\
		\end{split}
	\end{equation}
	Now by convexity, $\frac{1}{2}(y_1+y_2)\in S$. Thus
	$$\norm{\frac{y_1+y_2}{2}-x}\geq d$$
	and so
	$$
		\norm{y_1-y_2}^2=\norm{(y_1-x)-(y_2-x)}^2\leq0
	$$
	Hence $y_1=y_2$, giving the uniqueness.
\end{pf}
\end{proposition}



\subsubsection{Projection Theorem}
\begin{definition}[Projection mapping]\rm\nextline
	Let \hbs be a Hilbert space. A mapping $P:\hbp\xrightarrow{}\hbp$ is a projection mapping if $$
		P(Px)=Px,\,\forall x\in\hbp
	$$


\end{definition}


\begin{proposition}[Projection Theorem]\rm\nextline
	Let \hbs be a Hilbert space and M a closed subspace. Then there exists unique pair of projection mapping $P:\hbp\xrightarrow{}M$ and $Q:\hbp\xrightarrow{}M^\perp$ satisfying $x=Px+Qx$ for any $x\in\hbp$, with the following property:
	\begin{enumerate}[(1)]
		\item $x\in M$ if and only if $Px=x,\,Qx=0$
		\item $x\in M^\perp$ if and only if $Px=0,\,Qx=x$
		\item $\norm{Px}^2+\norm{Qx}^2=\norm{x}^2$
		\item $P$ and $Q$ are linear maps
		\item $Px$ is the closest vector in $M$ to x.
		\item $Qx$ is the closest vector in $M^\perp$ to x.
	\end{enumerate}
	\textit{proof:}\\
	We shall use nearest point property. Since $M$ is a closed subspace, it is clearly a convex subset of \hbs. Thus for each $x\in \hbp$ there exists a unique point in $M$ that is closest to $x$. So we define $Px$ to be the unique nearest point in $M$ for each x. Uniqueness of nearest point gives the uniqueness of mapping $P$. $Q$ is then defined as $x-Px$. We should show that this definition of $Q$ indeed gives a mapping from \hbs to $M^\perp$.\\
	Fix $x$, let $m\in M$ be such that $\norm{m}=1$. We must have, for all $a\in\mathbb{C}$:
	\begin{equation}
		\begin{split}
			\norm{Qx}^2&\leq\norm{Qx+am}^2\\
			&=\norm{Qx}^2+|a|^2\norm{m}^2+\inne{Qx}{m}+\inne{m}{Qx}
		\end{split}
	\end{equation}
	Then we choose $a=-\inne{Qx}{m}$ and simplify the equation, we have
	\begin{equation}
		\begin{split}
			0&\leq|a|^2\norm{m}^2+\inne{Qx}{am}+\inne{am}{Qx}\\
			&=|\inne{Qx}{m}|^2\norm{m}^2+\inne{Qx}{-\inne{Qx}{m}m}+\inne{-\inne{Qx}{m}m}{Qx}\\
			&=|\inne{Qx}{m}|^2-
			\overline{\inne{Qx}{m}}\inne{Qx}{m}-
			\overline{\inne{m}{Qx}}\inne{m}{Qx}\\
			&=-|\inne{Qx}{m}|^2
		\end{split}
	\end{equation}
	Thus we must have $\inne{Qx}{m}=0$, so $Qx\in M^\perp$.\\
	We proceed to prove (1) and (2).\\
	First $x\in M$. We have $Qx\in M^\perp$. Note that $x\in M$ and $Px\in M$ we have $Qx=x-Px\in M$. Hence $Qx\in(M\cap M^\perp)={0}$. Thus $Qx=0$ and $Px=x-0=x$. \\
	To see other direction, we simply notice that $x=Px\in M$ by definition.\\
	Proof of (2) is similar.\\
	Now let's prove (3). We shall use the fact that $Px\perp Qx$, giving $\inne{Px}{Qx}=0$
	\begin{equation}
		\begin{split}
			\norm{x}^2&=\inne{x}{x}\\
			&=\inne{Px+Qx}{Px+Qx}\\
			&=\inne{Px}{Px}+\inne{Qx}{Px}+\inne{Px}{Qx}+\inne{Qx}{Qx}\\
			&=\inne{Px}{Px}+\inne{Qx}{Qx}\\
			&=\norm{Px}^2+\norm{Qx}^2
		\end{split}
	\end{equation}
	Proof of (4) is a routine work verifying linearity. Left as an exercise.\\
	Now let's prove (5) and (6).\\
	(3) is given by the construction of P. To see (4), consider $y\in M^\perp$, we have:
	$$
		\norm{x-y}=\norm{Px+Qx-y}=\norm{Px}+\norm{Qx-y}\geq \norm{Px}
	$$
	So minimal distance is $\norm{Px}$, Obtained at $y=Qx$
\end{proposition}


\newpage
\section{Operator and Functional}
\subsection{Linear functionals}\label{Linear Functionals}
Linear functionals are a special class of linear mapping from a vector space to its scalar field.

\begin{definition}[Linear mapping]\rm\nextline
	Let $X$ and $Y$ be vector spaces over the same scalar field. A mapping $\mathrm{\Lambda}:X\xrightarrow{}Y$ is called a {\bf linear mapping} if
	$$
		\Lambda({\alpha x+\beta y})=\alpha\Lambda{x}+\beta \Lambda{y}
	$$
	for all $x,y$ vectors and $\alpha,\beta $ scalars.\\
	When Y is the scalar field, $\Lambda$ is called a {\bf{linear functional}}. In the following text, we assume $Y=\mathbb{C}$ unless mentioned otherwise.
\end{definition}

\begin{definition}[Image, preimage, kernel]\rm\nextline
	Let $T$ be linear mapping :$T:X\xrightarrow{}Y$, let $M\subset X$, $N\subset Y$.\\
	Then the {\bf Image} of X is
	$$
		T(M)\equiv \{y\in Y: y=T(x),x\in M\}\subset Y
	$$
	The {\bf Preimage} of Y is
	$$
		T^{-1}(Y)\equiv \{x\in X: y=T(x),y\in N\}\subset X
	$$
	Specially, {\bf Kernel} of $T$ is defined to be the preimage of $\{0_Y\}$,a set containing only the null-element of Y ($0_Y$ is a scalar 0 when $Y$ is the scalar field:
	$$
		\text{Ker}(Y)\equiv T^{-1}(\{0_Y\})=\{x\in X: T(x)=0_Y,y\in N\}\subset X
	$$
\end{definition}

\subsubsection{Boundedness and continuity}
In this section we shall see a very nice result about linear functional, which links their boundedness and continuity

\begin{definition}[Boundedness]\rm\nextline
	Let $X$ be a nomred vector space. $\Lambda:X\xrightarrow{}\mathbb{C}$ a linear functional. Then $\Lambda$ is {\bf bounded} if there exists a constant $M$, such that for all $x\in X$, we have $$\norm{\Lambda(x)}\leq M\norm{x}$$


\end{definition}

\begin{definition}[Functional norm]\rm\nextline
	For a bounded linear functional $\Lambda:X\xrightarrow{}\mathbb{C}$, we define its norm as follows:

	$$
		\norm{\Lambda}=\sup_{x\in X}{\left\{\norm{\Lambda(x)},\norm{x}\leq1\right\}}
	$$
\end{definition}

\begin{proposition}[An inequality of no name]\rm\nextline
	Let $X$ be a nomred vector space. $\Lambda:X\xrightarrow{}\mathbb{C}$ a bounded linear functional.
	Then
	$$
		\norm{\Lambda(x)}\leq\norm{\Lambda}\norm{x}
	$$
	\begin{pf}{}{}
	Let $e\in X$ be such that $\norm{e}=1$.
	Then:
	\begin{equation}
		\begin{split}
			\norm{\Lambda}\norm{e}&=\sup_{x\in X}{\left\{\norm{\Lambda(x)},\norm{x}\leq1\right\}}\norm{e}\\
			&\leq \norm{\Lambda(e)}\norm{e}\\
			&=\norm{\Lambda(e)}
		\end{split}
	\end{equation}
	Then for arbitrary non-zero vector $x\in\dual X$ we write $x=\norm{x}\frac{x}{\norm{x}}$ and exploit linearity to finish the proof.
\end{pf}
\end{proposition}
Definition for continuous linear functional is similar to the that of continuous function in real-analysis, changing absolute value to corresponding norms. It is a routine exercise to check that this indeed defines a norm.

\begin{definition}[Continuous linear functional]\rm\nextline
	Let $\Lambda:X\xrightarrow{}\mathbb{C}$ be a linear functional. It is continuous at $y\in X$ if for all $\varepsilon>0$, there exists $\delta>0$ such that whenever $x\in\{x\in X, \norm{x-y}<\delta\}$, we have $\norm{\Lambda(x)-\Lambda(y)}<\varepsilon$. If $\Lambda$ is continuous at all $y\in X$, then it is called a {\bf continuous linear map}.


\end{definition}

\begin{proposition}[Continuity $\Longleftrightarrow$ Continuity at 0]\rm\nextline
	Let $\Lambda:X\xrightarrow{}\mathbb{C}$ be a linear functional. Then $\lambda$ is continuous if and only if it is continuous at 0. Proof of this proposition is trivial, left as an exercise.

\end{proposition}

\begin{proposition}[Continuity $\Longleftrightarrow$ Boundedness]\rm\nextline
	Let $\Lambda:X\xrightarrow{}\mathbb{C}$ be a linear functional. Then $\lambda$ is continuous if and only if it is bounded.
	\begin{pf}{Part I: Continuity implies Boundedness}{}
	$(\Longrightarrow)$We first show that continuity implies boundedness. We know $\Lambda$ is continuous at 0, so fix $\varepsilon>0$, we may find $\delta_0>0$ such that $\norm{x}<\delta_0\Longrightarrow \norm{\Lambda(x)}<\varepsilon$. So for any $y\in X$ we have that:
	\begin{equation}
		\begin{split}
			\Lambda(y)&=\Lambda\left(\frac{2\norm{y}}{\delta_0}\left(\frac{\delta_0}{2}\frac{y}{\norm{y}}\right)\right)\\
			&=\frac{2\norm{y}}{\delta_0}\Lambda\left(\frac{\delta_0}{2}\frac{y}{\norm{y}}\right)\\
			&\leq\frac{2\norm{y}}{\delta_0}\varepsilon=\frac{2\varepsilon}{\delta_0}\norm{y}
		\end{split}
	\end{equation}
	Hence $M=2\varepsilon/\delta_0$ gives a bound of the functional.
\end{pf}
\begin{pf}{Part II:Boundedness implies Continuity}{}
	$(\Longleftarrow)$Now we show that boundedness implies continuity. By last proposition we may reduce this to showing that boundedness implies continuity at 0. Let $K$ be such that
	$$
		\Lambda(x) \leq K\norm{x},\quad\forall x\in X
	$$
	Now fix $\varepsilon>0$, let $\delta=\frac{\varepsilon}{2K}$. Then when $x<\delta$, we have
	\begin{equation}
		\begin{split}
			\Lambda(x)&\leq K\norm{x}\\
			&< K \delta\\
			&= K \frac{\varepsilon}{2K}\\&
			=\frac{\varepsilon}{2}<\varepsilon\\
		\end{split}
	\end{equation}
	Hence $\Lambda$ is continuous at 0, which finishes the proof.
\end{pf}
\end{proposition}

\begin{proposition}[Sequential continuity]\rm
	Let $\Lambda:f\xrightarrow{}Y$ be a mapping, $X$ is a metric space and $Y$ is topological space. Then it is continuous if and only if, for any sequence $\{x_n\}$ in $X$ such that $x_n\longrightarrow{}x_0\in X$, we have $f(x)\longrightarrow{}f(x_0)\in Y$
\end{proposition}

\newpage



\subsection{Basic Operator Theory}
\subsubsection{Bounded linear operator}
\begin{definition}[Linear operator]\rm\nextline
	Let $X$, $Y$ be normed vector spaces. A mapping $\Gamma:X\xrightarrow{}Y$ is a linear operator if
	$$
		\Gamma(k_1x_1+k_2x_2)=k_1\Gamma(x_1)+k_2\Gamma(x_2)
	$$
	for all $x_1,x_2\in X$ and $k_1,k_2$ scalars.
\end{definition}

\begin{definition}[Bounded linear operator]\label{continuity of LO}\rm\nextline
	Let $X$, $Y$ be normed vector spaces. Linear operator $\Gamma:X\xrightarrow{}Y$ is bounded if there is a finite constant $C>0$ such that
	$$
		\norm{\Gamma(x)}_Y\leq\norm{x}_X
	$$
	holds for all $x\in X$.
\end{definition}

\begin{definition}[Operator norm]\label{operator norm}\rm\nextline
	Let $X$, $Y$ be normed vector spaces, $\Gamma:X\xrightarrow{}Y$ a bounded linear operator, then

	$$
		\norm{\Gamma}\equiv\sup_{x\in X}{\left\{\norm{\Gamma(x)}_Y,\norm{x}_X\leq1\right\}}
	$$
	The definition is very similar to that for functionals.
\end{definition}

\begin{proposition}[Boundedness and continuity]\rm\nextline
	Let $X$, $Y$ be normed vector spaces, $\Gamma:X\xrightarrow{}Y$ a linear operator. Then the following three are equivalent:
	\begin{itemize}
		\item $\Gamma$ is bounded
		\item $\Gamma$ is continuous
		\item $\Gamma$ is continuous at 0
		\item {\bf Also:} $\Gamma$ is Lipschitz, i.e.$\exists C>0$ with $\norm{Aa-Ab}_Y\leq C\norm{a-b}_X,\,\,\forall a,b\in X$
		\item {\bf Also:} $\Gamma$ is continuous at any $x\in X$

	\end{itemize}
\end{proposition}

\begin{definition}[$\mathscr B$]\rm\nextline
	We define $\mathscr B(X,Y)$ to be the collection of all bounded linear operators from $X$ to $Y$. We also write $\mathscr B(X,X)$ as $\mathscr B(X)$
\end{definition}

\begin{proposition}\rm\nextline
	$\mathscr B(X,Y)$ is normed linear space in operator norm. Specially if $Y$ is Banach then $\mathscr B(X,Y)$ is also Banach.
\end{proposition}

\begin{theorem}\rm\nextline
	Every finite dimensional linear operator is bounded.
\end{theorem}

\begin{theorem}\rm\nextline
	A bounded linear operator attains its inf and sup on  a compact set.
\end{theorem}

%\subsubsection{Adjoints of Hilbert space operators.}



\subsection{Duality}

\subsubsection{Dual space: A nice self-symmetry}
One may notice, that linear maps also form a vector space: multiple of a linear map are also linear, sum of linear maps are also linear. We'll formalize this idea to the concept of dual space.

\begin{definition}[Dual Space]\rm\nextline
	Let $X$ be a Banach space. Define its dual space $X^*$ as follows:
	$$
		X^*=\left\{
		T: T\text{ is a bounded linear functional on } X
		\right\}
	$$
\end{definition}
\begin{proposition}[Dual space of Banach Space]\label{dual space Banach}\rm\nextline
	Dual space of a Banach space is also Banach, under functional norm.\\
	\begin{pf}{}{}

	Here we assume scalar field to be $\mathbb{C}$.\\
	First,to check that $\dual{X}$ is a vector space, it suffices to show that multiple and sum of bounded linear functional remains to be linear and bounded. This part of proof is trivial.\\
	Second, we shall check that $\dual X$ is complete under functional norm.
	To show this, we let $\{T_n\}$ to be a Cauchy sequence in $\dual X$, which means given $\varepsilon>0$, there exists a positive constant $N$ such that for all $a,b>N$, we have
	$$
		\norm{T_a-T_b}<\varepsilon
	$$
	So given any point $x\in X$, we have that
	\begin{equation}
		\begin{split}
			\norm{T_a(x)-T_b(x)}&=\norm{(T_a-T_b)x}\\
			&\leq \norm{T_a-T_b}\norm{x}\\
			&<\varepsilon\norm{x}
		\end{split}
	\end{equation}
	This implies that $\{T_n(x)\}$ is a Cauchy sequence on $\mathbb{C}$. By completeness of $\mathbb{C}$, it converges to a point on $\mathbb{C}$. Note that this works for arbitrary $x\in X$, we can define $T$ to be the pointwise limit in following form:
	$$
		T(x)\equiv\lim_{n\to\infty}{T_n(x)},\quad\forall x\in X
	$$
	We can verify that $T$ is linear:
	\begin{equation}
		\begin{split}
			T(ax)&=\lim_{n\to\infty}{T_n(ax)}\\
			&=\lim_{n\to\infty}{aT_n(x)}\\
			&=a\lim_{n\to\infty}{T_n(x)}\\
			&=aT(x)
		\end{split}
	\end{equation}

	\begin{equation}
		\begin{split}
			T(x)+T(y)&=\lim_{n\to\infty}{T_n(x)}+\lim_{n\to\infty}{T_n(y)} \\
			&=\lim_{n\to\infty}{T_n(x)+T_n(y)}\\
			&=\lim_{n\to\infty}{T_n(x+y)}\\
			&=T(x+y)
		\end{split}
	\end{equation}

	It remains to show that $T$ is bounded. Fix $r>0$, we can find $a\in\mathbb{N}$ such that $\norm{T-T_a}<r$,
	\begin{equation}
		\begin{split}
			\norm{T}&=\norm{T-T_a+T_a}\\
			&\leq\norm{T-T_a}+\norm{T_a}\\
			&=r+\norm{T_a}
		\end{split}
	\end{equation}
	Hence T is also bounded, so $T\in \dual X$, which finishes the proof for completeness.
\end{pf}
\end{proposition}



We have shown that dual space of a Banach space is Banach, what about dual space of a Hilbert space? In the following result, we'll see that the inner product of a Hilbert space allows great elegance in structure of its dual space, which is identity in the sense of isomorphism.

\subsubsection{Riesz representation Theorem}\label{Riesz representation theory}
\begin{theorem}[Riesz representation Theorem]\rm
	A bounded linear functional $T$ on Hilbert space \hbs is uniquely associated with a vector $h_0\in \hbp$ in the sense that
	$$
		T(h)=\inne{h}{h_0},\quad\forall h\in\hbp\quad\text{ and  }\norm{T}=\norm{h}
	$$
	\begin{pf}{}{}
	If $T=0$ we simply choose $h_0=0$.\\
	When $T\neq0$,let $S=Ker(T)$, pick a non-zero vector $w\in S^\perp$, without loss of generality, we may assume $T(w)=1$. Then, for any $x\in \hbp$, we observe for vector $(T(x)w-x)$ that
	$$
		T\left(T(x)w-x\right)=T(w)T(x)-T(x)=T(x)-T(x)=0
	$$
	Therefore $(T(x)w-x)\in S$, which means $(T(x)w-x)\perp w$. Hence
	$$
		\inne{(T(x)w-x)}{w}=0,\quad\forall w\in\hbp
	$$
	By (left) linearity of inner product,
	$$
		\inne{x}{w}=T(x)\inne{w}{w}=T(x)\norm{w}^2
	$$
	Thus
	$$
		T(x)=\frac{\inne{x}{w}}{\norm{w}}=\inne{x}{\frac{w}{\norm{w}^2}}
	$$
	So $h=w/\norm{w}^2$ is the desired vector. Also,
	\begin{equation}
		\begin{split}
			\norm{T}&=\sup_{x\in \hbp}{\left\{\norm{T(x)},\norm{x}\leq1\right\}}\\
			&=\sup_{x\in \hbp}{\left\{\norm{\inne{x}{h}},\norm{x}\leq1\right\}}\\
			&=\norm{\inne{\frac{h}{\norm{h}}}{h}}\\
			&=\norm{h}
		\end{split}
	\end{equation}
	The result shows that \hbs=$\dual\hbp$ in the sense that the ma from $h$ to corresponding linear functional is an isometry: $\norm{h}=\norm{T}$
\end{pf}
\end{theorem}
\begin{remark}[Alternative proof of Riesz Representation theorem]\rm\nextline
	The proof above is purely algebraic, in the sense that no analysis is used. However, one may realise that the proof relies heavily on the construction of $T(x)w-x$, which may not be that easy to thought of. Hence another completely different proof is presented here. 
	\begin{pf}{Riesz Representation theorem}{}
	The idea of the proof is to construct a sequence of points that finally converges to the desired $h_0$.\\
	If $\norm{T}=0$, we simply choose $h_0=0$.\\
	If not, without loss of generality, let's consider operators with norm equal to 1.We first show existence of such $h_0$. Let $T\in\hbp^*$ be such that $\norm{T}=1$.\\
	Now by definition of operator norm, specializes to 1-Euclidean norm on $\real$:
	$$
		\norm{T}=\sup_{x\in\hbp}\frac{\norm{Tx}}{\norm{x}} =\sup_{\norm{x}=1}\norm{Tx}
	$$
	The existence of {\bf supremum} in the definition guarantees that we can find a sequence of points $(x_n)_0^\infty$ in $\{x\in\hbp:\norm{x}=1\}$ that gives
	$$
		\lim_{n\to\infty}{T(x_n)}=1
	$$
	The norm can be removed here as we're discussing functionals \func{T}{\hbp}{\real}, and for each $y\in\hbp$ if $T(y)<0$ there is $(-y)\in\hbp$ and $T(-y)=-T(y)>0$ by linearity.\\
	We claim that $(x_n)_1^\infty$ is Cauchy. To prove this, we first notice  parallelogram equality in Hilbert space:
	$$
		2\norm{a}^2+2\norm{b}^2=\norm{a+b}^2+\norm{a-b}^2
	$$
	we have that
	\begin{equation}
		\begin{split}
			\norm{x_m-x_n}^2&=2\norm{x_m}^2+2\norm{x_n}^2-\norm{x_m+x_n}^2\\
			&=2\times1+2\times1-\norm{x_m+x_n}^2\\
			&=4-\norm{x_m+x_n}^2\\
			&=4-\norm{T}^2\norm{x_m+x_n}^2\\
			&\leq4-\norm{Tx_m+Tx_n}^2\\
			&=\leq4-(Tx_m+Tx_n)^2
		\end{split}
	\end{equation}
	Taking $n$ and $m$ to infinity we have that $\norm{x_m-x_n}\to 0$, which means that $(x_n)_1^\infty$ is Cauchy.
	\end{pf}
\end{remark}
\begin{theorem}[Dual space of Hilbert space]\label{dual space Hilbert}\rm\nextline
	Let $\hbp$ be a Hilbert space, then its dual space $\hbp^*$ is isomorphic to  itself: $\hbp\cong\hbp^*$, moreover, the natural norm on them is an isometry, which means that the operator norm of a linear functional equals the element that generates this functional by inner product. This is an immediate corollary from results of \hyperref[Riesz representation theory]{Riesz representation theory}.
\end{theorem}

\subsubsection{Dual space of lp space}\label{lp dual}
In this section, we consider sequence spaces, $l_p$ space. Specially, when $p=2$, the space is Hilbert. In other cases,  when $p\not=2$, what happens? One may realise that $q=p=2$ is precisely a solution to $1/p+1/q=1$, and $p=q$ gives the self-duality. We will investigate this intuition here and prove that the dual space of $l_p$ is precisely $l_q$, where $1/p+1/q=1$,$p,q\in\real$. One should pay attention here, that $p=\infty$ is slightly different, the result may not hold in given $p=\infty$.

\begin{theorem}[Dual of \texorpdfstring{$\ell^p$}.]\rm\nextline
	Let $p\in (1,\infty)$, then $(\ell^p)^*\cong\ell^q $, where $1/p+1/q=1$.\\
	\prf\\
	The proof has two parts. First we show that every element  $y\in\ell^q$ uniquely defines a linear linear functional \func{\Lambda_y}{X}{\real}, then we prove surjectivity, which means that every functional $l\in(\ell^p)^*$ can be uniquely represented by an element in $\ell^q$. Equation $1/p+1/q=1$ and its variation like $p+q=pq$ and  $1+q/p=q$ will be useful at some steps.
	\begin{pf}{STEP I}{}
	Let $y=\sequ{y_n}\in\ell^q$,,define \func{\Lambda_y}{X}{\real}, by 
	$$\Lambda_y(x)=\sum_{k=1}^{\infty}x_ny_n,\,\,\forall x=\sequ{x_n}\in\ell^p$$
	We claim that:\begin{itemize}
		\item $\Lambda_y\in(\ell^p)^*$, which means that $\Lambda_y$ is a bounded linear functional.
		\item $\norm{\Lambda_y}\equiv\sup\{|\Lambda_y(x)|:\norm{x}_p=1\}=\norm{y}_q$, i.e. the map $y\rightarrow\Lambda_y$ is an isometry.
	\end{itemize}
	Linearity immediately follows from definition. To check boundedness, we notice that
	\begin{comment}
	\begin{equation}
		\begin{split}
		\left|\Lambda_y(x)\right|&=\left|\sum_{k=1}^{\infty}x_ny_n\right|\\
		&\leq\sum_{k=1}^{\infty}|x_ny_n|\\
		&\leq\norm{xy}_1\\
		\text{(Hölder)}\quad&\leq\norm{x}_p\norm{y}_q
		\end{split}
	\end{equation}
	\end{comment}
	\begin{equation}
		\left|\Lambda_y(x)\right|=\left|\sum_{k=1}^{\infty}x_ny_n\right|
		\leq\sum_{k=1}^{\infty}|x_ny_n|
		\leq\norm{xy}_1
		\quad\leq\norm{x}_p\norm{y}_q
	\end{equation}
	Now, plugging in any $x\in\ell^p$ with norm 1 we get boundedness, and to see isometry we need to find a proper $x$ in $\ell^p$ that saturates the supremum.\\
	The candidate here is $x=\sequ{x_n}$, with $x_n=sign(y_n)\cdot(y_n)^{q-1}$. We need to check followings:
	\begin{itemize}
		\item $x\in\ell^p$
		\item $|\Lambda_y(x)|=\norm{y}_q\norm{x}_p$
	\end{itemize}
	First, we compute the p-norm of $x$:
	\begin{equation}
		\begin{split}
			\norm{x}_p&=\left(\sum_{k=1}^\infty |x_n|^p\right)^{1/p}=\left(\sum_{k=1}^\infty |sign(y_n)\cdot(y_n)^{q-1}|^p\right)^{1/p}\\
			&=\left(\sum_{k=1}^\infty |(y_n)^{q-1}|^p\right)^{1/p}=\left(\sum_{k=1}^\infty |(y_n)|^q\right)^{1/p}\\
			&=\left(\sum_{k=1}^\infty |(y_n)|^q\right)^{(1/q)\cdot (q/p)}=\norm{y}_q^{q/p}<\infty
		\end{split}
	\end{equation}
	This gives that $x\in\ell^p$. Then
	\begin{equation}
		\begin{split}
			\Lambda_y(x)&=\sum_{k=1}^{\infty}x_ny_n\\
			&=\sum_{k=1}^{\infty}(sign(y_n)\cdot(y_n)^{q-1}\cdot y_n)\\
			&=\sum_{k=1}^{\infty}|y_n|^q\\
			&=\norm{y}_q^q
		\end{split}
	\end{equation}
	Thus we have that 
	\begin{equation}
		\begin{split}
			\frac{|\Lambda_y(x)|}{\norm{x}}&=\frac{|\norm{y}_q^q|}{\norm{y}_q^{q/p}}\\
			&=\norm{y}_q^{q-{q/p}}\\
			&=\norm{y}_q
		\end{split}
	\end{equation}
	Which shows isometry. \\
\end{pf}
\begin{pf}{STEP II}{}
	Now let us assume $l\in(\ell^p)^*$. We define natural basis of an $\ell^q$ space: $e_n=\sequ{(e_n)_k}$ , where $(e_n)_k=1$ if $k=n$, otherwise $(e_n)_k=0$. Each $e_n$ is a sequence with all entries zero, except at $k-th$ entry the value is 1. Now consider a sequence of $y_n=\sum_{k=1}^nl(e_k)\cdot e_k$,i.e. $y_n$ is a sequence where $(y_n)_k=l(e_k)$ for $k\leq n$. By denoting $x^n$ to be the sequence with first n-terms equals to corresponding term of x and any term afterwards set to zero, we claim the following three things:
	\begin{itemize}
		\item $y_n\in\ell^q$ for all $n\in\natu$. 
		\item $l(x^{(n)})=\sum_{k=1}^nx_k(y_n)_k$
		\item $\lim_{n\to\infty}\norm{l(x)-xy_n}=0$
	\end{itemize}
	First, note that each $y_n$ is a finite sum, its q-norm is finite, hence $y_n\in\ell^q$ for all $n\in\natu$.  Then we notice that we can express $x^(n)$ as a finite summation:
	$$
	x^{(n)}=\sum_{k=1}^nx_ke_k\implies l(x^{(n)})=\sum_{k=1}^nl(x_ke_k)=\sum_{k=1}^nx_k\cdot l(e_k)
	$$
	Then the second statement is clear.
	$$
	\sum_{k=1}^nx_k(y_n)_k=\sum_{k=1}^nx_k\cdot l(e_k)=l(x^{(n)})
	$$
	For the last statement, we use separability of any $\ell^p$ space and express $x$ uniquely by canonical Schauder basis $e_k$: $x=\sum_{k=1}^\infty x_ke_k$. Thus for any $\varepsilon>0$, an index $s\in\natu$ such that 
	$$
	\norm{y_nx-y_nx^{(n)}}\leq
	\norm{x-x^{(n)}}\cdot\norm{y_n}
	<\varepsilon/2
	$$
	Then, by continuity (boundedness) of $l$, we can choose another index $t\in\natu$ such that 
	$$
	\norm{l(x)-l(x^{(n)})}<\varepsilon/2
	$$
	Then 
	\begin{equation}
		\begin{split}
			\norm{l(x)-y_nx}&=\norm{l(x)-l(x^{(n)})+y_nx^{(n)}-y_nx}\\
			&\leq\norm{l(x)-l(x^{(n)})}+\norm{y_nx^{(n)}-y_nx}\\
			&<2\varepsilon/2=\varepsilon
		\end{split}
	\end{equation}
	By letting $n\to\infty$ we have that that $x\lim_{n\to\infty}y_n=l(x)$ for all $x\in\ell^p$. Thus we can define $y=\lim_{n\to\infty}y_n$. $y$ is an element of $\ell^q$ since $\sequ{y_n}$ is Cauchy, since the different of any $y_m$ and $y_n$ is the partial sum from m-th term to n-th term of a sequence that converges in q-norm, and thus can be arbitrarily small when n and m goes to infinity. Existence and uniqueness of $y$ is then follows from completeness of $\ell^q$. This shows that every linear functional in $(\ell^p)^*$ uniquely gives an element in $\ell^q$, which proves that $\Lambda_y$ is a surjection, hence we have an isomorphism as stated: $(\ell^p)^*\cong\ell^q$.
\end{pf}
\end{theorem}

\newpage
\subsubsection{Dual operator}\label{dual operator}
\begin{definition}[dual operator]\rm\nextline
	Let $X$, $Y$ be Banach spaces, with norm $\norm{\cdot}_X$ and $\norm{\cdot}_Y$. Let A be an operator from $X$ to $Y$ : \func{A}{X}{Y}. \\
	Then dual operator of $A$, \func{A^*}{Y^*}{X^*} is defined as follows:
	$$
	A^*y^*=y^* A:x\rightarrow \real,\quad \forall y^*\in Y^*
	$$
	Adopting the notation of bracket, which means to write a linear functional $l$ acting on an element $x$: $l(x)$ to be $\inne{l}{x}$, dual operator can be rewritten in this way:
	$$
	\inne{A^*y^*}{x}=\inne{y^*}{Ax}
	$$
\end{definition}
\begin{remark}\rm\nextline
	\rm This bracket here does not stand for inner product in Banach space. However it's used in a way such that this notation coincides with the adjoint under Hilbert space. This means that under Hilbert space context, one can think of this as an inner product, as every linear functional can be uniquely represented by an element in Hilbert space by Riesz representation theorem.
\end{remark}
\newpage
\section{Hahn-Banach Theorem}
One of the most import results in functional analysis, {\bf Hahn-Banach theorem} is a theorem dealing with extending linear maps from a subspace to the whole space. The theorem says that any bounded linear functional defined on a subspace can be  extended to the whole space, while preserving the norm. The result does not rely on completeness of the space, so it's a result for all normed linear spaces. The proof of this theorem involves using a version of \textit{\bf axiom of choice (AC) }, Zorn's lemma. We shall review this lemma first.

\begin{definition}[Partial Order]\rm \nextline
	A partial order on set $X$, is a binary relation, written generically $\leq$, satisfying following property.
	\begin{itemize}
		\item transitivity: if $a\leq b$ and $b\leq c$ then $a\leq c$
		\item reflexivity: $a\leq a$
		\item anti-symmetry: if $a\leq b$ and $b\leq a$ then $a=b$

	\end{itemize}
	If we also have that for any $a$ and $b$, either $a\leq b$ or $b\leq a$, then we say $\leq$ is a total order.

\end{definition}

\begin{definition}[Upper bound]\rm\nextline
	Let $X$ be a set partially ordered by $\leq$ and $Y\subset X$, we say an element $x\in X$ is an {\bf upper bound} of $Y$ if $y\leq x\,\,\forall y\in Y$.

\end{definition}

\begin{definition}[Maximal element]\rm\nextline
	Let $X$ be a set partially ordered by $\leq$ and $Y\subset X$. say $x\in X$ is a maximal element of $X$ if $x\leq m$ implies $m=x$.

\end{definition}
\begin{lemma}[Zorn's lemma]\rm\nextline
	If $X$ is a nonempty partially ordered set with the
	property that every totally ordered subset of $X$ has an upper bound in $X$, then $X$ has
	a maximal element.
\end{lemma}

\begin{theorem}[Hahn-Banach Theorem]\rm\nextline
	Let $X$ be a normed vector space over $\mathbb F$ ($\mathbb{C}$ or $\mathbb{R}$), $Y$ is a proper subspace of $X$. If $T_0:Y\xrightarrow{}\mathbb{F}$ is a bounded linear functional, then there exists a bounded linear functional $T:X\xrightarrow{}\mathbb{F}$ satisfying:
	\begin{itemize}
		\item $T(y)=T_0(y)$ for all $y\in Y$
		\item $\norm{T}=\norm{T_0}$
	\end{itemize}
\end{theorem}
To prove the theorem, the idea is first to show that we can extend linear functional by one dimension, with induction to show that extension can be done to "arbitrarily high dimension". Then by using Zorn's lemma we show that such extension "reaches" every dimension of the space. We first provide real version of the theorem.

\begin{lemma}[one-dimensional extension]\label{ODEX}\rm\nextline
	Let $X$ be a normed vector space over $\mathbb F$ ($\mathbb{C}$ or $\mathbb{R}$), $Y_n$ is a proper subspace of $X$. Let $v\in X\backslash Y$, $X_{n+1}=\{x+hv:x\in X_n,\,h\in\mathbb{C}\}$
	. If $T_n:Y\xrightarrow{}\mathbb{F}$ is a bounded linear functional, then there exists a bounded linear functional $T_{n+1}:X_{n+1}\xrightarrow{}\mathbb{F}$ satisfying:
	\begin{itemize}
		\item $T_{n+1}(x)=T_n(x)$ for all $x\in X_n$
		\item $\norm{T_{n+1}}=\norm{T_n}$
	\end{itemize}
	\textit{proof}:\\
	Define linear functional $P:X_{n+1}\xrightarrow{}\mathbb{R}$ by
	$$
		P(x+kv)=T_n(x)-Ck,\,\forall\,x\in X_n,\,k\in\mathbb{R}
	$$
	where C is a constant to be determined.
	First we shall check linearity, which is left as an exercise.
	Then we shall show that we can find a proper constant $C$ so that $\norm{P}=\norm{T_n}$. Note that $X_n\subset X_{n+1}$, so we have
	\begin{equation}
		\begin{split}
			\norm{P}&=\sup_{x\in X_{n+1}}(\{|Px|:\norm{x}=1\})\\
			&\geq\sup_{x\in X_{n}}(\{|Px|:\norm{x}=1\})\\
			&=\sup_{x\in X_{n}}(\{|T_nx|:\norm{x}=1\})\\
			&=1
		\end{split}
	\end{equation}
	So by choosing $C$ such that $P(x+kv)\leq \norm{x+kv}$ for any $x\in X_n$ and $k\in \mathbb{R}$, we will have that $\norm{P}\leq 1$, giving $\norm{P}=1$. Thus it remains to show that we can find such a constant $C$.\\
	We aim to find $C$ such that
	$$
		|P(x+kv)|=|T_n(x)-Ck|\leq \norm{x+kv},\,\forall x\in X_n,\,\forall k\in\mathbb{R}
	$$
	Hence,
	$$
		T_n(x)-\norm{x+kv}\leq Ck\leq T_n(x)+\norm{x+kv},\,\forall x\in X_n,\,\forall k\in\mathbb{R}
	$$
	Note that for all $x,y\in X_n$ we have:
	\begin{equation}
		\begin{split}
			T_nx-T_ny&=T_n(x-y)\\
			&\leq \norm{x-y}\\
			&=\norm{(x+kv)-(kv+y)}\\
			&\leq\norm{x+kv}+\norm{y+kv}
		\end{split}
	\end{equation}
	Thus
	$$
		l^-=\sup_{x\in X_n,k\in\mathbb{R}}(T_n(x)-\norm{x+kv})\leq  \inf_{x\in X_n,k\in\mathbb{R}}(T_n(x)+\norm{x+kv})=l^+
	$$
	Hence we can always find a $C$ such that
	$$
		T_n(x)-\norm{x+kv}\leq l^-\leq Ck\leq l^+ \leq T_n(x)+\norm{x+kv},\,\forall x\in X_n,\,\forall k\in\mathbb{R}
	$$
	Which finishes the proof.
\end{lemma}
\subsection{Proof of Hahn-Banach theorem,real case}
Starting from $T_0:Y\xrightarrow{}\mathbb{R}$, by \ref{ODEX} we can define  $T_{n+1}$ to be the one-dimensional extension of $T_n$ for any $n\in\mathbb{N}$, with domain $Y_{n+1}$ extended from $Y_n$, for convenience we let$Y_0=Y$.
Then consider the set
$$
	M=
	\left\{
	(T_n,Y_n),n\in\mathbb{N}
	\right\}
$$
which can be partially ordered by $\leq$ defined as
$$
	(T_a,Y_a)\leq(T_b,Y_b)\,\text{ if } Y_a\subset Y_b,\,\text{ and } T_b=T_a\, \text{ on }\, Y_a
$$
Now let $S=\{(T_i,Y_i),i\in I\}$ (where $I$ is the index set) be a totally ordered subset of $M$. Consider $Y'=\cup_{i\in I}Y_i$ with $T'(x)=T_i(x)$ if $x\in Y_i$, we have that $(T',Y')\in M$ is an upper bound of $S$. By Zorn's lemma, we know that $M$ has a maximal element, denoted as $(T_\infty,Y_\infty)$. We claim that $Y_\infty=X$, because if not, we can do one-dimensional extension to $Y_\infty$, resulting in $X\subset Y_\infty+1$, contradicting with maximality. Thus we have $Y_\infty=X$, and  $T_\infty$ is the desired extension to $X$.

\subsection{Proof of Hahn-Banach theorem,complex case}
To prove the statement for complex case, we shall exploit a connection between real valued functional and complex one.\\
\begin{proposition}\rm\nextline
	Let $T:X\xrightarrow{}\mathbb{C}$ be a complex linear functional. Define $u(x)=Re(T(x))$ for all $x\in X$. Then
	\begin{itemize}
		\item $u(x)$ is a real-valued linear functional
		\item $T(x)=u(x)-iu(ix)$
		\item $\norm{u}=\norm{T}$
	\end{itemize}
	Moreover, given any linear functional $u(x)$, $T(x)=u(x)-iu(ix)$ defines a complex linear functional
	\textit{proof:}\\
	The first two are very easy to show. The hardest part is on the third statement. We first show that $\norm{T}\geq\norm{u}$:
	\begin{equation}
		\begin{split}
			\norm{T}^2&=\sup\{|T x|^2,\norm{x}=1\}\\
			&=\sup\{|u(x)-iu(ix)|^2,\norm{x}=1\}\\
			&=\sup \{[u(x)]^2+[u(ix)]^2,\norm{x}=1\}\\
			&\geq \sup \{[u(x)]^2,\norm{x}=1\}\\
			&=\norm{u}^2
		\end{split}
	\end{equation}
	On the other hand, pick any $x\in X$ with $\norm{x}=1$, denote $T(x)=re^{i\theta}$, we have
	$$|T(x)|=|e^{-i\theta}||T(x)|=|T(e^{-i\theta}x)|=|u(e^{-i\theta}x)-iu(e^{-i\theta}x)|$$
	\begin{equation}
		\begin{split}
			|T(x)|&=|e^{-i\theta}||T(x)|\\
			&=|T(e^{-i\theta}x)|\\
			&=|u(e^{-i\theta}x)-iu(e^{-i\theta}x)|
		\end{split}
	\end{equation}
	But we have that $T(e^{-i\theta}x)=r\in\mathbb{R}$, thus
	$$
		|T(e^{-i\theta}x)|=|Re(T(e^{-i\theta}x))|=|u(e^{-i\theta}x)|\leq\norm{u}
	$$
	Hence $\norm{T}=\norm{u}$.
\end{proposition}
The remaining part of the proof is simply combining last result and the proof of real case. However, last result also gives a insight on complex bounded linear functional
\newpage
\section{Uniform Boundedness principle}
Uniform boundedness principle is sometimes called Banach–Steinhaus theorem. In its basic form, it asserts that for a family of bounded linear operators  whose domain is a Banach space, pointwise boundedness is equivalent to uniform boundedness in operator norm.

\begin{theorem}[Uniform Boundedness principle]\label{UBP}\rm\nextline
	Let $X$ be a Banach space, $Y$ a normed vector space. Let $F$ be a collection of bounded linear operators from $X$ to $Y$. Then if
	$$
		\sup_{f\in F}\norm{fx}<\infty,,\,\forall x\in X
	$$
	Then
	$$
		\sup_{f\in F}\norm{f}<\infty
	$$
\end{theorem}
To prove this theorem we shall use {\textbf{Baire category theorem}}.

\begin{definition}[Nowhere dense]\rm\nextline
	A set $S$ in metric space $X$ is {\bf nowhere dense} if its closure has empty interior. i.e. $\overline{S}\strut^\mathrm{o}=\emptyset$.
\end{definition}

\begin{theorem}[Baire Category Theorem]\label{BCT}\rm\nextline
	A complete metric space is not countable union of nowhere dense sets.
\begin{pf}{Baire Category Theorem}{}
	The idea of the proof is to construct a Cauchy sequence in the space with no limit point, giving contradiction. First, let $M$ be a complete metric space. Assume
	$$M=\bigcup_{n=1}^\infty A_n$$
	We know that $A_1$ is nowhere dense, which means $\overline{A_1}\strut^\mathrm{o}=\emptyset$. Since $\overline{A_1}$ is closed, $M\backslash \overline{A_1}$ is open, we may find open ball $B_1$ with radius $r_1<1$ such that $B_1\cap\overline{A_1}=\emptyset$. Clearly we have that $B_1\not\subset\overline{A_2}$ otherwise $\overline{A_2}$ has non-empty interior. So we have that $(M\backslash\overline{A_2})\cap B_1$ is open and non-empty. Now we may choose open ball $B_2\subset((M\backslash\overline{A_2})\cap B_1)\subset B_1$ with radius $r_2<1/2$. \\
	We repeat this process, so that $B_n\subset((M\backslash\overline{A_n})\cap B_{n-1})\subset B_{n-1}$ is an  open ball with radius $r_n<2^{1-n}$. and name the center of the open ball $B_i$ to be $x_i$. Clearly, $\{x_i\}$ gives a Cauchy sequence (why?). Thus it converges to a point $x$. Since $x$ is a limit point in open ball $B_j$, it has  the property that
	$$x\in \overline{B_{j+1}}\subset B_j\subset (M\backslash\overline{A_j})\subset(M\backslash A_j),\,\forall j\in\mathbb{N}$$
	So we have that $x\not\in A_j,\,\forall j\in\mathbb{N}$.
	So
	$$
		x\not\in \bigcup_{j=1}^{\infty}A_j=M
	$$
	Which leads to contradiction.
	\end{pf}
\end{theorem}

\begin{pf}{Uniform Boundedness principle}{}
Let $A_n=\{
	x\in X, \norm{fx}\leq n,\,\forall f\in F
	\},\,n\in\mathbb{N}$. By assumption we have $\bigcap_{n=1}^\infty A_n=X$.\\
We claim that there exists some $j\in \mathbb{N}$ such that $A_j$ is non-empty and closed. To see this, first by by Baire category theorem, there is some $A_j$ such that $\overline{A_j}\strut^\mathrm{o}\not=\emptyset$. Then let $\{x_m\}$ be a Cauchy sequence in $A_j$ with $x_n\xrightarrow{}x$, then by continuity of $f$, $\norm{fx}=\lim_{n\to\infty}\norm{fx_m}\leq n,\,\forall f\in F$. So $x\in A_j$, hence $A_j$ is closed, thus $\overline{A_j}=A_j$, $A_j\strut^\mathrm{o}=\overline{A_j}\strut^\mathrm{o}\not=\emptyset$. So we can choose a point $p$ from interior of $A_j$, and $\varepsilon>0$ such that open ball $B_{\varepsilon}(p)\subseteq A_j$.\\
Now for any $x<\norm{\varepsilon}$ with any $T\in F$ we have
$$
	\norm{T(x)}=\norm{T(x+p-p)}=\norm{T(x+p)-T(p)}\leq\norm{T(x+p)}+\norm{T(p)}\leq n+n=2n
$$
So for any non-zero vector $x\in X$, we have
$$
	\norm{T(x)}=\frac{\norm{x}}{\varepsilon}\norm{T(\varepsilon \frac{x}{\norm{x}})}\leq\frac{2n}{\varepsilon}\norm{x}
$$
This holds for any $T\in F$, thus
$$
	\sup_{f\in F}\norm{f}\leq\frac{2n}{\varepsilon}<\infty
$$
\end{pf}
A simple corollary of the theorem is Banach limit.
\begin{corollary}[Banach Limit]\label{Banach Steinhaus}\rm\nextline
	Let $T_n:X\xrightarrow{}Y$ be a sequence of operators, where $X$ and $Y$ are Banach spaces. Suppose $\{T_n\}$ converges pointwise,
	then these pointwise limits define a bounded linear operator $T$.
\end{corollary}
\newpage
\section{Open mapping theorem}
\begin{comment}
\bsegin{comment}
\begin{definition}[Open map]\label{open map}\nl
	An open map is one for which the image of every open set is open.
\end{definition}
\begin{theorem}[Open Mapping Theorem]\label{OMT}\nl
	Let $X,Y$ be Banach spaces, $T:X\xrightarrow{}Y$ a continuous surjective linear map. Then $T$ is an open map.
\end{theorem}
\eend{comment}

\begin{definition}[Open Ball]\nl
	An open ball in normed linear space $X$ with radius $r>0$ centered at $x\in X$ is
	$$
		B_X(x,r)=\{y\in X:\norm{y-x}_X<r\}
	$$
	Also, when $x=0$ we write
	$$
		B_X(0,r)\equiv B_x(r)
	$$
\end{definition}

\begin{definition}[Open map]\label{open map}\nl
	Let $X$, $Y$ be linear spaces. \func{A}{X}{Y} is {\underline{open}} if $A(U)\subset Y $ is open.
\end{definition}

\begin{remark}\hfill

	\begin{itemize}
		\item $A$ being continuous means $A^{-1}(V)\subset{X}$ open $\forall V\subset Y$ open.
		\item $A$ being continuous need not be open. e.g. $Ax\stackrel{def}{=}0\in Y$
	\end{itemize}
\end{remark}

\begin{theorem}[Open Mapping Theorem]\label{OMT}\nl
	Let $X,Y$ be Banach, $A\subset\mathcal{L}(X,Y)$. Then:
	\begin{itemize}
		\item[i)] if $A$ is surjective, $A$ is open.
		\item[ii)] if $A$ is bijective, then $A^{-1}\in \mathcal{L}(X,Y)$. (Inverse operator theorem)
	\end{itemize}
\end{theorem}

\begin{remark}\nl
	ii) important in application. If $A\in\mathcal{L}(X,Y)$ is bijective then \func{A^{-1}}{X}{Y} liner is easy (why?). The point is $A^{-1}$ is also bounded, or equivalently continuous.
\end{remark}

The main step of the proof is the following:
\begin{lemma}[$A$ as in i)]\nl
	$\exists r>0$ s.t. $B_Y(r)\subset \overline{A(B_x(1))}$
	\begin{pf}{}{}\rm
		Since $A$ is surjective
		$$
			Y=\bigcup_{k=1}^\infty A(B_X(k))
		$$
		Since $Y$ is complete, by  Baire Category theorem, $\exists k_0$ s.t.
		$$ int(\overline{A(B_X(1))})\neq\emptyset$$
		So by surjectivity of $A$, one can find $y_0=Ax_0\in Y$, $r_0>0$ s.t.
		$$ \underbrace{B_Y(y_0,r_0)}_{=Ax_0+B_Y(r_0)}\subset \overline{A(B_X(k_0))}$$
		By linearity of $A$,
		\begin{equation}\nonumber
			\begin{split}
				B_Y(r_0)&\subset\overline{A(B_X(k_0))}-Ax_0 \\ &=\overline{A(B_X(k_0)-x_0)}\\
				&\subset \overline{A(B_X(k_0+M))}   \\
				&=(k_0+M)\overline{A(B_X(1))}\\
			\end{split}
		\end{equation}
		Where $M\stackrel{def}{=}\norm{x_0}_X$. So pick $r=\frac{r_0}{k_0+M}$.
	\end{pf}

\end{lemma}
Proof of theorem:
\begin{pf}{}{}
	i) Pick $r$ as in Lemma.\\
	Claim: $B_Y(r/2)\subset A(B_X(1)))$.\\
	If claim holds, then for $U\subset X$ open, pick $x_0\in U$, $s>0$ small so that $B_X(x_0,s)\subset U$. Letting $y_0\stackrel{def}{=}Ax_0$, get
	$$
		B_Y(y_0,rs/2)=y_0+sB_Y(r/2)\stackrel{claim}{\subset}Ax_0+sA(B_X(1)\stackrel{lin.}{=}A(B_X(x_0,s))\subset A(U)
	$$
	which proves i). To see i) $\implies$ ii), it's enough to show that $B=A^{-1}:Y\to X$ is continuous; but for any $U\subset X$ open, $B^{-1}(U)=(A^{-1})^{-1}(U)=A(U)$
	which is open by i). $\square$
\end{pf}
Proof of claim:
\begin{pf}{}{}
	Fix $y\in B_Y(r/2)$. Need to show: $y=Ax$ for some $x\in X$ with $\norm{x}_X<1.$\\
	We construct a sequence $(x_k)\subset X$ with
	$$
		\sum_{k=1}^\infty \norm{x_k}_X<1\,\,\text{and}\,\,\sum_{k=1}^\infty Ax_k\stackrel{wrt\norm{\cdot}_Y}{\longrightarrow{}}y,\,n\to\infty
	$$
	By completeness of $X$, $\sum_{k=1}^\infty x_k\stackrel{def.}{=}x$ exists, $x\in B_X(1)$ and by continuity of $A$,
	$$Ax=\sum_{k=1}^\infty Ax_k=y$$
	By lemma above,
	$$\forall s>0, B_Y(sr)\subset \overline{A(B_X(s))}\,\,(*)$$
	$s=1/2$. Pick $x_1\in B_X(1/2)$ s.t. $\norm{Ax_1-y}<r/2$. Now set $y_1=y-Ax(\in B_X(r/2)$. Iterate. Assume that for some $\geq 1$ have $x_1,......x_k,y_1,......y_k$ s.t.
	$$
		\forall 1\leq\tilde{k}\leq k:\,\,\norm{\tilde{x_k}}_X<2^{-k},\,y_{\tilde{k}}=y_{\tilde{k}-1}-Ax_{\tilde{k}}\in B_Y(2^{-\tilde{k}}r
	$$
	Then using $(*)$ with $s=2^{-(k+1)}$ find $x_{k+1}\in B_X(2^{-(k+1)})$ such that
	$$
		y_{k+1}\stackrel{def}{=}y_k-Ax_{k+1}\in B_Y(2^{-(k+1)}r
	$$
	This yields $\sum{k=1}^{\infty}\norm{x_k}_X<1$ and
	$$
		y-\sum_{k=1}^n Ax_k=y_1-\sum_{k=2}^n Ax_k=...=y_n\to0\,(n\to \infty)\quad\square
	$$
\end{pf}

\begin{example}[Equivalence of Norm]\label{Equinorm}\nl
	Let $X=Y$, with norms $\norm{\cdot}_1$ and $\norm{\cdot}_2$ and assume $\exists C>0$ s.t. $$\norm{x}_2\leq C\norm{x}_1,\,\forall x\in X\quad(1)$$
	If $X$ is complete, with respect to both $\norm{\cdot}_1$ and $\norm{\cdot}_2$ then consider $A=id:(X,\norm{\cdot}_1)\to(X,\norm{\cdot}_2)$ is open by Theorem (indeed thm applies $b/c$ $A$ is bounded by $(1)$. Since $A$ is bijective, ii) gives that $A^{-1}=id:(X,\norm{\cdot}_2)\to(X,\norm{\cdot}_1)$ is bounded, i.e.
	$$
		\exists C': \norm{A^{-1}}_1=\norm{x}_1\leq C'\norm{x}_2
	$$
	so $\norm{\cdot}_1$ and $\norm{\cdot}_2$ are actually equivalent.
\end{example}

\begin{example}[Completeness of $Y$]\nl
	Consider $X=C(=C^0[0,1])$ with $\norm{\cdot}_1=\norm{\cdot}_\infty$, $\norm{\cdot}_2=\norm{\cdot}_{L^1}$. Then $A=id:(X,\norm{\cdot}_1)\to(X,\norm{\cdot}_2)$ is continuous:
	$$
		\norm{Af}_2
		=\norm{f}_2
		=\int_0^1|f(t)|dt
		\leq\norm{f}_\infty
		=\norm{f}_1
	$$
	but not open. Else by 1),  $\norm{\cdot}_1$ and $\norm{\cdot}_2$ would be equivalent. However consider counter example:
	\begin{equation}\nonumber
		f_n(x)=\left\{
		\begin{aligned}
			 & {2n^2x}     & x\in[0,\frac{1}{2n}]           \\
			 & {-2n^2x+2n} & x\in(\frac{1}{2n},\frac{1}{n}] \\
			 & 0           & x\in(\frac{1}{n},1]            \\
		\end{aligned}
		\right.\quad\text{satisfy}\quad\norm{f_n}_2=1,\norm{f_n}_1=n\to\infty
	\end{equation}
	This shows $Y$ needs to be complete in theorem.
\end{example}
\begin{example}[Completeness of $X$]\nl
	This example shows completness of $X$ is also required.
	Take
	$$
		X=Y=\{(x_n)\in\ell^\infty:\exists N:x_n=0\,\forall m\geq N\}\subset\ell^\infty
	$$
	with norm $\norm{\cdot}_X=\norm{\cdot}_Y=\norm{\cdot}_\infty$. This is a linear normed space. It's not complete (Exercise: show directly $\overline{X}=c_0$). Another way:
	Define \func{A}{X}{X},
	$$
		Ax=(x_1,\frac{x_2}{2},\frac{x_3}{3}\underbrace{......}_{0\,eventually})\quad if \,\,x=(x_1,x_2......)
	$$
	Then $A$ is linear, bijective with
	$$
		A^{-1}:X\to ,\,\,\,A^{-1}x=(x_1,2x_2,3x_3\underbrace{......}_{0\,eventually})
	$$
	and $A$ is bounded.
	$$
		\norm{Ax}_\infty=\sup_{n\geq1}\frac{|x_n|}{n}\leq\sup_{n\geq1}|x_n|=\norm{x}_\infty
	$$
	so $\norm{A}\leq1$. But $A^{-1}$ is unbounded.
	Pick $x^{(n)}=(\overbrace{1,1,1,1}^{n},0,......)$ then $\norm{x^{(n)}}_\infty=1$ but $\norm{A^{-1}x^{(n)}}=n$. Hence $A^{-1}\not\in\mathcal{L}(X)$ and $X$ cannot be complete, else by theorem i), $A^{-1}$ would be bounded.
\end{example}



\subsection{Closed Graph Theorem}

\begin{definition}[Graph,closed graph and closed operator]\label{Closed Operator}\nl
	Let $X$, $Y$ be normed spaces. Linear operator $T$ defined on $D(T)\subset X$, linear subspace of $X$. Graph of $T$ is $\Gamma (T)=\{(x,Tx):x\in D(T)\}$. If $\Gamma (T)$ is closed in $(X\times Y,\norm{\cdot}_{X\times Y}$, where $\norm{\cdot}_{X\times Y}$ can be set to $max(\norm{\cdot}_{X},\norm{\cdot}_{Y)}$ or $\norm{\cdot}_{X}+\norm{\cdot}_{Y}$, we say $\Gamma T$ is closed, and $T$ is called closed operator.
\end{definition}





\begin{theorem}[Closed Graph Theorem]\label{CGT}\nl
	Let $X$, $Y$ be Banach. $T$ an linear operator \func {T}{X}{Y}. The following two statements are equivalent:
	\begin{itemize}
		\item $T$ is closed.
		\item $T$ is continuous.
	\end{itemize}
	\begin{pf}{$bounded\implies closed$}{}
		We should check that limit of a convergent sequence sequence in $\Gamma(T)$ is inside $\Gamma(T)$.
		By
		\sref{Completeness of product of Banach spaces} 
		$X\times Y$ is complete.
		Since $T$ is continuous, consider $\sequ{a_n}\subset X\times Y$ where $a_n=(x_n,Tx_n)$ with $x_n\to x\in X$. By continuity of $T$ and completeness of $Y$ we have that $Ax_n\to y=Ax\in T(X)\subset Y$. Then we have that
		$$\norm{(x,Tx)-(x_n,Tx_n)}_{X \times Y}=\underbrace{\norm{x-x_n}_{X}}_{\to 0}+\underbrace{\norm{y-Tx_n}}_{\to 0}\to0\quad \text{as  }n\to0$$
		This means that $a_n\to(x,Tx)\in \Gamma(T)$. Thus the graph is closed, hence $T$ is closed.
	\end{pf}
	\begin{pf}{$closed\implies bounded$}{}
		Idea of this part of the proof is recover $T$ by the partly inverting the map from $(X,Y)$ to $\Gamma(T)$.
		Consider two linear maps:
		$$
			\Pi_A():\Gamma(T)\to X,\,\Pi_A((x,Tx))=x\qquad \Pi_B():\Gamma(T)\to Y,\,\Pi_A((x,Tx))=Ax
		$$
		We can use sequential continuity to check that both maps are continuous. Moreover, $\Pi_A$ is  surjective and injective from its definition, thus bijective. By open mapping theorem, we have that $\exists {\Pi_A}^{-1}\in\mathcal{L}(X,\Gamma(T))$, which means that ${\Pi_A}^{-1}$ is continuous. Thus we have
		$$
			T=\underbrace{{\Pi_A}^{-1}}_{conti.}\circ\underbrace{{\Pi_B}}_{conti.}
		$$
		Is also countinuous.
	\end{pf}
\end{theorem}
\begin{corollary}[Continuous Inverse]\nl
	$X$, $Y$ Banach, $A:(DA)\subset X\to Y$ linear, closed and bijective. Then $\exists B=A^{-1}\in\mathcal{L}(Y,X)$ with $AB=id_Y$ and $BA=id_{D(A)}$.
	Proof is left as an exercise. Hint: similar to CGT, consider $\Pi_Y:\Gamma_A\to Y$, $B\stackrel{def.}{=}\Pi_X\circ \Pi_Y^{-1}$
	
\end{corollary}

\begin{example}[???]\nl
	A is surjective: for $g\in C$ define $f(t)=\int_0^t g(s) ds$. Then by FTC, $Af=g$.\\
	A is not injective: $Af=A\tilde{f}\implies f=\tilde{f}+c,c\in\real$. 
	Let $D(A)\defeq C_0^1[0,1]=\{f\in C^1[0,1]:f(0)=0\}$
	Then $A:D(A)\to C$ is bijective and has continuous inverse $B=A^{-1}$ by corollary. In fact, $Bf(t)=\int_0^tf(s)ds$ with $Bf\in D(A)$.
\end{example}

\end{comment}


\section{Open mapping theorem}
\begin{definition}[Open Ball]\nl
An open ball in normed linear space $X$ with radius $r>0$ centered at $x\in X$ is
$$
B_X(x,r)=\{y\in X:\norm{y-x}_X<r\}
$$
Also, when $x=0$ we write 
$$
B_X(0,r)\equiv B_x(r)
$$
\end{definition}

\begin{definition}[Open map]\label{open map}\nl
	Let $X$, $Y$ be linear spaces. \func{A}{X}{Y} is {\underline{open}} if $A(U)\subset Y $ is open.
\end{definition}
\begin{remark}\hfill

\begin{itemize}
    \item $A$ being continuous means $A^{-1}(V)\subset{X}$ open $\forall V\subset Y$ open.
    \item $A$ being continuous need not be open. e.g. $Ax\stackrel{def}{=}0\in Y$
\end{itemize}
\end{remark}

\begin{theorem}[Open Mapping Theorem]\label{OMT}\nl
	Let $X,Y$ be Banach, $A\subset\mathcal{L}(X,Y)$. Then:
	\begin{itemize}
	    \item[i)] if $A$ is surjective, $A$ is open.
	    \item[ii)] if $A$ is bijective, then $A^{-1}\in \mathcal{L}(X,Y)$. (Inverse operator theorem)
	\end{itemize}
\end{theorem}

\begin{remark}\nl
ii) important in application. If $A\in\blf{X}{Y}$ is bijective then \func{A^{-1}}{X}{Y} liner is easy (why?). The point is $A^{-1}$ is also bounded, or equivalently continuous.
\end{remark}

The main step of the proof is the following:
\begin{lemma}[$A$ as in i)]\nl
$\exists r>0$ s.t. $B_Y(r)\subset \overline{A(B_x(1))}$
\begin{pf}{}{}\rm
Since $A$ is suerjective 
$$
Y=\bigcup_{k=1}^\infty A(B_X(k))
$$
Since $Y$ is complete, by  Baire Category theorem, $\exists k_0$ s.t. 
$$ int(\overline{A(B_X(1))})\neq\emptyset$$
So by surjectivity of $A$, one can find $y_0=Ax_0\in Y$, $r_0>0$ s.t. 
$$ \underbrace{B_Y(y_0,r_0)}_{=Ax_0+B_Y(r_0)}\subset \overline{A(B_X(k_0))}$$
By linearity of $A$,
\begin{equation}\nonumber
    \begin{split}
        B_Y(r_0)&\subset\overline{A(B_X(k_0))}-Ax_0 \\ &=\overline{A(B_X(k_0)-x_0)}\\
        &\subset \overline{A(B_X(k_0+M))}   \\
        &=(k_0+M)\overline{A(B_X(1))}\\
    \end{split}
\end{equation}
Where $M\stackrel{def}{=}\norm{x_0}_X$. So pick $r=\frac{r_0}{k_0+M}$.
\end{pf}

\end{lemma}
Proof of theorem:
\begin{pf}{}{}
i) Pick $r$ as in Lemma.\\
Claim: $B_Y(r/2)\subset A(B_X(1)))$.\\
If claim holds, then for $U\subset X$ open, pick $x_0\in U$, $s>0$ small so that $B_X(x_0,s)\subset U$. Letting $y_0\stackrel{def}{=}Ax_0$, get 
$$
B_Y(y_0,rs/2)=y_0+sB_Y(r/2)\stackrel{claim}{\subset}Ax_0+sA(B_X(1)\stackrel{lin.}{=}A(B_X(x_0,s))\subset A(U)
$$
which proves i). To see i) $\implies$ ii), it's enough to show that $B=A^{-1}:Y\to X$ is continuous; but for any $U\subset X$ open, $B^{-1}(U)=(A^{-1})^{-1}(U)=A(U)$
which is open by i). $\square$
\end{pf}
Proof of claim:
\begin{pf}{}{}
Fix $y\in B_Y(r/2)$. Need to show: $y=Ax$ for some $x\in X$ with $\norm{x}_X<1.$\\
We construct a sequence $(x_k)\subset X$ with 
$$
\sum_{k=1}^\infty \norm{x_k}_X<1\,\,\text{and}\,\,\sum_{k=1}^\infty Ax_k\stackrel{wrt\norm{\cdot}_Y}{\longrightarrow{}}y,\,n\to\infty
$$
By completeness of $X$, $\sum_{k=1}^\infty x_k\stackrel{def.}{=}x$ exists, $x\in B_X(1)$ and by continuity of $A$,
$$Ax=\sum_{k=1}^\infty Ax_k=y$$
By lemma above, 
$$\forall s>0, B_Y(sr)\subset \overline{A(B_X(s))}\,\,(*)$$
$s=1/2$. Pick $x_1\in B_X(1/2)$ s.t. $\norm{Ax_1-y}<r/2$. Now set $y_1=y-Ax(\in B_X(r/2)$. Iterate. Assume that for some $\geq 1$ have $x_1,......x_k,y_1,......y_k$ s.t.
$$
\forall 1\leq\tilde{k}\leq k:\,\,\norm{\tilde{x_k}}_X<2^{-k},\,y_{\tilde{k}}=y_{\tilde{k}-1}-Ax_{\tilde{k}}\in B_Y(2^{-\tilde{k}}r
$$
Then using $(*)$ with $s=2^{-(k+1)}$ find $x_{k+1}\in B_X(2^{-(k+1)})$ such that
$$
y_{k+1}\stackrel{def}{=}y_k-Ax_{k+1}\in B_Y(2^{-(k+1)}r
$$
This yields $\sum{k=1}^{\infty}\norm{x_k}_X<1$ and 
$$
y-\sum_{k=1}^n Ax_k=y_1-\sum_{k=2}^n Ax_k=...=y_n\to0\,(n\to \infty)\quad\square
$$
\end{pf}

\begin{example}[Equivalence of Norm]\label{Equinorm}\nl
Let $X=Y$, with norms $\norm{\cdot}_1$ and $\norm{\cdot}_2$ and assume $\exists C>0$ s.t. $$\norm{x}_2\leq C\norm{x}_1,\,\forall x\in X\quad(1)$$
If $X$ is complete, with respect to both $\norm{\cdot}_1$ and $\norm{\cdot}_2$ then consider $A=id:(X,\norm{\cdot}_1)\to(X,\norm{\cdot}_2)$ is open by Theorem (indeed thm applies $b/c$ $A$ is bounded by $(1)$. Since $A$ is bijective, ii) gives that $A^{-1}=id:(X,\norm{\cdot}_2)\to(X,\norm{\cdot}_1)$ is bounded, i.e.
$$
\exists C': \norm{A^{-1}}_1=\norm{x}_1\leq C'\norm{x}_2
$$
so $\norm{\cdot}_1$ and $\norm{\cdot}_2$ are actually equivalent.
\end{example}

\begin{example}[Completeness of $Y$]\nl
Consider $X=C(=C^0[0,1])$ with $\norm{\cdot}_1=\norm{\cdot}_\infty$, $\norm{\cdot}_2=\norm{\cdot}_{L^1}$. Then $A=id:(X,\norm{\cdot}_1)\to(X,\norm{\cdot}_2)$ is continuous:
$$
\norm{Af}_2
=\norm{f}_2
=\int_0^1|f(t)|dt
\leq\norm{f}_\infty
=\norm{f}_1
$$
but not open. Else by 1),  $\norm{\cdot}_1$ and $\norm{\cdot}_2$ would be equivalent. However, consider counter-example:
\begin{equation}\nonumber
f_n(x)=\left\{
\begin{split}
    &{2n^2x} &x\in[0,\frac{1}{2n}]\\
    &{-2n^2x+2n} &x\in(\frac{1}{2n},\frac{1}{n}]\\
    &0 &x\in(\frac{1}{n},1]\\
\end{split}
\right.\quad\text{satisfy}\quad\norm{f_n}_2=1,\norm{f_n}_1=n\to\infty
\end{equation}
This shows $Y$ needs to be complete in theorem.
\end{example}
\begin{example}[Completeness of $X$]\nl
This example shows completeness of $X$ is also required.
Take 
$$
X=Y=\{(x_n)\in\ell^\infty:\exists N:x_n=0\,\forall m\geq N\}\subset\ell^\infty
$$
with norm $\norm{\cdot}_X=\norm{\cdot}_Y=\norm{\cdot}_\infty$. This is a linear normed space. It's not complete (Exercise: show directly $\overline{X}=c_0$). Another way:
Define \func{A}{X}{X}, 
$$
Ax=(x_1,\frac{x_2}{2},\frac{x_3}{3}\underbrace{......}_{0\,eventually})\quad if \,\,x=(x_1,x_2......)
$$
Then $A$ is linear, bijective with 
$$
A^{-1}:X\to ,\,\,\,A^{-1}x=(x_1,2x_2,3x_3\underbrace{......}_{0\,eventually})
$$
and $A$ is bounded. 
$$
\norm{Ax}_\infty=\sup_{n\geq1}\frac{|x_n|}{n}\leq\sup_{n\geq1}|x_n|=\norm{x}_\infty
$$
so $\norm{A}\leq1$. But $A^{-1}$ is unbounded. 
Pick $x^{(n)}=(\overbrace{1,1,1,1}^{n},0,......)$ then $\norm{x^{(n)}}_\infty=1$ but $\norm{A^{-1}x^{(n)}}=n$. Hence $A^{-1}\not\in\mathcal{L}(X)$ and $X$ cannot be complete, else by theorem i), $A^{-1}$ would be bounded.
\end{example}

\section{Closed Graph Theorem}
Consider $X$, $Y$ normed spaces. Often an operator $A$ not defined on all of $A$ but on a "domain" $D(A)$. So we assume that 
$D(A)\subset X$ is a linear subspace on which $A:D(A)(\subset X)\to Y$, linear is defined.
\begin{example}{Running Example}\nl
$Y=X=C=C^0[0,1]$ with $\norm{\cdot}_X=\norm{\cdot}_\infty$ and $A=\frac{d}{dt}$, with $D(A)\stackrel{eg}{=}C^1[0,1]\subset X$ or subspaces thereof. Prime example of  (\underline{unbounded}) operator with dense domain $D(A)$: indeed $C^1[0,1]=C$ using e.g. Weierstrass Approximation Theorem (Polynomials are already $\norm{\cdot}_\infty$-dense in C)
\end{example}
\begin{definition}[Graph]\nl
Let $X$, $Y$ be normed space, $A:D(A)(\subset X)\to Y$ . Graph of $A$ (really  of $(A,D(A))$) is the linear (!) space 
$$
\Gamma_A=\{(x,Ax):x\in D(A)\}\subset X\times Y
$$
We endowed $X\times Y$ with the norm $\norm{(x,y)}_{X\times Y}=\norm{x}_X+\norm{y}_Y$, for all $x\in X$, $y\in Y$.
    
\end{definition}
\begin{definition}[Closed Operator]\label{Closed Operator}
$A$ is called \underline{closed} if $\Gamma_A$ is closed in  $(X\times Y,\norm{\cdot}_{X\times Y})$
\end{definition}

\begin{example}
Let $A\in\mathcal{L}(X,Y)$ with $D(A)=X$. Then $A$ is closed.
\begin{pf}{}{}
	Let $(x_k,y_k)_k\subset \Gamma_A$  with $\norm{(x_k,y_k)-(x,y)}_{X\times Y}\xrightarrow{k\to\infty}0$ for some $(x,y)\in X\times Y$\\
	NTS: $(x,y)\in \Gamma_A$ i.e. $y=Ax$. 
	Know $ y_k=Ax_k$ and $\norm{x_k-x}_X\xrightarrow{k\to\infty}0$, $\norm{Ax-y}_Y\xrightarrow{k\to\infty}0$
	But $\forall k\geq 1$
	$$\norm{y-Ax}_Y\leq\norm{y-Ax}_Y+\norm{Ax_k-ax}_Y
	\leq \norm{y-Ax}_Y\norm{A}\norm{x_k-x}_X$$
	Thus 
	$$\lim_{k\to\infty}\norm{y-Ax}_Y\leq\lim_{k\to\infty}\norm{y-Ax}_Y\norm{A}\norm{x_k-x}_X=0$$
\end{pf}
\end{example}

\begin{theorem}[Closed Graph]\label{CGT}\nl
Let $X$, $Y$ be Banach $A:X\to Y$ linear. The following are equivalent:
\begin{itemize}
    \item [i)] $A\in\mathcal{L}(X,Y)$
    \item [ii)] $A$ is closed
\end{itemize} 
\begin{pf}{}{}
    i) $\implies$ ii): see example\\
    ii) $\implies$ i): If $X$, $Y$ complete, then so is $(X\times Y,\norm{\cdot}_{X\times Y})$ (exercise). A closed means $\Gamma_A$ is closed in $(X\times Y,\norm{\cdot}_{X\times Y})$, so $(\Gamma_A,\norm{\cdot}_{X\times Y})$ is complete. Consider:
    \begin{equation}
        \begin{aligned}
            \Pi_X:\,\,\Gamma_A&\to X \qquad\qquad& \Pi_Y:\Gamma_A&\to Y\\
        (x,Ax)&\mapsto x  & (x,Ax)&\mapsto Ax\\
        \end{aligned}    
    \end{equation}
$\Pi_X$, $\Pi_Y$ are continuous with $\norm{\Pi_X},\norm{\Pi_Y}\leq 1$, $\Pi_X$ is injective, and surjective. By OMT, ii), ${\Pi_X}^{-1}\in\mathcal{L}{(X,\Gamma_A)}$ and so
$$
A=\Pi_Y\circ \Pi_X^{-1}\in\mathcal{L}(X,Y)
$$
\end{pf}
\end{theorem}

\begin{remark}
	ii) is simpler than i), but equivalent.\\
	i) says A is continuous, i.e. if $(x_n)\subset X$, $x\in X$
	$$\norm{x_n-x}_X\rightarrow{n\to\infty}0\implies\norm{Ax_n-Ax}_Y\rightarrow{n\to\infty}0$$
	This contains two things to check: $(Ax_n)$ converges and limit is $Ax$.\\
	ii) says $A$ is closed, i.e.
	\begin{equation}
		\left\{
		\begin{aligned}
			&\norm{x_n-x}_X\rightarrow{n\to\infty}0\\
			&\norm{Ax_n-y}_Y\rightarrow{n\to\infty}0\\
		\end{aligned}
		\right.\implies
		Ax=y
	\end{equation}
	Which is only one condition to check.
\end{remark}

\begin{example}[running example continues]
	$(D(A),\norm{\cdot}_\infty)$ with $D(A)=C^1[0,1]$ is NOT Banach, and $A:D(A)\to C$ is an example of an operator which is:\\
	claim: \\
	i) closed, but\\
	ii) not continuous\\
	For ii), take $f_n(t)=t^n\in D(A)$, $Af_n=nf_{n-1}$ so $\norm{f_n}_\infty=1$, $\norm{Af_n}\infty=n\norm{f_{n-1}}\infty=n$. So 
	$$ \sup_{f\in D(A),\norm{f}\infty\leq1}\norm{Af}_\infty=\infty$$
	For i), if $(f_n,f_n')\to(f,g)$ in $(D(A)\times C)$ then $\norm{f-f_n}_\infty\to 0$, $\norm{f_n'-g}_\infty\to0$ but
	$$
	\forall t\in(0,1],\,\underbrace{f_n(t)}_{\rightarrow{n\to\infty}f(t)}
	=\underbrace{\int_0^t f'_n(x) dx}_{\rightarrow{DCT}\int_0^t g(x) dx}
	+f_n(0)
	$$
	so $f'=g$ by fundamental theorem of calculus(FTC), i.e. $(f,g)=(f,f')\in\Gamma_A$.
\end{example}


\begin{corollary}[Continuous Inverse]\nl
	$X$, $Y$ Banach, $A:(DA)\subset X\to Y$ linear, closed and bijective. Then $\exists B=A^{-1}\in\mathcal{L}(Y,X)$ with $AB=id_Y$ and $BA=id_{D(A)}$.
	Proof is left as an exercise. Hint: similar to CGT, consider $\Pi_Y:\Gamma_A\to Y$, $B\stackrel{def.}{=}\Pi_X\circ \Pi_Y^{-1}$
	
\end{corollary}

\begin{example}[???]\nl
	A is surjective: for $g\in C$ define $f(t)=\int_0^t g(s) ds$. Then by FTC, $Af=g$.\\
	A is not injective: $Af=A\tilde{f}\implies f=\tilde{f}+c,c\in\real$. 
	Let $D(A)\stackrel{def.}{=} C_0^1[0,1]=\{f\in C^1[0,1]:f(0)=0\}$
	Then $A:D(A)\to C$ is bijective and has continuous inverse $B=A^{-1}$ by corollary. In fact, $Bf(t)=\int_0^tf(s)ds$ with $Bf\in D(A)$.
\end{example}



\end{document}


\newpage
\section{Appendix}

\subsection{Young's Inequality}\label{Young's Inequality}
\placeholder

\subsection{Minkowski's inequality}\label{Minkowski-holder}
\placeholder


\subsection{Hölder's Inequality}\label{Hölder's inequality}
\begin{theorem}[Hölder's inequality,]\rm\nextline
Let $p,q\in[1,\infty]$,	satisfying:
$$
\frac{1}{p}+\frac{1}{q}=1
$$
Then if $x\in\ell^p$ and $y\in\ell^q$ then 
$$
\norm{x}_p\norm{y}_q\geq\norm{xy}_1
\iff
\left(\sum_{n=1}^\infty{|x_n|}^p\right)^{1/p}
\left(\sum_{n=1}^\infty{|y_n|}^q\right)^{1/q}\geq\sum_{n=1}^\infty|x_ny_n|
$$
Then if $g\in L^p(S)$ and $g\in L^q(S)$ then 
$$
\norm{f}_p\norm{g}_q\geq\norm{fg}_1
\iff
\left(\int_S{|f|}^pdx\right)^{1/p}
\left(\int_S{|g|}^qdx\right)^{1/q}
\geq\int_S|fg|dx
$$
\end{theorem}
\begin{remark}[Short version]\rm\nextline
Here's the key information of the above theorem:
$$
\frac{1}{p}+\frac{1}{q}=1\implies \norm{x}_p\norm{y}_q\geq\norm{xy}_1
$$
\end{remark}
\begin{remark}
When supremum occurs, which is  one of $p$ and $q$ becomes infinity and another becomes 1, the inequality still holds. One should pay attention that for $L^p$ spaces the supremum norm is actually essential supremum.
\end{remark}

\begin{remark}[Proof of Hölder's inequality]\rm\nextline
The proof uses young's inequality.\\
\placeholder
\end{remark}

\subsection{Jensen's Inequality (convex function)}\label{Jensen's inquality}
Let $\mathbb{I}\subset \real$. A function \func{f}{\mathrm{I}}{\real} is {\bf convex} if the following holds for all $t\in[0,1]$:
$$
	t\cdot f(x)+(1-t)\cdot f(y)\geq f(tx+(1-t)y),\,\,\forall x,\,y\in \real\quad\quad\text{(Jensen)}
$$
Similarly,  a function \func{f}{\mathrm{I}}{\real} is {\bf concave} if the following holds for all $t\in[0,1]$:
$$
	t\cdot f(x)+(1-t)\cdot f(y)\leq f(tx+(1-t)y),\,\,\forall x,\,y\in \real
$$
For convex function, the inequality may be rewritten as
$$
	\frac{1}{t}(f(y+t
	=(x-y))-f(y))\leq f(x)-f(y)
$$
For concave function with $f(0)\geq 0$ we have {\bf sub-additivity},which is for $t\in[0,1]$,
$$
	f(tx)=f(tx+(1-t)0)\geq tf(x)+(1-t)f(0)\geq tf(x)
$$
and thus,for $a,b\in\real^+$
$$
	f(a)+f(b)=f(\frac{a}{a+b}(a+b))+f(\frac{b}{a+b}(a+b))\geq \frac{a}{a+b}f(a+b)+\frac{b}{a+b}f(a+b)=f(a+b)
$$

\begin{remark}[Generalised Jensen's Inequality]\rm\nextline
	Jensen's inequality can be generalized to a sequence variables with weight. Consider a convex function evaluated at $x_1,x_2...$, $f(x_1),f(x_2)...f(x_n)$, with weight $w_1,w_2...$ with $\sum_{j\in\natu}w_j=1$, where
	$$
		\sum_{j\in\natu}w_j f(x_j)<\infty \quad\text{and}\quad
		\sum_{j\in\natu} f(w_jx_j)<\infty
	$$
	Then,
	$$
		\sum_{j\in\natu} f(w_jx_j)\leq\sum_{j\in\natu}w_j f(x_j)
	$$
\end{remark}

\begin{remark}[$f''>0$]\rm\nextline
	For single variable twice differentiable function, second derivative being non=negative implies convexity.
\end{remark}


\subsection{Completion of metric space}\label{completion of metric space}


\begin{theorem}[Completion of metric space]\rm\nextline
	Every metric space has a completion.

\end{theorem}
Idea of proof: First construct a space of Cauchy sequence and define a metric on this space, then show that the original space can be embedded to the space of Cauchy sequence as a dense subset by an isometry. Proof given step by step.\\
\begin{lemma}[Step I:Space of Cauchy sequence]\rm\nextline
	Let $(X,d)$ be a metric space. Let $C[X]$ denote set of all Cauchy sequence in $X$. Define equivalence relation $\sim$ on $C[X]$:
	$x\sim y\iff \lim_{n\to\infty}d(x_n,y_n)=0$. Define set $X^*=\{[(x_n)],(x_n)\in C[X]\}$ and metric on this set $d^*((x_n),(y_n))=\lim_{n\to\infty}d(x_n,y_n)$. One can check that $d^*$ is indeed a well-defined metric on $X^*$
\end{lemma}


\subsection{\texorpdfstring{$L^p$}. space and \texorpdfstring{${l}^p$}. space}
$L^p$ space and $\ell^p$ space are very important for both Lebesgue measure and also functional analysis. We shall fist introduce $\ell^p$ space by introducing its $p-norm$. Here the definition is valid on both $\real$ and $\comp$.

\begin{definition}[p-norm,discrete]\rm\nextline
	Let $p\in\real$ with $1\leq p\leq\infty$. \\
	For $p<\infty$,we define p-norm of sequence $\sequ{x_n}$ to be:
	$$
		\norm{\sequ{x_n}}_\infty\equiv\left(\sum_{s=1}^\infty|x_n|^p\right)^{1/p}
	$$
	And for when "$p=\infty$" the p-norm becomes $\infty$-norm, defined as
	$$
		\norm{\sequ{x_n}}_p\equiv\sup_{n\in\natu}|x_n|
	$$
	And $\ell^p$ space is defined to be the set of all sequences with finite p-norm:
	$$
		\ell^p\equiv\left\{\sequ{x_n}:\norm{\sequ{x_n}}_p<\infty\right\}
	$$
\end{definition}

\begin{definition}[p-norm,continuous]\rm\nextline
	Let $p\in\real$ with $1\leq p\leq\infty$. Again,$\infty$ gives supremum.(why?)\\
	Consider measurable functions \func{f}{[a,b]}{\real}.\\
	For $p<\infty$,we define p-norm of function $f$ to be:
	$$
		\norm{f}_p\equiv\left(\int_a^b{|f(x)|}^p dx \right)^{1/p}
	$$
	And for when "$p=\infty$" the p-norm again becomes $\infty$-norm, defined as
	$$
		\norm{f}_\infty\equiv\sup_{n\in[a,b}|f(x)|
	$$
	And $L^p$ space is defined to be the set of measurable all functions with finite p-norm from $[a,b]$ to $\real$ or $\comp$:
	$$
		L^p\equiv\left\{f:\norm{f}_p<\infty\right\}
	$$
\end{definition}

\begin{remark}\rm\nextline
	There are very typical $\ell^p$ spaces, like
	\begin{itemize}
		\item $\ell^1$, space of absolutely convergent sequences
		\item $\ell^2$, gives a set of square-summable sequences, forms a Hilbert space
		\item $\ell^\infty$, set of all bounded sequences.
	\end{itemize}

\end{remark}
\end{document}

