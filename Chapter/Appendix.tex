\newpage
\section{Appendix}
\subsection{Minkowski's inequality}\label{Minkowski-holder}

\subsection{Hölder's Inequality}\label{Hölder's inequality}
\begin{theorem}[Hölder's inequality,]\rm\nextline
Let $p,q\in[1,\infty]$,	satisfying:
$$
\frac{1}{p}+\frac{1}{q}=1
$$
Then if $x\in\ell^p$ and $y\in\ell^q$ then 
$$
\norm{x}_p\norm{y}_q\geq\norm{xy}_1
\iff
\left(\sum_{n=1}^\infty{|x_n|}^p\right)^{1/p}
\left(\sum_{n=1}^\infty{|y_n|}^q\right)^{1/q}\geq\sum_{n=1}^\infty|x_ny_n|
$$
Then if $g\in L^p(S)$ and $g\in L^q(S)$ then 
$$
\norm{f}_p\norm{g}_q\geq\norm{fg}_1
\iff
\left(\int_S{|f|}^pdx\right)^{1/p}
\left(\int_S{|g|}^qdx\right)^{1/q}
\geq\int_S|fg|dx
$$
\end{theorem}
\begin{remark}[Short version]\rm\nextline
Here's the key information of the above theorem:
$$
\frac{1}{p}+\frac{1}{q}=1\implies \norm{x}_p\norm{y}_q\geq\norm{xy}_1
$$
\end{remark}
\begin{remark}
When supremum occurs, which is  one of $p$ and $q$ becomes infinity and another becomes 1, the inequality still holds. One should pay attention that for $L^p$ spaces the supremum norm is actually essential supremum.
\end{remark}


\subsection{Jensen's Inequality (convex function)}\label{Jensen's inquality}
Let $\mathbb{I}\subset \real$. A function \func{f}{\mathrm{I}}{\real} is {\bf convex} if the following holds for all $t\in[0,1]$:
$$
	t\cdot f(x)+(1-t)\cdot f(y)\geq f(tx+(1-t)y),\,\,\forall x,\,y\in \real\quad\quad\text{(Jensen)}
$$
Similarly,  a function \func{f}{\mathrm{I}}{\real} is {\bf concave} if the following holds for all $t\in[0,1]$:
$$
	t\cdot f(x)+(1-t)\cdot f(y)\leq f(tx+(1-t)y),\,\,\forall x,\,y\in \real
$$
For convex function, the inequality may be rewritten as
$$
	\frac{1}{t}(f(y+t
	=(x-y))-f(y))\leq f(x)-f(y)
$$
For concave function with $f(0)\geq 0$ we have {\bf sub-additivity},which is for $t\in[0,1]$,
$$
	f(tx)=f(tx+(1-t)0)\geq tf(x)+(1-t)f(0)\geq tf(x)
$$
and thus,for $a,b\in\real^+$
$$
	f(a)+f(b)=f(\frac{a}{a+b}(a+b))+f(\frac{b}{a+b}(a+b))\geq \frac{a}{a+b}f(a+b)+\frac{b}{a+b}f(a+b)=f(a+b)
$$

\begin{remark}[Generalised Jensen's Inequality]\rm\nextline
	Jensen's inequality can be generalized to a sequence variables with weight. Consider a convex function evaluated at $x_1,x_2...$, $f(x_1),f(x_2)...f(x_n)$, with weight $w_1,w_2...$ with $\sum_{j\in\natu}w_j=1$, where
	$$
		\sum_{j\in\natu}w_j f(x_j)<\infty \quad\text{and}\quad
		\sum_{j\in\natu} f(w_jx_j)<\infty
	$$
	Then,
	$$
		\sum_{j\in\natu} f(w_jx_j)\leq\sum_{j\in\natu}w_j f(x_j)
	$$
\end{remark}

\begin{remark}[$f''>0$]\rm\nextline
	For single variable twice differentiable function, second derivative being non=negative implies convexity.
\end{remark}


\subsection{Completion of metric space}\label{completion of metric space}


\begin{theorem}[Completion of metric space]\rm\nextline
	Every metric space has a completion.

\end{theorem}
Idea of proof: First construct a space of Cauchy sequence and define a metric on this space, then show that the original space can be embedded to the space of Cauchy sequence as a dense subset by an isometry. Proof given step by step.\\
\begin{lemma}[Step I:Space of Cauchy sequence]\rm\nextline
	Let $(X,d)$ be a metric space. Let $C[X]$ denote set of all Cauchy sequence in $X$. Define equivalence relation $\sim$ on $C[X]$:
	$x\sim y\iff \lim_{n\to\infty}d(x_n,y_n)=0$. Define set $X^*=\{[(x_n)],(x_n)\in C[X]\}$ and metric on this set $d^*((x_n),(y_n))=\lim_{n\to\infty}d(x_n,y_n)$. One can check that $d^*$ is indeed a well-defined metric on $X^*$
\end{lemma}


\subsection{\texorpdfstring{$L^p$}. space and \texorpdfstring{${l}^p$}. space}
$L^p$ space and $\ell^p$ space are very important for both Lebesgue measure and also functional analysis. We shall fist introduce $\ell^p$ space by introducing its $p-norm$. Here the definition is valid on both $\real$ and $\comp$.

\begin{definition}[p-norm,discrete]\rm\nextline
	Let $p\in\real$ with $1\leq p\leq\infty$. \\
	For $p<\infty$,we define p-norm of sequence $\sequ{x_n}$ to be:
	$$
		\norm{\sequ{x_n}}_\infty\equiv\left(\sum_{s=1}^\infty|x_n|^p\right)^{1/p}
	$$
	And for when "$p=\infty$" the p-norm becomes $\infty$-norm, defined as
	$$
		\norm{\sequ{x_n}}_p\equiv\sup_{n\in\natu}|x_n|
	$$
	And $\ell^p$ space is defined to be the set of all sequences with finite p-norm:
	$$
		\ell^p\equiv\left\{\sequ{x_n}:\norm{\sequ{x_n}}_p<\infty\right\}
	$$
\end{definition}

\begin{definition}[p-norm,continuous]\rm\nextline
	Let $p\in\real$ with $1\leq p\leq\infty$. Again,$\infty$ gives supremum.(why?)\\
	Consider measurable functions \func{f}{[a,b]}{\real}.\\
	For $p<\infty$,we define p-norm of function $f$ to be:
	$$
		\norm{f}_p\equiv\left(\int_a^b{|f(x)|}^p dx \right)^{1/p}
	$$
	And for when "$p=\infty$" the p-norm again becomes $\infty$-norm, defined as
	$$
		\norm{f}_\infty\equiv\sup_{n\in[a,b}|f(x)|
	$$
	And $L^p$ space is defined to be the set of measurable all functions with finite p-norm from $[a,b]$ to $\real$ or $\comp$:
	$$
		L^p\equiv\left\{f:\norm{f}_p<\infty\right\}
	$$
\end{definition}

\begin{remark}\rm\nextline
	There are very typical $\ell^p$ spaces, like
	\begin{itemize}
		\item $\ell^1$, space of absolutely convergent sequences
		\item $\ell^2$, gives a set of square-summable sequences, forms a Hilbert space
		\item $\ell^\infty$, set of all bounded sequences.
	\end{itemize}

\end{remark}