\newpage
\section{Open mapping theorem}
\begin{definition}[Open map]\label{open map}\nl
	An open map is one for which the image of every open set is open.
\end{definition}
\begin{theorem}[Open Mapping Theorem]\label{OMT}\nl
	Let $X,Y$ be Banach spaces, $T:X\xrightarrow{}Y$ a continuous surjective linear map. Then $T$ is an open map.
\end{theorem}

\begin{proposition}[Equivalence of norms]\label{Equinorm}
	\placeholder
\end{proposition}
\subsection{Proof of Open Mapping Theorem}
\placeholder
\subsection{Closed Graph Theorem}

\begin{definition}[Graph,closed graph and closed operator]\nl
	Let $X$, $Y$ be normed spaces. Linear operator $T$ defined on $D(T)\subset X$, linear subspace of $X$. Graph of $T$ is $\Gamma (T)=\{(x,Tx):x\in D(T)\}$. If $\Gamma (T)$ is closed in $(X\times Y,\norm{\cdot}_{X\times Y}$, where $\norm{\cdot}_{X\times Y}$ can be set to $max(\norm{\cdot}_{X},\norm{\cdot}_{Y)}$ or $\norm{\cdot}_{X}+\norm{\cdot}_{Y}$, we say $\Gamma T$ is closed, and $T$ is called closed operator.
\end{definition}


\begin{theorem}[Closed Graph Theorem]\label{CGT}\nl
	Let $X$, $Y$ be Banach. $T$ an linear operator \func {T}{X}{Y}. The following two statements are equivalent:
	\begin{itemize}
	    \item $T$ is closed.
	    \item $T$ is continuous.
	\end{itemize}
	\begin{pf}{$bounded\implies closed$}{}
	    We should check that limit of a convergent sequence sequence in $\Gamma(T)$ is inside $\Gamma(T)$.
	    By \hyperref{Completeness of product of Banach spaces}{Completeness of product of Banach spaces}, $X\times Y$ is complete.
	    Since $T$ is continuous, consider $\sequ{a_n}\subset X\times Y$ where $a_n=(x_n,Tx_n)$ with $x_n\to x\in X$. By continuity of $T$ and completeness of $Y$ we have that $Ax_n\to y=Ax\in \T(X)\subset Y$. Then we have that
	    $$\norm{(x,Tx)-(x_n,Tx_n)}_{X \times Y}=\underbrace{\norm{x-x_n}_{X}}_{\to 0}+\underbrace{\norm{y-Tx_n}}_{\to 0}\to0\quad \text{as  }n\to0$$
	    This means that $a_n\to(x,Tx)\in \Gamma(T)$. Thus the graph is closed, hence $T$ is closed.
	\end{pf}
	\begin{pf}{$closed\implies bounded$}{}
	Idea of this part of the proof is recover $T$ by the partly inverting the map from $(X,Y)$ to $\Gamma(T)$.
	Consider two linear maps:
	$$
	\Pi_A():\Gamma(T)\to X,\,\Pi_A((x,Tx))=x\qquad \Pi_B():\Gamma(T)\to Y,\,\Pi_A((x,Tx))=Ax
	$$
	We can use sequential continuity to check that both maps are continuous. Moreover, $\Pi_A$ is  surjective and injective from its definition, thus bijective. By open mapping theorem, we have that $\exists {\Pi_A}^{-1}\in\mathcal{L}(X,\Gamma(T))$, which means that ${\Pi_A}^{-1}$ is continuous. Thus we have
	$$
	T=\underbrace{{\Pi_A}^{-1}}_{conti.}\circ\underbrace{{\Pi_B}}_{conti.}
	$$
	Is also countinuous.
	\end{pf}
\end{theorem}

\begin{example}[Space not Banach]
\placeholder
\end{example}


\begin{example}[Inverse exists, not continuous]
\placeholder
\end{example}