\newpage
\section{Open mapping theorem}

\begin{comment}
\begin{definition}[Open map]\label{open map}\nl
	An open map is one for which the image of every open set is open.
\end{definition}
\begin{theorem}[Open Mapping Theorem]\label{OMT}\nl
	Let $X,Y$ be Banach spaces, $T:X\xrightarrow{}Y$ a continuous surjective linear map. Then $T$ is an open map.
\end{theorem}
\end{comment}

\begin{definition}[Open Ball]\nl
	An open ball in normed linear space $X$ with radius $r>0$ centered at $x\in X$ is
	$$
		B_X(x,r)=\{y\in X:\norm{y-x}_X<r\}
	$$
	Also, when $x=0$ we write
	$$
		B_X(0,r)\equiv B_x(r)
	$$
\end{definition}

\begin{definition}[Open map]\label{open map}\nl
	Let $X$, $Y$ be linear spaces. \func{A}{X}{Y} is {\underline{open}} if $A(U)\subset Y $ is open.
\end{definition}

\begin{remark}\hfill

	\begin{itemize}
		\item $A$ being continuous means $A^{-1}(V)\subset{X}$ open $\forall V\subset Y$ open.
		\item $A$ being continuous need not be open. e.g. $Ax\stackrel{def}{=}0\in Y$
	\end{itemize}
\end{remark}

\begin{theorem}[Open Mapping Theorem]\label{OMT}\nl
	Let $X,Y$ be Banach, $A\subset\mathcal{L}(X,Y)$. Then:
	\begin{itemize}
		\item[i)] if $A$ is surjective, $A$ is open.
		\item[ii)] if $A$ is bijective, then $A^{-1}\in \mathcal{L}(X,Y)$. (Inverse operator theorem)
	\end{itemize}
\end{theorem}

\begin{remark}\nl
	ii) important in application. If $A\in\mathcal{L}(X,Y)$ is bijective then \func{A^{-1}}{X}{Y} liner is easy (why?). The point is $A^{-1}$ is also bounded, or equivalently continuous.
\end{remark}

The main step of the proof is the following:
\begin{lemma}[$A$ as in i)]\nl
	$\exists r>0$ s.t. $B_Y(r)\subset \overline{A(B_x(1))}$
	\begin{pf}{}{}\rm
		Since $A$ is suerjective
		$$
			Y=\bigcup_{k=1}^\infty A(B_X(k))
		$$
		Since $Y$ is complete, by  Baire Category theorem, $\exists k_0$ s.t.
		$$ int(\overline{A(B_X(1))})\neq\emptyset$$
		So by surjectivity of $A$, one can find $y_0=Ax_0\in Y$, $r_0>0$ s.t.
		$$ \underbrace{B_Y(y_0,r_0)}_{=Ax_0+B_Y(r_0)}\subset \overline{A(B_X(k_0))}$$
		By linearity of $A$,
		\begin{equation}\nonumber
			\begin{split}
				B_Y(r_0)&\subset\overline{A(B_X(k_0))}-Ax_0 \\ &=\overline{A(B_X(k_0)-x_0)}\\
				&\subset \overline{A(B_X(k_0+M))}   \\
				&=(k_0+M)\overline{A(B_X(1))}\\
			\end{split}
		\end{equation}
		Where $M\stackrel{def}{=}\norm{x_0}_X$. So pick $r=\frac{r_0}{k_0+M}$.
	\end{pf}

\end{lemma}
Proof of theorem:
\begin{pf}{}{}
	i) Pick $r$ as in Lemma.\\
	Claim: $B_Y(r/2)\subset A(B_X(1)))$.\\
	If claim holds, then for $U\subset X$ open, pick $x_0\in U$, $s>0$ small so that $B_X(x_0,s)\subset U$. Letting $y_0\stackrel{def}{=}Ax_0$, get
	$$
		B_Y(y_0,rs/2)=y_0+sB_Y(r/2)\stackrel{claim}{\subset}Ax_0+sA(B_X(1)\stackrel{lin.}{=}A(B_X(x_0,s))\subset A(U)
	$$
	which proves i). To see i) $\implies$ ii), it's enough to show that $B=A^{-1}:Y\to X$ is continuous; but for any $U\subset X$ open, $B^{-1}(U)=(A^{-1})^{-1}(U)=A(U)$
	which is open by i). $\square$
\end{pf}
Proof of claim:
\begin{pf}{}{}
	Fix $y\in B_Y(r/2)$. Need to show: $y=Ax$ for some $x\in X$ with $\norm{x}_X<1.$\\
	We construct a sequence $(x_k)\subset X$ with
	$$
		\sum_{k=1}^\infty \norm{x_k}_X<1\,\,\text{and}\,\,\sum_{k=1}^\infty Ax_k\stackrel{wrt\norm{\cdot}_Y}{\longrightarrow{}}y,\,n\to\infty
	$$
	By completeness of $X$, $\sum_{k=1}^\infty x_k\stackrel{def.}{=}x$ exists, $x\in B_X(1)$ and by continuity of $A$,
	$$Ax=\sum_{k=1}^\infty Ax_k=y$$
	By lemma above,
	$$\forall s>0, B_Y(sr)\subset \overline{A(B_X(s))}\,\,(*)$$
	$s=1/2$. Pick $x_1\in B_X(1/2)$ s.t. $\norm{Ax_1-y}<r/2$. Now set $y_1=y-Ax(\in B_X(r/2)$. Iterate. Assume that for some $\geq 1$ have $x_1,......x_k,y_1,......y_k$ s.t.
	$$
		\forall 1\leq\tilde{k}\leq k:\,\,\norm{\tilde{x_k}}_X<2^{-k},\,y_{\tilde{k}}=y_{\tilde{k}-1}-Ax_{\tilde{k}}\in B_Y(2^{-\tilde{k}}r
	$$
	Then using $(*)$ with $s=2^{-(k+1)}$ find $x_{k+1}\in B_X(2^{-(k+1)})$ such that
	$$
		y_{k+1}\stackrel{def}{=}y_k-Ax_{k+1}\in B_Y(2^{-(k+1)}r
	$$
	This yields $\sum{k=1}^{\infty}\norm{x_k}_X<1$ and
	$$
		y-\sum_{k=1}^n Ax_k=y_1-\sum_{k=2}^n Ax_k=...=y_n\to0\,(n\to \infty)\quad\square
	$$
\end{pf}

\begin{example}[Equivalence of Norm]\label{Equinorm}\nl
	Let $X=Y$, with norms $\norm{\cdot}_1$ and $\norm{\cdot}_2$ and assume $\exists C>0$ s.t. $$\norm{x}_2\leq C\norm{x}_1,\,\forall x\in X\quad(1)$$
	If $X$ is complete, with respect to both $\norm{\cdot}_1$ and $\norm{\cdot}_2$ then consider $A=id:(X,\norm{\cdot}_1)\to(X,\norm{\cdot}_2)$ is open by Theorem (indeed thm applies $b/c$ $A$ is bounded by $(1)$. Since $A$ is bijective, ii) gives that $A^{-1}=id:(X,\norm{\cdot}_2)\to(X,\norm{\cdot}_1)$ is bounded, i.e.
	$$
		\exists C': \norm{A^{-1}}_1=\norm{x}_1\leq C'\norm{x}_2
	$$
	so $\norm{\cdot}_1$ and $\norm{\cdot}_2$ are actually equivalent.
\end{example}

\begin{example}[Completeness of $Y$]\nl
	Consider $X=C(=C^0[0,1])$ with $\norm{\cdot}_1=\norm{\cdot}_\infty$, $\norm{\cdot}_2=\norm{\cdot}_{L^1}$. Then $A=id:(X,\norm{\cdot}_1)\to(X,\norm{\cdot}_2)$ is continuous:
	$$
		\norm{Af}_2
		=\norm{f}_2
		=\int_0^1|f(t)|dt
		\leq\norm{f}_\infty
		=\norm{f}_1
	$$
	but not open. Else by 1),  $\norm{\cdot}_1$ and $\norm{\cdot}_2$ would be equivalent. However consider counter example:
	\begin{equation}\nonumber
		f_n(x)=\left\{
		\begin{aligned}
			 & {2n^2x}     & x\in[0,\frac{1}{2n}]           \\
			 & {-2n^2x+2n} & x\in(\frac{1}{2n},\frac{1}{n}] \\
			 & 0           & x\in(\frac{1}{n},1]            \\
		\end{aligned}
		\right.\quad\text{satisfy}\quad\norm{f_n}_2=1,\norm{f_n}_1=n\to\infty
	\end{equation}
	This shows $Y$ needs to be complete in theorem.
\end{example}
\begin{example}[Completeness of $X$]\nl
	This example shows completness of $X$ is also required.
	Take
	$$
		X=Y=\{(x_n)\in\ell^\infty:\exists N:x_n=0\,\forall m\geq N\}\subset\ell^\infty
	$$
	with norm $\norm{\cdot}_X=\norm{\cdot}_Y=\norm{\cdot}_\infty$. This is a linear normed space. It's not complete (Exercise: show directly $\overline{X}=c_0$). Another way:
	Define \func{A}{X}{X},
	$$
		Ax=(x_1,\frac{x_2}{2},\frac{x_3}{3}\underbrace{......}_{0\,eventually})\quad if \,\,x=(x_1,x_2......)
	$$
	Then $A$ is linear, bijective with
	$$
		A^{-1}:X\to ,\,\,\,A^{-1}x=(x_1,2x_2,3x_3\underbrace{......}_{0\,eventually})
	$$
	and $A$ is bounded.
	$$
		\norm{Ax}_\infty=\sup_{n\geq1}\frac{|x_n|}{n}\leq\sup_{n\geq1}|x_n|=\norm{x}_\infty
	$$
	so $\norm{A}\leq1$. But $A^{-1}$ is unbounded.
	Pick $x^{(n)}=(\overbrace{1,1,1,1}^{n},0,......)$ then $\norm{x^{(n)}}_\infty=1$ but $\norm{A^{-1}x^{(n)}}=n$. Hence $A^{-1}\not\in\mathcal{L}(X)$ and $X$ cannot be complete, else by theorem i), $A^{-1}$ would be bounded.
\end{example}



\subsection{Closed Graph Theorem}

\begin{definition}[Graph,closed graph and closed operator]\label{Closed Operator}\nl
	Let $X$, $Y$ be normed spaces. Linear operator $T$ defined on $D(T)\subset X$, linear subspace of $X$. Graph of $T$ is $\Gamma (T)=\{(x,Tx):x\in D(T)\}$. If $\Gamma (T)$ is closed in $(X\times Y,\norm{\cdot}_{X\times Y}$, where $\norm{\cdot}_{X\times Y}$ can be set to $max(\norm{\cdot}_{X},\norm{\cdot}_{Y)}$ or $\norm{\cdot}_{X}+\norm{\cdot}_{Y}$, we say $\Gamma T$ is closed, and $T$ is called closed operator.
\end{definition}





\begin{theorem}[Closed Graph Theorem]\label{CGT}\nl
	Let $X$, $Y$ be Banach. $T$ an linear operator \func {T}{X}{Y}. The following two statements are equivalent:
	\begin{itemize}
		\item $T$ is closed.
		\item $T$ is continuous.
	\end{itemize}
	\begin{pf}{$bounded\implies closed$}{}
		We should check that limit of a convergent sequence sequence in $\Gamma(T)$ is inside $\Gamma(T)$.
		By
		\ref{Completeness of product of Banach spaces} 
		$X\times Y$ is complete.
		Since $T$ is continuous, consider $\sequ{a_n}\subset X\times Y$ where $a_n=(x_n,Tx_n)$ with $x_n\to x\in X$. By continuity of $T$ and completeness of $Y$ we have that $Ax_n\to y=Ax\in T(X)\subset Y$. Then we have that
		$$\norm{(x,Tx)-(x_n,Tx_n)}_{X \times Y}=\underbrace{\norm{x-x_n}_{X}}_{\to 0}+\underbrace{\norm{y-Tx_n}}_{\to 0}\to0\quad \text{as  }n\to0$$
		This means that $a_n\to(x,Tx)\in \Gamma(T)$. Thus the graph is closed, hence $T$ is closed.
	\end{pf}
	\begin{pf}{$closed\implies bounded$}{}
		Idea of this part of the proof is recover $T$ by the partly inverting the map from $(X,Y)$ to $\Gamma(T)$.
		Consider two linear maps:
		$$
			\Pi_A():\Gamma(T)\to X,\,\Pi_A((x,Tx))=x\qquad \Pi_B():\Gamma(T)\to Y,\,\Pi_A((x,Tx))=Ax
		$$
		We can use sequential continuity to check that both maps are continuous. Moreover, $\Pi_A$ is  surjective and injective from its definition, thus bijective. By open mapping theorem, we have that $\exists {\Pi_A}^{-1}\in\mathcal{L}(X,\Gamma(T))$, which means that ${\Pi_A}^{-1}$ is continuous. Thus we have
		$$
			T=\underbrace{{\Pi_A}^{-1}}_{conti.}\circ\underbrace{{\Pi_B}}_{conti.}
		$$
		Is also countinuous.
	\end{pf}
\end{theorem}

\begin{example}[Space not Banach]\nl
	\placeholder
\end{example}


\begin{example}[Inverse exists, not continuous]\nl
	\placeholder
\end{example}