\newpage
\section{Operator and Functional}
\subsection{Linear functionals}\label{Linear Functionals}
Linear functionals are a special class of linear mapping from a vector space to its scalar field.

\begin{definition}[Linear mapping]\rm\nextline
	Let $X$ and $Y$ be vector spaces over the same scalar field. A mapping $\mathrm{\Lambda}:X\xrightarrow{}Y$ is called a {\bf linear mapping} if
	$$
		\Lambda({\alpha x+\beta y})=\alpha\Lambda{x}+\beta \Lambda{y}
	$$
	for all $x,y$ vectors and $\alpha,\beta $ scalars.\\
	When Y is the scalar field, $\Lambda$ is called a {\bf{linear functional}}. In the following text, we assume $Y=\mathbb{C}$ unless mentioned otherwise.
\end{definition}

\begin{definition}[Image, preimage, kernel]\rm\nextline
	Let $T$ be linear mapping :$T:X\xrightarrow{}Y$, let $M\subset X$, $N\subset Y$.\\
	Then the {\bf Image} of X is
	$$
		T(M)\equiv \{y\in Y: y=T(x),x\in M\}\subset Y
	$$
	The {\bf Preimage} of Y is
	$$
		T^{-1}(Y)\equiv \{x\in X: y=T(x),y\in N\}\subset X
	$$
	Specially, {\bf Kernel} of $T$ is defined to be the preimage of $\{0_Y\}$,a set containing only the null-element of Y ($0_Y$ is a scalar 0 when $Y$ is the scalar field:
	$$
		\text{Ker}(Y)\equiv T^{-1}(\{0_Y\})=\{x\in X: T(x)=0_Y,y\in N\}\subset X
	$$
\end{definition}

\subsubsection{Boundedness and continuity}
In this section we shall see a very nice result about linear functional, which links their boundedness and continuity

\begin{definition}[Boundedness]\rm\nextline
	Let $X$ be a nomred vector space. $\Lambda:X\xrightarrow{}\mathbb{C}$ a linear functional. Then $\Lambda$ is {\bf bounded} if there exists a constant $M$, such that for all $x\in X$, we have $$\norm{\Lambda(x)}\leq M\norm{x}$$


\end{definition}

\begin{definition}[Functional norm]\rm\nextline
	For a bounded linear functional $\Lambda:X\xrightarrow{}\mathbb{C}$, we define its norm as follows:

	$$
		\norm{\Lambda}=\sup_{x\in X}{\left\{\norm{\Lambda(x)},\norm{x}\leq1\right\}}
	$$
\end{definition}

\begin{proposition}[An inequality of no name]\rm\nextline
	Let $X$ be a nomred vector space. $\Lambda:X\xrightarrow{}\mathbb{C}$ a bounded linear functional.
	Then
	$$
		\norm{\Lambda(x)}\leq\norm{\Lambda}\norm{x}
	$$
	\textit{proof}:\\
	Let $e\in X$ be such that $\norm{e}=1$.
	Then
	\begin{equation}
		\begin{split}
			\norm{\Lambda}\norm{e}&=\sup_{x\in X}{\left\{\norm{\Lambda(x)},\norm{x}\leq1\right\}}\norm{e}\\
			&\leq \norm{\Lambda(e)}\norm{e}\\
			&=\norm{\Lambda(e)}
		\end{split}
	\end{equation}
	Then for arbitrary non-zero vector $x\in\dual X$ we write $x=\norm{x}\frac{x}{\norm{x}}$ and exploit linearity to finish the proof.
\end{proposition}
Definition for continuous linear functional is similar to the that of continuous function in real-analysis, changing absolute value to corresponding norms. It is a routine exercise to check that this indeed defines a norm.

\begin{definition}[Continuous linear functional]\rm\nextline
	Let $\Lambda:X\xrightarrow{}\mathbb{C}$ be a linear functional. It is continuous at $y\in X$ if for all $\varepsilon>0$, there exists $\delta>0$ such that whenever $x\in\{x\in X, \norm{x-y}<\delta\}$, we have $\norm{\Lambda(x)-\Lambda(y)}<\varepsilon$. If $\Lambda$ is continuous at all $y\in X$, then it is called a {\bf continuous linear map}.


\end{definition}

\begin{proposition}[Continuity $\Longleftrightarrow$ Continuity at 0]\rm\nextline
	Let $\Lambda:X\xrightarrow{}\mathbb{C}$ be a linear functional. Then $\lambda$ is continuous if and only if it is continuous at 0. Proof of this proposition is trivial, left as an exercise.

\end{proposition}

\begin{proposition}[Continuity $\Longleftrightarrow$ Boundedness]\rm\nextline
	Let $\Lambda:X\xrightarrow{}\mathbb{C}$ be a linear functional. Then $\lambda$ is continuous if and only if it is bounded.\\
	\textit{proof:}\\
	$(\Longrightarrow)$We first show that continuity implies boundedness. We know $\Lambda$ is continuous at 0, so fix $\varepsilon>0$, we may find $\delta_0>0$ such that $\norm{x}<\delta_0\Longrightarrow \norm{\Lambda(x)}<\varepsilon$. So for any $y\in X$ we have that:
	\begin{equation}
		\begin{split}
			\Lambda(y)&=\Lambda\left(\frac{2\norm{y}}{\delta_0}\left(\frac{\delta_0}{2}\frac{y}{\norm{y}}\right)\right)\\
			&=\frac{2\norm{y}}{\delta_0}\Lambda\left(\frac{\delta_0}{2}\frac{y}{\norm{y}}\right)\\
			&\leq\frac{2\norm{y}}{\delta_0}\varepsilon=\frac{2\varepsilon}{\delta_0}\norm{y}
		\end{split}
	\end{equation}
	Hence $M=2\varepsilon/\delta_0$ gives a bound of the functional.\\
	$(\Longleftarrow)$Now we show that boundedness implies continuity. By last proposition we may reduce this to showing that boundedness implies continuity at 0. Let $K$ be such that
	$$
		\Lambda(x) \leq K\norm{x},\quad\forall x\in X
	$$
	Now fix $\varepsilon>0$, let $\delta=\frac{\varepsilon}{2K}$. Then when $x<\delta$, we have
	\begin{equation}
		\begin{split}
			\Lambda(x)&\leq K\norm{x}\\
			&< K \delta\\
			&= K \frac{\varepsilon}{2K}\\&
			=\frac{\varepsilon}{2}<\varepsilon\\
		\end{split}
	\end{equation}
	Hence $\Lambda$ is continuous at 0, which finishes the proof.
\end{proposition}

\begin{proposition}[Sequential continuity]\rm
	Let $\Lambda:f\xrightarrow{}Y$ be a mapping, $X$ is a metric space and $Y$ is topological space. Then it is continuous if and only if, for any sequence $\{x_n\}$ in $X$ such that $x_n\longrightarrow{}x_0\in X$, we have $f(x)\longrightarrow{}f(x_0)\in Y$
\end{proposition}

\newpage
\subsection{Basic Operator Theory}
\subsubsection{Bounded linear operator}
\begin{definition}[Linear operator]\rm\nextline
	Let $X$, $Y$ be normed vector spaces. A mapping $\Gamma:X\xrightarrow{}Y$ is a linear operator if
	$$
		\Gamma(k_1x_1+k_2x_2)=k_1\Gamma(x_1)+k_2\Gamma(x_2)
	$$
	for all $x_1,x_2\in X$ and $k_1,k_2$ scalars.
\end{definition}

\begin{definition}[Bounded linear operator]\label{continuity of LO}\rm\nextline
	Let $X$, $Y$ be normed vector spaces. Linear operator $\Gamma:X\xrightarrow{}Y$ is bounded if there is a finite constant $C>0$ such that
	$$
		\norm{\Gamma(x)}_Y\leq\norm{x}_X
	$$
	holds for all $x\in X$.
\end{definition}

\begin{definition}[Operator norm]\label{operator norm}\rm\nextline
	Let $X$, $Y$ be normed vector spaces, $\Gamma:X\xrightarrow{}Y$ a bounded linear operator, then

	$$
		\norm{\Gamma}\equiv\sup_{x\in X}{\left\{\norm{\Gamma(x)}_Y,\norm{x}_X\leq1\right\}}
	$$
	The definition is very similar to that for functionals.
\end{definition}

\begin{proposition}[Boundedness and continuity]\rm\nextline
	Let $X$, $Y$ be normed vector spaces, $\Gamma:X\xrightarrow{}Y$ a linear operator. Then the following three are equivalent:
	\begin{itemize}
		\item $\Gamma$ is bounded
		\item $\Gamma$ is continuous
		\item $\Gamma$ is continuous at 0
		\item {\bf Also:} $\Gamma$ is Lipschitz, i.e.$\exists C>0$ with $\norm{Aa-Ab}_Y\leq C\norm{a-b}_X,\,\,\forall a,b\in X$
		\item {\bf Also:} $\Gamma$ is continuous at any $x\in X$

	\end{itemize}
\end{proposition}

\begin{definition}[$\mathscr B$]\rm\nextline
	We define $\mathscr B(X,Y)$ to be the collection of all bounded linear operators from $X$ to $Y$. We also write $\mathscr B(X,X)$ as $\mathscr B(X)$
\end{definition}

\begin{proposition}\rm\nextline
	$\mathscr B(X,Y)$ is normed linear space in operator norm. Specially if $Y$ is Banach then $\mathscr B(X,Y)$ is also Banach.
\end{proposition}

\begin{theorem}\rm\nextline
	Every finite dimensional linear operator is bounded.
\end{theorem}

\begin{theorem}\rm\nextline
	A bounded linear operator attains its inf and sup on  a compact set.
\end{theorem}

%\subsubsection{Adjoints of Hilbert space operators.}
\subsection{Duality}

\subsubsection{Dual space: A nice self-symmetry}
One may notice, that linear maps also form a vector space: multiple of a linear map are also linear, sum of linear maps are also linear. We'll formalize this idea to the concept of dual space.

\begin{definition}[Dual Space]\rm\nextline
	Let $X$ be a Banach space. Define its dual space $X^*$ as follows:
	$$
		X^*=\left\{
		T: T\text{ is a bounded linear functional on } X
		\right\}
	$$
\end{definition}
\begin{proposition}[Dual space of Banach Space]\label{dual space Banach}\rm\nextline
	Dual space of a Banach space is also Banach, under functional norm.\\
	\textit{proof:}\\
	Here we assume scalar field to be $\mathbb{C}$.\\
	First,to check that $\dual{X}$ is a vector space, it suffices to show that multiple and sum of bounded linear functional remains to be linear and bounded. This part of proof is trivial.\\
	Second, we shall check that $\dual X$ is complete under functional norm.
	To show this, we let $\{T_n\}$ to be a Cauchy sequence in $\dual X$, which means given $\varepsilon>0$, there exists a positive constant $N$ such that for all $a,b>N$, we have
	$$
		\norm{T_a-T_b}<\varepsilon
	$$
	So given any point $x\in X$, we have that
	\begin{equation}
		\begin{split}
			\norm{T_a(x)-T_b(x)}&=\norm{(T_a-T_b)x}\\
			&\leq \norm{T_a-T_b}\norm{x}\\
			&<\varepsilon\norm{x}
		\end{split}
	\end{equation}
	This implies that $\{T_n(x)\}$ is a Cauchy sequence on $\mathbb{C}$. By completeness of $\mathbb{C}$, it converges to a point on $\mathbb{C}$. Note that this works for arbitrary $x\in X$, we can define $T$ to be the pointwise limit in following form:
	$$
		T(x)\equiv\lim_{n\to\infty}{T_n(x)},\quad\forall x\in X
	$$
	We can verify that $T$ is linear:
	\begin{equation}
		\begin{split}
			T(ax)&=\lim_{n\to\infty}{T_n(ax)}\\
			&=\lim_{n\to\infty}{aT_n(x)}\\
			&=a\lim_{n\to\infty}{T_n(x)}\\
			&=aT(x)
		\end{split}
	\end{equation}

	\begin{equation}
		\begin{split}
			T(x)+T(y)&=\lim_{n\to\infty}{T_n(x)}+\lim_{n\to\infty}{T_n(y)} \\
			&=\lim_{n\to\infty}{T_n(x)+T_n(y)}\\
			&=\lim_{n\to\infty}{T_n(x+y)}\\
			&=T(x+y)
		\end{split}
	\end{equation}

	It remains to show that $T$ is bounded. Fix $r>0$, we can find $a\in\mathbb{N}$ such that $\norm{T-T_a}<r$,
	\begin{equation}
		\begin{split}
			\norm{T}&=\norm{T-T_a+T_a}\\
			&\leq\norm{T-T_a}+\norm{T_a}\\
			&=r+\norm{T_a}
		\end{split}
	\end{equation}
	Hence T is also bounded, so $T\in \dual X$, which finishes the proof for completeness.
\end{proposition}



We have shown that dual space of a Banach space is Banach, what about dual space of a Hilbert space? In the following result, we'll see that the inner product of a Hilbert space allows great elegance in structure of its dual space, which is identity in the sense of isomorphism.

\subsubsection{Riesz representation Theorem}\label{Riesz representation theory}
\begin{theorem}[Riesz representation Theorem]\rm
	A bounded linear functional $T$ on Hilbert space \hbs is uniquely associated with a vector $h_0\in \hbp$ in the sense that
	$$
		T(h)=\inne{h}{h_0},\quad\forall h\in\hbp\quad\text{ and  }\norm{T}=\norm{h}
	$$
	\textit{proof}:\\
	If $T=0$ we simply choose $h_0=0$.\\
	When $T\neq0$,let $S=Ker(T)$, pick a non-zero vector $w\in S^\perp$, without loss of generality, we may assume $T(w)=1$. Then, for any $x\in \hbp$, we observe for vector $(T(x)w-x)$ that
	$$
		T\left(T(x)w-x\right)=T(w)T(x)-T(x)=T(x)-T(x)=0
	$$
	Therefore $(T(x)w-x)\in S$, which means $(T(x)w-x)\perp w$. Hence
	$$
		\inne{(T(x)w-x)}{w}=0,\quad\forall w\in\hbp
	$$
	By (left) linearity of inner product,
	$$
		\inne{x}{w}=T(x)\inne{w}{w}=T(x)\norm{w}^2
	$$
	Thus
	$$
		T(x)=\frac{\inne{x}{w}}{\norm{w}}=\inne{x}{\frac{w}{\norm{w}^2}}
	$$
	So $h=w/\norm{w}^2$ is the desired vector. Also,
	\begin{equation}
		\begin{split}
			\norm{T}&=\sup_{x\in \hbp}{\left\{\norm{T(x)},\norm{x}\leq1\right\}}\\
			&=\sup_{x\in \hbp}{\left\{\norm{\inne{x}{h}},\norm{x}\leq1\right\}}\\
			&=\norm{\inne{\frac{h}{\norm{h}}}{h}}\\
			&=\norm{h}
		\end{split}
	\end{equation}
	The result shows that \hbs=$\dual\hbp$ in the sense that the ma from $h$ to corresponding linear functional is an isometry: $\norm{h}=\norm{T}$
\end{theorem}
\begin{remark}[Alternative proof of Riesz Representation theorem]\rm\nextline
	The proof above is purely algebraic, in the sense that no analysis is used. However, one may realise that the proof relies heavily on the construction of $T(x)w-x$, which may not be that easy to thought of. Hence another completely different proof is presented here. \\
	\pf\\
	The idea of the proof is to construct a sequence of points that finally converges to the desired $h_0$.\\
	If $\norm{T}=0$, we simply choose $h_0=0$.\\
	If not, without loss of generality, let's consider operators with norm equal to 1.We first show existence of such $h_0$. Let $T\in\hbp^*$ be such that $\norm{T}=1$.\\
	Now by definition of operator norm, specializes to 1-Euclidean norm on $\real$:
	$$
		\norm{T}=\sup_{x\in\hbp}\frac{\norm{Tx}}{\norm{x}} =\sup_{\norm{x}=1}\norm{Tx}
	$$
	The existence of {\bf supremum} in the definition guarantees that we can find a sequence of points $(x_n)_0^\infty$ in $\{x\in\hbp:\norm{x}=1\}$ that gives
	$$
		\lim_{n\to\infty}{T(x_n)}=1
	$$
	The norm can be removed here as we're discussing functionals \func{T}{\hbp}{\real}, and for each $y\in\hbp$ if $T(y)<0$ there is $(-y)\in\hbp$ and $T(-y)=-T(y)>0$ by linearity.\\
	We claim that $(x_n)_1^\infty$ is Cauchy. To prove this, we first notice  parallelogram equality in Hilbert space:
	$$
		2\norm{a}^2+2\norm{b}^2=\norm{a+b}^2+\norm{a-b}^2
	$$
	we have that
	\begin{equation}
		\begin{split}
			\norm{x_m-x_n}^2&=2\norm{x_m}^2+2\norm{x_n}^2-\norm{x_m+x_n}^2\\
			&=2\times1+2\times1-\norm{x_m+x_n}^2\\
			&=4-\norm{x_m+x_n}^2\\
			&=4-\norm{T}^2\norm{x_m+x_n}^2\\
			&\leq4-\norm{Tx_m+Tx_n}^2\\
			&=\leq4-(Tx_m+Tx_n)^2
		\end{split}
	\end{equation}
	Taking $n$ and $m$ to infinity we have that $\norm{x_m-x_n}\to 0$, which means that $(x_n)_1^\infty$ is Cauchy.
\end{remark}
\begin{theorem}[Dual space of Hilbert space]\label{dual space Hilbert}\rm\nextline
	Let $\hbp$ be a Hilbert space, then its dual space $\hbp^*$ is isomorphic to  itself: $\hbp\cong\hbp^*$, moreover, the natural norm on them is an isometry.\placeholder
\end{theorem}

\subsubsection{Dual space of lp space}\label{lp dual}
In this section, we consider sequence spaces, $l_p$ space. Specially, when $p=2$, the space is Hilbert. In other cases,  when $p\not=2$, what happens? One may realise that $q=p=2$ is precisely a solution to $1/p+1/q=1$, and $p=q$ gives the self-duality. We will investigate this intuition here and prove that the dual space of $l_p$ is precisely $l_q$, where $1/p+1/q=1$,$p,q\in\real$. One should pay attention here, that $p=\infty$ is slightly different, the result may not hold in given $p=\infty$.

\begin{theorem}[Dual of \texorpdfstring{$\ell^p$}.]\rm\nextline
	Let $p\in (1,\infty)$, then $(\ell^p)^*=\ell^q $, where $1/p+1/q=1$.\\
	\pf\\
	Let 
\end{theorem}


\subsubsection{Dual operator}\label{dual operator}
